\documentclass[output=paper]{langsci/langscibook} 
\title{Basic properties and elements ( and Anne Abeillé) } 
\author{%
 Bob Borsley  \affiliation{Essex}
  Anne Abeillé \affiliation{Somewhere in Paris}
}
% \chapterDOI{} %will be filled in at production

\epigram{Change epigram in chapters/01.tex or remove it there }

\abstract{Change the  abstract in chapters/01.tex
Lorem ipsum dolor sit amet, consectetur adipiscing elit. Aenean nec fermentum risus, vehicula gravida magna. Sed dapibus venenatis scelerisque. In elementum dui bibendum, ultricies sem ac, placerat odio. Vivamus rutrum lacus eros, interdum scelerisque est euismod eget. Class aptent taciti sociosqu ad litora torquent per conubia nostra, per inceptos himenaeos.
}

\maketitle
\begin{document}

\section{Introduction} 
Phasellus maximus erat ligula, accumsan rutrum augue facilisis in. Proin sit amet pharetra nunc, sed maximus erat. Duis egestas mi eget purus venenatis vulputate vel quis nunc. Nullam volutpat facilisis tortor, vitae semper ligula dapibus sit amet. Suspendisse fringilla, quam sed laoreet maximus, ex ex placerat ipsum, porta ultrices mi risus et lectus. Maecenas vitae mauris condimentum justo fringilla sollicitudin. Fusce nec interdum ante. Curabitur tempus dui et orci convallis molestie \citep{Chomsky1957}.

\citet{Meier2017}
\ea\label{ex:1:descartes}
\langinfo{Latin}{}{personal knowledge}\\
\gll cogit-o ergo sum \\
     think-1{\sg}.{\prs}.{\ind} hence exist.1{\sg}.{\prs}.{\ind}\\
\glt `I think therefore I am'
\z

Sed nisi urna, dignissim sit amet posuere ut, luctus ac lectus. Fusce vel ornare nibh. Nullam non sapien in tortor hendrerit suscipit. Etiam sollicitudin nibh ligula. Praesent dictum gravida est eget maximus. Integer in felis id diam sodales accumsan at at turpis. Maecenas dignissim purus non libero scelerisque porttitor. Integer porttitor mauris ac nisi iaculis molestie. Sed nec imperdiet orci. Suspendisse sed fringilla elit, non varius elit. Sed varius nisi magna, at efficitur orci consectetur a. Cras consequat mi dui, et cursus lacus vehicula vitae. Pellentesque sit amet justo sed lectus luctus vehicula. Suspendisse placerat augue eget felis sagittis placerat. 


\begin{table}
\caption{Frequencies of word classes}
\label{tab:1:frequencies}
 \begin{tabular}{lllll} % add l for every additional column or remove as necessary
  \lsptoprule
            & nouns & verbs & adjectives & adverbs\\ %table header
  \midrule
  absolute  &   12 &    34  &    23     & 13\\
  relative  &   3.1 &   8.9 &    5.7    & 3.2\\
  \lspbottomrule
 \end{tabular}
\end{table}


\is{prolegomena}
Sed cursus \footnote{eros condimentum mi consectetur, ac consectetur} sapien pulvinar. Sed consequat, magna\footnote{eu scelerisque laoreet, ante erat tristique justo, nec cursus eros diam eu nisl. Vestibulum non arcu tellus}. Nunc dignissim tristique massa ut gravida. Nullam auctor orci gravida tellus egestas, vitae pharetra nisl porttitor. Pellentesque turpis nulla, venenatis id porttitor non, volutpat ut leo. Etiam hendrerit scelerisque luctus. Nam sed egestas est. Suspendisse potenti. Nunc vestibulum nec odio non laoreet. Proin lacinia nulla lectus, eu vehicula erat vehicula sed. 

\section*{Abbreviations}
\begin{tabularx}{.45\textwidth}{lX}
\textsc{cop} & copula\\ 
\textsc{fv} & final vowel\\
\end{tabularx}
\begin{tabularx}{.45\textwidth}{lX}
\textsc{neg} & negation\\ 
\textsc{sm} & subject marker\\
\end{tabularx}


\section*{Acknowledgements}

{\sloppy
\printbibliography[heading=subbibliography,notkeyword=this]
}
\end{document}
