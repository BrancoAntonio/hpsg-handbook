\documentclass[output=paper]{langsci/langscibook} 
\author{%
	Bob Borsley\affiliation{University of Essex}%
	\lastand Stefan Müller\affiliation{Humboldt-Universität zu Berlin}%
}
\title{HPSG and Minimalism}

% \chapterDOI{} %will be filled in at production

\maketitle

\begin{document}

\section{Introduction}
\label{sec:30-1}

The Minimalist\is{Minimalism|(} framework, which was first outlined by Chomsky in the early 1990s \citep{Chomsky93a-u,Chomsky95a-u}, still seems to be the dominant approach to syntax. It is important, therefore, to consider how HPSG compares with this framework. The issues are clouded by the rhetoric that surrounds the framework. At one time ‘virtual conceptual necessity’ was said to be its guiding principle. A little later, it was said to be concerned with the ``perfection of language'', with `how closely human language approaches an optimal solution to design conditions that the system must meet to be usable at all' \citet[58]{Chomsky2002a-u}. Much of this rhetoric seems designed to suggest that Minimalism is quite different from other approaches and should not be assessed in the same way. In the words of Postal \citet[19]{Postal2003a}, it looks like `an attempt to provide certain views with a sort of privileged status, with the goal of placing them at least rhetorically beyond the demands of serious argument or evidence'. However, the two frameworks have enough in common to allow meaningful comparisons.

Both frameworks seek to provide an account of what is and is not possible both in specific languages and in language in general. Moreover, both are concerned not just with local relations such as that between a head and its complement or complements but also with non-local relations such as those in the following:
\ea\label{ex:30-1}
The student knows the answer.
\z
\ea\label{ex:30-2}
It seems to be raining,
\z
\ea\label{ex:30-3}
Which student do you think knows the answer? 
\z
In (\ref{ex:30-3}), \textit{the student} is subject of \textit{thinks} and is responsible for the fact that \textit{thinks} is a third person singular form, but they are not sisters if \textit{knows} and \textit{the answer} form a VP. In (\ref{ex:30-2}) the subject is it because the complement of \textit{be} is \textit{raining}, but it and raining are obviously not sisters. Finally, in (\ref{ex:30-3}), \textit{which student} is understood as the subject of \textit{thinks} and is responsible for the fact that it is third person singular, but again the two elements are structurally quite far apart. Both frameworks provide analyses for these and other central syntactic phenomena, and it is quite reasonable to compare them and ask which is the more satisfactory.%
	\footnote{As noted below, comparison is complicated somewhat by the fact that Minimalists typically provides only sketches of analyses in which various details are left quite vague.}%

Although HPSG and Minimalism have enough in common to permit comparisons, there are obviously many differences. Some are more important than others, and some relate to the basic approach and outlook, while others concern the nature of grammatical systems and syntactic structures. In this chapter we will explore the full range of differences.

The chapter is organized as follows. In Section~\ref{sec:30-2}, we look at differences of approach between the two frameworks. Then in Section~\ref{sec:30-3}, we consider the quite different views of grammar that the two frameworks espouse, and in Section~\ref{sec:30-4}, we look at the very different syntactic structures which result. Finally, in Section~\ref{sec:30-5}, we will look at a further issue which deserves some attention.

\section{Differences of approach and outlook}
\label{sec:30-2}
As many of the chapters in this volume have emphasized, HPSG is a framework which places considerable emphasis on detailed formal analyses of the kind that one might expect within generative grammar. Thus, it is not uncommon to find lengthy appendices setting out formal analyses. See, for example, Sag's (\citeyear{Sag97a}) paper on English relative clauses and especially \citet{GSag2000a-u}, which has a 50 page appendix. One consequence of this, discussed in Chapter ??, is that HPSG has had considerable influence in computational linguistics.

In Minimalism things are very different. Detailed formal analyses are virtually non-existent. There appear to be no appendices like those in \citet{Sag97a} and \citet{GSag2000a-u}. In fact the importance of formalization has long been downplayed in Chomskyan work. Thus, in a 1980 conversation, Chomsky remarked that `I do not see any point in formalizing for the sake of formalizing' (see Huybregts and van Riemsdijk 1982: 73\todo{bibkey needed -- RF}), and this view seems fairly standard within Minimalism. Chomsky and Lasnik (1995: 28)\todo{bibkey needed; 1995? -- RF} attempt to justify the absence of detailed analyses when they suggest that providing a rule system from which some set of phenomena can be derived is not `a real result' since `it is often possible to devise one that will more or less work'. Instead, they say, `the task is now to show how the phenomena \ldots{} can be deduced from the invariant principles of UG with parameters set in one of the permissible ways'. In other words, providing detailed analyses is a job for unambitious drudges, and real linguists pursue a more ambitious agenda. \citet[5]{Postal2004a-u} comments that what we see here is `the fantastic and unsupported notion that descriptive success is not really that hard and so not of much importance'. He points out that if this were true, one would expect successful descriptions to be abundant within transformational frameworks. However, he suggests that `the actual descriptions in these frameworks so far are not only not successful but so bad as to hardly merit being taken seriously'. Postal does much to justify this assessment with detailed discussions of Chomskyan work on strong crossover phenomena and passives in chapters~\ref{chap:constituents} and \ref{chap:argumentstr} of his book.

\section{Different views of grammar}
\label{sec:30-3}

\section{Different views of syntactic structure}
\label{sec:30-4}

\section{Restrictiveness}
\label{sec:30-5}


\is{Minimalism|)}

\section*{Abbreviations}
\section*{Acknowledgements}

\printbibliography[heading=subbibliography,notkeyword=this] 
\end{document}
