\documentclass[output=paper]{langsci/langscibook} 
\author{%
	Bob Borsley\affiliation{University of Essex}%
	\lastand Stefan Müller\affiliation{Humboldt-Universität zu Berlin}%
}
\title{HPSG and Minimalism}

% \chapterDOI{} %will be filled in at production

\maketitle

\begin{document}

\section{Introduction}
\label{sec:min-intro}

The Minimalist framework, which was first outlined by Chomsky in the early 1990s, still seems to be the dominant approach to syntax. It is important, therefore, to consider how HPSG compares with this framework. The issues are clouded by the rhetoric that surrounds the framework. At one time `virtual conceptual necessity' was said to be its guiding principle. A little later, it was said to be concerned with the ``perfection of language'', with `how closely human language approaches an optimal solution to design conditions that the system must meet to be usable at all' \citet[58]{Chomsky2002a-u}. Much of this rhetoric seems designed to suggest that Minimalism is quite different from other approaches and should not be assessed in the same way. In the words of Postal \citet[19]{Postal2003a}, it looks like `an attempt to provide certain views with a sort of privileged status, with the goal of placing them at least rhetorically beyond the demands of serious argument or evidence'. However, the two frameworks have enough in common to allow meaningful comparisons.

Both frameworks seek to provide an account of what is and is not possible both in specific languages and in language in general. Moreover, both are concerned not just with local relations such as that between a head and its complement or complements but also with non-local relations such as those in the following:
\ea\label{ex:min-student-knows}
The student knows the answer.
\z
\ea\label{ex:min-raining}
It seems to be raining,
\z
\ea\label{ex:min-which-student}
Which student do you think knows the answer? 
\z
In (\ref{ex:min-student-knows}), \textit{the student} is subject of \textit{thinks} and is responsible for the fact that \textit{thinks} is a third person singular form, but they are not sisters if \textit{knows} and \textit{the answer} form a VP. In (\ref{ex:min-raining}) the subject is it because the complement of \textit{be} is \textit{raining}, but it and raining are obviously not sisters. Finally, in (\ref{ex:min-which-student}), \textit{which student} is understood as the subject of \textit{thinks} and is responsible for the fact that it is third person singular, but again the two elements are structurally quite far apart. Both frameworks provide analyses for these and other central syntactic phenomena, and it is quite reasonable to compare them and ask which is the more satisfactory.%
	\footnote{As noted below, comparison is complicated somewhat by the fact that Minimalists typically provides only sketches of analyses in which various details are left quite vague.}%

Although HPSG and Minimalism have enough in common to permit comparisons, there are obviously many differences. Some are more important than others, and some relate to the basic approach and outlook, while others concern the nature of grammatical systems and syntactic structures. In this chapter we will explore the full range of differences.

The chapter is organized as follows. In Section~\ref{sec:min-difference}, we look at differences of approach between the two frameworks. Then in Section~\ref{sec:min-views-grammar}, we consider the quite different views of grammar that the two frameworks espouse, and in Section~\ref{sec:min-views-structure}, we look at the very different syntactic structures which result. Finally, in Section~\ref{sec:min-restrictive}, we will look at a further issue which deserves some attention.

\section{Differences of approach and outlook}
\label{sec:min-difference}
As many of the chapters in this volume have emphasized, HPSG is a framework which places considerable emphasis on detailed formal analyses of the kind that one might expect within generative grammar. Thus, it is not uncommon to find lengthy appendices setting out formal analyses. See, for example, Sag's (\citeyear{Sag97a}) paper on English relative clauses and especially \citet{GSag2000a-u}, which has a 50 page appendix. One consequence of this, discussed in Chapter ??\todo{insert chapter reference -- RF}, is that HPSG has had considerable influence in computational linguistics.

In Minimalism things are very different. Detailed formal analyses are virtually non-existent. There appear to be no appendices like those in \citet{Sag97a} and \citet{GSag2000a-u}. In fact the importance of formalization has long been downplayed in Chomskyan work. Thus, in a 1980 conversation, Chomsky remarked that `I do not see any point in formalizing for the sake of formalizing' (see Huybregts and van Riemsdijk 1982: 73\todo{bibkey needed -- RF}), and this view seems fairly standard within Minimalism. Chomsky and Lasnik (1995: 28)\todo{bibkey needed; 1995? -- RF} attempt to justify the absence of detailed analyses when they suggest that providing a rule system from which some set of phenomena can be derived is not `a real result' since `it is often possible to devise one that will more or less work'. Instead, they say, `the task is now to show how the phenomena \ldots{} can be deduced from the invariant principles of UG with parameters set in one of the permissible ways'. In other words, providing detailed analyses is a job for unambitious drudges, and real linguists pursue a more ambitious agenda. \citet[5]{Postal2004a-u} comments that what we see here is `the fantastic and unsupported notion that descriptive success is not really that hard and so not of much importance'. He points out that if this were true, one would expect successful descriptions to be abundant within transformational frameworks. However, he suggests that `the actual descriptions in these frameworks so far are not only not successful but so bad as to hardly merit being taken seriously'. Postal does much to justify this assessment with detailed discussions of Chomskyan work on strong crossover phenomena and passives in chapters~\ref{chap:constituents} and \ref{chap:argumentstr} of his book.

There has also been a strong tendency to focus on just a subset of the facts in whatever domain is being investigated. As \citet[535]{CJ2005a} note, `much of the fine detail of traditional constructions has ceased to garner attention'. This tendency has sometimes been buttressed by a distinction between core grammar, which is supposedly a fairly straightforward reflection of the language faculty, and a periphery of marked constructions, which are of no great importance and which can reasonably be ignored. However, as \citet{Culicover99a-u} and others have argued, there is no evidence for a clear cut distinction between core and periphery. It follows that a satisfactory approach to grammar needs to account both for such core phenomena as \textit{wh}-interrogatives, relative clauses, and passives but also with more peripheral phenomena such as the following:
\eal
\ex It's amazing the people you see here.\label{ex:min-amazing-people}
\ex The more I read, the more I understand.\label{ex:min-read-understand}
\ex Chris lied his way into the meeting.\label{ex:min-chris-meeting}
\zl 
These exemplify the nominal extraposition construction (\citealt{ML96a}), the comparative correlative construction (Borsley 2011)\todo{bibkey needed -- RF}, and the \textit{X's Way} construction (\citealt{Sag2012a}). As has been emphasized in other chapters, the HPSG system of types and constraints is able to accommodate broad linguistic generalizations and highly idiosyncratic facts and everything in between.

The general absence in Minimalism of detailed formal analyses is quite important. It means that Minimalists may not be fully aware of the complexity of the structures they are committed to and allows them to sidestep the question whether it is really justified. It also allows them to avoid the question of whether the very simple conception of grammar that they favour is really satisfactory. Finally, it may be that they are unaware of how many phenomena remain unaccounted for. These are all important matters. 

The general absence of detailed formal analyses has also led to Minimalism having little impact on computational linguistics. There has been some work that has sought to implement Minimalist ideas, but Minimalism has not had anything like the productive relation with computational work that HPSG has enjoyed.

There are, then, issues about the quantity of data that is considered in Minimalist work. There are also issues about its quality. Research in HPSG is typically quite careful about data and often makes use of corpus and experimental data. Research in Minimalism is often rather less careful. In a review of a collection of Minimalist papers, \citet[434]{Bender2002a} comments that: `In these papers, the data appears to be collected in an off-hand, unsystematic way, with unconfirmed questionable judgments often used at crucial points in the argumentation'. She goes on to suggest that the framework encourages `lack of concern for the data, above and beyond what is unfortunately already the norm in formal syntax, because the connection between analysis and data is allowed to be remote.' Similar things could be said about a variety of Minimalist work. Consider, for example, Aoun and Li (2003)\todo{bibkey needed -- RF}, who argue for quite different analyses of \textit{that}-relatives and \textit{wh}-relatives on the basis of the following (supposed) contrasts, which appear to represent nothing more than their own judgements:
\eal
\ex[\phantom{??}]{The headway that Mel made was impressive.}\label{ex:min-headway-that}
\ex[??]{The headway which Mel made was impressive.}\label{ex:min-headway-which}
\zl
\eal
\ex[\phantom{?*}]{We admired the picture of himself that John painted in art class}\label{ex:min-admire-that}
\ex[\phantom{?}*]{We admired the picture of himself which John painted in art class}\label{ex:min-admire-which} 
\zl
\eal
\ex[\phantom{*?}]{The picture of himself that John painted in art class is impressive.}\label{ex:min-picture-that}
\ex[*?]{The picture of himself which John painted in art class is impressive.}\label{ex:min-picture-which} 
\zl
None of the native speakers we have consulted find significant contrasts here which could support different analyses. 

There are also differences in the kind of arguments that the two frameworks find acceptable. It is common within Minimalism to assume that some phenomenon which cannot be readily observed in some languages must be part of their grammatical system because it is clearly present in other languages. Notable examples would be case or agreement. This stems from the longstanding Chomskyan assumption that language is the realization of a complex innate language faculty. From this perspective, there is much in any grammatical system that is a reflection of the language faculty and not in any simple way of the observable phenomena of the language in question. If some phenomenon plays an important role in many languages it is viewed as a reflection of the language faculty, and hence it must be a feature of all grammatical systems even those in which it is hard to see any evidence for it. This line of argument would be reasonable if a complex innate language faculty was an established fact, but it isn't, and since \citet{HCF2002a}, it seems to have been rejected within Minimalism.  It follows that ideas about an innate language faculty should not be used to guide research on individual languages. Rather, as \citet[25]{MuellerCoreGram} puts it, `grammars should be motivated on a language-specific basis.' Does this mean that other languages are irrelevant when one investigating a specific language? Clearly not. As \Citeauthor{MuellerCoreGram} also puts it, `In situations where more than one analysis would be compatible with a given dataset for language X, the evidence from language Y with similar constructs is most welcome and can be used as evidence in favor of one of the two analyses for language X.' (\citeyear[43]{MuellerCoreGram}) In practice, any linguist working on a new language will use apparently similar phenomena in other languages as a starting point. It is important, however, to recognize that apparently similar phenomena may turn out on careful investigation to be significantly different.%
	\footnote{Equally, of course, apparently rather different phenomena may turn out on careful investigation to be quite similar. For further discussion of HPSG and comparative syntax, see Borsley (forthcoming).}
\section{Different views of grammar}
\label{sec:min-views-grammar}
We turn now to more substantive differences between HPSG and Minimalism, differences in their conceptions of grammar, especially syntax, and differences in their views of syntactic structure. As we will see, these differences are related. In this section we consider the former, and in the next we will look at the latter.

As has been emphasized throughout this volume, HPSG assumes a declarative or constraint-based view of grammar. It also assumes that the grammar involves a complex systems of types and constraints. Finally, it assumes that syntactic analyses complemented by separate semantic and morphological analyses. In each of these areas, Minimalism is different. It assumes a procedural view of grammar. It assumes that grammar involves just a few general operations. Finally, it assumes that semantics and morphology are simple reflections of syntax. We comment on each of these matters in the following paragraphs.

Whereas HPSG is a declarative or constraint-based approach, Minimalism seems to be firmly committed to a procedural approach. \citet[219]{Chomsky95a-u} remarks that: `We take L [a particular language] to be a generative procedure that constructs pairs (π, λ) that are interpreted at the articulatory-perceptual (A-P) and conceptual-intentional (C-I) interfaces, respectively, as ``instructions'' to the performance systems'. Various arguments have been presented within HPSG for a declarative view, but no argument seems to be offered within Minimalism for a procedural view. Obviously, speakers and hearers do construct representations and must have procedures that enable them to do so, but this is a matter of performance, and there is no reason to think that the knowledge that is used in performance has a procedural character. Rather, the fact that it used in both production and comprehension suggests that it should be neutral between the two and hence declarative. For further discussion of the issues, see e.\,g.\ \citet{PS2001a}, \citet{Postal2003a} and \citeauthor{SW2011a} (\citeyear{SW2011a,SW2015a}).

The declarative-procedural contrast is an important one, but the contrast between the complex systems of types and constraints that are assumed within HPSG and the few general operations that form a Minimalist grammar is arguably more important.%
	\footnote{A procedural approach doesn't necessarily involve a very simple grammatical system. The Standard Theory of transformational grammar \citep{Chomsky65a} is procedural but has many different rules, both phrase structure rules and transformations.}
Much work in Minimalism has three main operations Merge, Agree, and Move or Internal Merge. Merge combines two expressions, either words or phrases, to form a larger expression with the same label as one of the expressions \citep[244]{Chomsky95a-u}. Its operation can be presented as follows:
\begin{figure}[h!]
		\centering
	\raisebox{1\baselineskip}{X, Y $\Rightarrow$}
	\hspace{1em}
	\begin{forest}
		[X [X] [Y]]
	\end{forest}
\hspace{1em}
\raisebox{1\baselineskip}{or}
\hspace{1em}
	\begin{forest}
		[Y [X] [Y]]
	\end{forest}
	\caption{\label{fig:min-merge}insert caption}
\end{figure}

\noindent In the case of English, the first alternative is represented by situations where a lexical head combines with a complement while the second is represented by situations where a specifier combines with a phrasal head.

Agree, as one might suppose, offers an approach to various kinds of agreement phenomena. It involves a probe, which is a feature or features of some kind on head, and a goal, which the head c-commands. At least normally, the probe is an uninterpretable feature or features with no value and the goal has a matching interpretable feature or features with appropriate values. Agree values the uninterpretable feature or features and they are ultimately deleted, commonly after they have triggered some morphological effect. Agree can be represented as follows (where the `\textit{u}' prefix identifies a feature as uninterpretable.):%
	\footnote{On standard assumptions, the goal also has some uninterpretable feature, which renders it `active', i.\,e.\ capable of undergoing Agree. This is ultimately deleted, possibly after they have triggered some morphological effect.}
\begin{figure}
	\centering
\begin{forest} sn edges, empty nodes
	[{}
	[X \\ {[\textit{u}F]}]
	[{},tier=flat
	[Y \\ {[F v]}, roof]]]
\end{forest}
\end{figure}	
	
\section{Different views of syntactic structure}
\label{sec:min-views-structure}

\section{Restrictiveness}
\label{sec:min-restrictive}

\section*{Abbreviations}
\section*{Acknowledgements}

\printbibliography[heading=subbibliography,notkeyword=this] 
\end{document}