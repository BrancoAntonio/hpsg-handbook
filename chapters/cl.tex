\documentclass[output=paper]{langsci/langscibook} 
\author{%
	Emily M.\ Bender\affiliation{University of Washington}%
	\lastand Guy Emerson\affiliation{University of Cambridge}
}
\title{Computational linguistics and grammar engineering}

% \chapterDOI{} %will be filled in at production

\abstract{%
	We discuss the relevance of HPSG for computational linguistics.
	% TODO
}

\maketitle

\begin{document}
\label{chap-cl}

% Comments from Gerald:

%I assume that the contributors would agree with me that this handbook is probably the wrong place to go into great technical detail on mathematical or computational subjects, but there needs to be at least something there to serve as a starting point from which the reader could choose to jump into some of the papers cited.‎ A summary of the major trends and the high-level intuitions behind a lot of the technical terminology that readers will encounter in HPSG-related CL papers, for example, would both be most welcome.

\section{Introduction}

% First draft: Emily

From the inception of HPSG in the 1980s,
there has been a close integration between theoretical and computational work
\citep{FIXME-CLobit-or-other}.
% at HP Labs
% in the joint work of Ivan Sag, Geoff Pullum, Tom Wasow, Mark Gawron, Carl Pollard and Dan Flickinger
% GE: I think this list of authors is too long for a first sentence.  Their names would be in a citation?
In this chapter, we give an overview of computational work in HPSG, starting with the infrastructure that supports it (both theoretical and practical) in \S\ref{sec:infrastructure}. Next we describe several existing large-scale projects which build HPSG or HPSG-inspired grammars (\S\ref{sec:resources}) and the deployment of such grammars in applications including both those within linguistic research and otherwise (\S\ref{sec:deployment}).  Finally, we turn to linguistic insights gleaned from broad-coverage grammar development.

% EMB 2018-07-26 That seems like a relatively weak intro, but putting it in as a place-holder for now. 

% EMB 2018-07-26 Also, I wonder if ``lessons for linguistics'' might not come across as condescending...
% GE 2018-07-31 Yes, maybe best to reword that somehow... ``Linguistic insights''?

\section{Infrastructure}
\label{sec:infrastructure}

% First draft: By subsection

\subsection{Theoretical considerations}
\label{sec:theoretical}

There are several properties of HPSG as a theory that make it well-suited to computational implementation. First, the theory is kept separate from the formalism: the formalism is expressive enough to encode a wide variety of possible theories. While some theoretical work does argue for or against the necessity of particular formal devices (e.g.\ the shuffle operator \citep{FIXME-Reape}), much of it proceeds within shared assumptions about the formalism. This is in contrast to work in the context of the Minimalist Program \citep{Chomsky93b-u}, where theoretical results are typically couched in terms of modifications to the formalism itself. From a computational point of view, the benefit of differentiating between theory and formalism is that it means that the formalism is relatively stable. That in turns enables the development and maintenance of software systems that target the formalism, for parsing, generation, and grammar exploration (see \S\ref{sec:history} below for some examples).\footnote{There are implementations of Minimalism, notably \citet{FIXME-Stabler} and \citet{FIXME-Indianadiss}. However, writing an implementation requires fixing the formalism, and so these are unlikely to be useful for testing theoretical ideas as the theory moves on.}

A second important property of HPSG that supports a strong connection between theoretical and computational work is an interest in both so-called `core' and so-called `peripheral' phenomena. Most implemented grammars are built with the goal of handling naturally occurring text.\footnote{Though it is possible to do implementation work strictly against testsuites of sentences constructed specifically to focus on phenomena of interest.} This means that they will need to handle a wide variety of linguistic phenomena not always treated in theoretical syntactic work \citep{FIXME-Baldwin-et-al-Beauty}. A syntactic framework that excludes research on `peripheral' phenomena as uninteresting provides less support for implementational work than does one, like HPSG or Construction Grammar \citep{FIXME}, that values such topics.

Finally, the type hierarchy characteristic of HPSG lends itself well to developing broad-coverage grammars which are maintainable over time \citep{FIXME-find-cite?}. The use of the type hierarchy to manage complexity at scale comes out of the work of \citet{Flickinger87} and others at HP labs in the project where HPSG was originally developed. The core idea is that any given constraint is (ideally) expressed only once on types which serve as supertypes to all entities that bear that constraint.\footnote{Originally this only applied to lexical entries in Flickinger's work. Now it also applies phrase structure rules, lexical rules, and types below the level of the sign which are used in the definition of all of these.} Such constraints might represent broad generalizations that apply to many entities or relatively narrow, indiosyncratic properties. By isolating any given constraint on one type (as opposed to repeating it in mutiple places), we build grammars that are easier to update and adapt in light of new data that require refinements to constraints. Having a single locus for each constraint also makes the types a very useful target for documentation \citep{FIXME:LTDB} and grammar exploration \citep{FIXME:typediff}. 


\subsection{Practical considerations}
\label{sec:practical}

The formalism of HPSG allows practical implementations,
since feature structures are well-defined data structures.
Furthermore, because HPSG is defined to be bi-directional,
an implemented grammar can be used for both parsing and generation.
This allows HPSG grammars to be used in a range of NLP applications,
as we will discuss in~\S\ref{sec:deployment}.

\subsubsection{Computational complexity}

One way to measure how easy or difficult it is to use a syntactic theory
is to consider the \textit{computational complexity} of parsing and generation algorithms.
For example, we can consider how much computational time
a parsing algorithm needs to process a particular sentence.
For longer sentences, we would expect the amount of time to increase,
but the more complex the algorithm,
the more quickly the amount of time increases.
If we consider sentences containing $n$~tokens,
we can find the average amount of time taken,
or the longest amount of time.
We can then increase~$n$, and see how the amount of time changes,
both in the average case, and in the worst case.

At first sight, analysing computational complexity
would seem to paint HPSG in a bad light,
because the formalism allows us to write grammars
with any possible computational complexity
(the formalism can be called \textit{Turing-complete}).
However, as discussed in the previous section,
there is a clear distinction between theory and formalism.
Although the feature-structure formalism does not allow efficient algorithms
that could cope with any possible grammar,
a particular theory (or a particular grammar) might well allow efficient algorithms.

The difference between theory and formalism
becomes clear when comparing HPSG to other computationally-friendly frameworks,
such as Combinatory Categorial Grammar (CCG),\footnote{%
	For an introduction, see: \citealp{steedman2011ccg}.
	See also Kubota 2018, this volume. %TODO cite
}
or Tree Adjoining Grammar (TAG; \citealp{FIXME-Joshi}). % TODO historical note on influences between frameworks?
The formalisms of CCG and TAG inherently limit computational complexity:
for both of them, as the sentence length~$n$ increases,
worst-case parsing time increases proportional to~$n^6$.
This is a deliberate feature of these formalisms,
which aim to be just expressive enough to capture human language,
and not any more expressive.
Building this kind of constraint into the formalism itself
highlights a different way of thinking from HPSG.
As discussed above, separating formalism from theory
means that the formalism is stable, even as the theory develops.

%TODO discuss:
% - CFG backbone without recursive unary rules -> worst-case exponential time
% - grammar writers adapting to complexity
% - Delph-in grammars in practice have better average-case complexity than might be expected




\subsubsection{Parse ranking}

% GE: not sure if this should be here or go in a later section

For an ambiguous sentence,
a grammar gives multiple valid parses.
In practical applications, considering all possible parses can be infeasible,
and we may want to automatically disambiguate each sentence.
This can be done by \textit{ranking} the parses,
so that the application only uses the most highly-ranked parse,
or the top~$N$ parses.

Parse ranking is not usually determined by the grammar itself,
because of the difficulty of manually writing disambiguation rules.
Typically, a statistical system is used.
First, a corpus is \textit{treebanked}:
for each sentence in the corpus,
an annotator (often the grammar writer) chooses the best parse,
out of all parses produced by the grammar.
The set of all parses for a sentence is often referred to as the \textit{parse forest},
and the selected best parse is often referred to as the \textit{gold standard}.
Given the gold parses for the whole corpus, a statistical system is trained
to predict the gold parse from a parse forest.

In practical applications, a grammar is rarely used alone,
but rather in combination with a statistical parser (or statistical generator) trained on a treebank.

%TODO references...

Because of the large number of possible parses,
it can be helpful to \textit{prune} the search space:
rather than ranking the full set of parses,
we can restrict attention to a smaller set of parses,
which hopefully includes the correct parse.
By carefully choosing how to restrict our attention,
we can drastically reduce processing time without hurting performance.

Ubertagging.


\subsubsection{Semantic dependencies}

In practical applications of HPSG grammars,
the full derivation trees and the full feature structures are often seen as unwieldy,
containing far more information than necessary for the task at hand.
It is often desirable to extract a concise semantic representation.

Semantic dependency graphs.
Popular and easy to work with.
DMRS, EDS, DM.

(See also Hudson, 2018, this volume...)



\subsection{A brief history of HPSG grammar engineering}
\label{sec:history}

History: PAGE, VerbMobil, ?? % First draft: Emily

Current platforms:
    \begin{itemize}
    \item LKB/ACE/PET/Agree
    \item Trale
    \item Other
    \end{itemize}


\section{Development of HPSG resources}
\label{sec:resources}

% First draft: DELPH-IN, Emily / non-DELPH-IN, Guy

\begin{itemize}
 \item  CoreGram
 \item  DELPH-IN consortium
    \begin{itemize}
    \item ERG
    \item Other large-ish grammars
    \item Grammar Matrix
    \end{itemize}
 \item Systems inspired by HPSG:
   \begin{itemize}
     \item Alpino
     \item Enju
     \item RASP
   \end{itemize}
\end{itemize}

% Alpino cites from Gertjan:

%% Gosse Bouma, Gertjan van Noord, Robert Malouf. Alpino: Wide Coverage Computational Analysis of Dutch. In: Computational Linguistics in the Netherlands CLIN 2000.

%% Leonoor van der Beek, Gosse Bouma, Gertjan van Noord. Een brede computationele grammatica voor het Nederlands. Nederlandse Taalkunde, jaargang 7, 2002-4. [in Dutch]. 353--374. 

%% Gertjan van Noord. At Last Parsing Is Now Operational. In: Piet Mertens, Cedrick Fairon, Anne Dister, Patrick Watrin (editors): TALN06. Verbum Ex Machina. Actes de la 13e conference sur le traitement automatique des langues naturelles. Page 20--42.

%% Gertjan van Noord. Self-trained Bilexical Preferences to Improve Disambiguation Accuracy. In: Harry Bunt, Paola Merlo and Joakim Nivre (editors), Trends in Parsing Technology. Dependency Parsing, Domain Adaptation, and Deep Parsing. Springer Verlag. pp 183-200. 2010.

%% Barbara Plank and Gertjan van Noord. Dutch Dependency Parser Performance Across Domains. In: Proceedings of the 20th Meeting of Computational Linguistics in the Netherlands. 

%% Daniël de Kok and Barbara Plank and Gertjan van Noord. Reversible Stochastic Attribute-value Grammars. In: ACL 2011.

%% Gertjan van Noord, Robert Malouf. Wide Coverage Parsing with Stochastic Attribute Value Grammars. Draft. Improved version of:

%% Robert Malouf, Gertjan van Noord. Wide Coverage Parsing with Stochastic Attribute Value Grammars. In: IJCNLP-04 Workshop Beyond Shallow Analyses - Formalisms and statistical modeling for deep analyses. 

%% all are available from http://www.let.rug.nl/vannoord/papers/

\section{Deployment of HPSG resources}
\label{sec:deployment}

\subsection{Language documentation and linguistic hypothesis testing}

Deployment for linguistic goals.

% King (2016) ``Theoretical linguistics and grammar engineering as mutually constraining disciplines''
% http://web.stanford.edu/group/cslipublications/cslipublications/HPSG/2016/headlex2016-king.pdf

\subsubsection{CoreGram}
% Guy

\subsubsection{Grammar Matrix}
% Emily

\subsubsection{AGGREGATION}
% Emily

\subsubsection{Derived resources: Redwoods-style treebanks}
%Emily


\subsection{Downstream applications}
%Guy

Deployment for other tasks.
Large number of applications.
Focus on several important applications below.
See also \url{http://moin.delph-in.net/DelphinApplications}.


\subsubsection{Language teaching}

Redbird (McGraw-Hill)

RASP and Cambridge Assessment


\subsubsection{NLP tasks}

Information extraction

Summarisation

Machine translation



\subsubsection{Data for machine learning}

Unlike the previous sections,
not providing analyses, but providing data.
Not used at test time, only at training time.

Training deep learning systems
-- semantic parsing
-- skip over HPSG, go straight to semantic representations

Evaluation of deep learning systems
-- ShapeWorld
-- use a grammar to produce annotations


\subsection{Other?}

Alpino?

Enju?



\section{Linguistic Insights}
\begin{itemize}
    \item Ambiguity % Guy
    \item Long-tail phenomena (raising and control?) % Emily
    \item Scaling up (thematic roles) % Emily
    \item CLIMB methodology % Emily
\end{itemize}

\section{Summary}

\section*{Abbreviations}
\section*{Acknowledgements}

\printbibliography[heading=subbibliography,notkeyword=this] 
\end{document}
