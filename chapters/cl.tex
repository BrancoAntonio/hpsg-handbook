\documentclass[output=paper]{langsci/langscibook} 
\author{%
	Emily M.\ Bender\affiliation{University of Washington} \lastand
Guy Emerson\affiliation{Cambridge University}
}
\title{Computational linguistics and Language Engineering}

% \chapterDOI{} %will be filled in at production

%\epigram{Change epigram in chapters/03.tex or remove it there }
%\abstract{Change the  abstract in chapters/03.tex \lipsum[3]}
\maketitle

\begin{document}

% Comments from Gerald:

%I assume that the contributors would agree with me that this handbook is probably the wrong place to go into great technical detail on mathematical or computational subjects, but there needs to be at least something there to serve as a starting point from which the reader could choose to jump into some of the papers cited.‎ A summary of the major trends and the high-level intuitions behind a lot of the technical terminology that readers will encounter in HPSG-related CL papers, for example, would both be most welcome.

\section{Introduction}

% First draft: Emily

\section{Infrastructure}

% First draft: By subsection

\begin{itemize}
\item Relevant properties of HPSG that facilitate all of this  % First draft: Emily

    \begin{itemize}
    \item Stable formalism
    \item Differentiating formalism from theory
    \item Interest in core as well as periphery
    \item Type hierarchy (maintainability)
    \end{itemize}

\item Tractability / pratical considerations  % First draft: Guy
    \begin{itemize}
    \item Turing-completeness in theory
    \item Efficiency in practice
    \item Difference of perspective compared to CCG, TAG
    \item Parse ranking
    \end{itemize}

\item History: PAGE, VerbMobil, ?? % First draft: Emily
\item Current platforms:
    \begin{itemize}
    \item LKB/ACE/PET/Agree
    \item Trale
    \item Other
    \end{itemize}
\end{itemize}

\section{Development of HPSG resources}

% First draft: DELPH-IN, Emily / non-DELPH-IN, Guy

\begin{itemize}
 \item  CoreGram
 \item  DELPH-IN consortium
    \begin{itemize}
    \item ERG
    \item Other large-ish grammars
    \item Grammar Matrix
    \end{itemize}
 \item Systems inspired by HPSG:
   \begin{itemize}
     \item Alpino
     \item Enju
     \item RASP
   \end{itemize}
\end{itemize}

% Alpino cites from Gertjan:

%% Gosse Bouma, Gertjan van Noord, Robert Malouf. Alpino: Wide Coverage Computational Analysis of Dutch. In: Computational Linguistics in the Netherlands CLIN 2000.

%% Leonoor van der Beek, Gosse Bouma, Gertjan van Noord. Een brede computationele grammatica voor het Nederlands. Nederlandse Taalkunde, jaargang 7, 2002-4. [in Dutch]. 353--374. 

%% Gertjan van Noord. At Last Parsing Is Now Operational. In: Piet Mertens, Cedrick Fairon, Anne Dister, Patrick Watrin (editors): TALN06. Verbum Ex Machina. Actes de la 13e conference sur le traitement automatique des langues naturelles. Page 20--42.

%% Gertjan van Noord. Self-trained Bilexical Preferences to Improve Disambiguation Accuracy. In: Harry Bunt, Paola Merlo and Joakim Nivre (editors), Trends in Parsing Technology. Dependency Parsing, Domain Adaptation, and Deep Parsing. Springer Verlag. pp 183-200. 2010.

%% Barbara Plank and Gertjan van Noord. Dutch Dependency Parser Performance Across Domains. In: Proceedings of the 20th Meeting of Computational Linguistics in the Netherlands. 

%% Daniël de Kok and Barbara Plank and Gertjan van Noord. Reversible Stochastic Attribute-value Grammars. In: ACL 2011.

%% Gertjan van Noord, Robert Malouf. Wide Coverage Parsing with Stochastic Attribute Value Grammars. Draft. Improved version of:

%% Robert Malouf, Gertjan van Noord. Wide Coverage Parsing with Stochastic Attribute Value Grammars. In: IJCNLP-04 Workshop Beyond Shallow Analyses - Formalisms and statistical modeling for deep analyses. 

%% all are available from http://www.let.rug.nl/vannoord/papers/

\section{Deployment of HPSG resources}

% First draft: ...

\begin{itemize}
 \item Language documentation/linguistic hypothesis testing
    \begin{itemize}
    \item CoreGram % Guy
    \item Grammar Matrix % Emily
    \item AGGREGATION % Emily
    \end{itemize}
 \item DELPH-IN:
    \begin{itemize}
    \item DELPH-IN Applications: Things we do using DELPH-IN grammars directly % Guy
    \item Derived resources: Redwoods-style treebanks % Emily
    \item Training data for Deep Learning % Guy
    \end{itemize}
 \item Alpino % Guy
    \begin{itemize}
    \item ??
    \end{itemize}
 \item Other?
 \end{itemize}

\section{Lessons for Linguistics}
\begin{itemize}
    \item Ambiguity % Guy
    \item Long-tail phenomena (raising and control?) % Emily
    \item Scaling up (thematic roles) % Emily
    \item CLIMB methodology % Emily
\end{itemize}

\section{Summary}

\section*{Abbreviations}
\section*{Acknowledgements}

\printbibliography[heading=subbibliography,notkeyword=this] 
\end{document}
