\documentclass[output=paper]{langsci/langscibook} 
\author{Stefan Müller\affiliation{Humboldt-Universität zu Berlin}}
\title{HPSG and Construction Grammar}

% \chapterDOI{} %will be filled in at production

%\epigram{Change epigram in chapters/03.tex or remove it there }
\abstract{This chapter discusses the main tenets of Construction Grammar (CxG) and shows that HPSG adheres
to them. This discussion includes surface orientation, language acquisition without UG, inheritance
networks and shows how HPSG (and other frameworks) are positioned along these dimensions. Formal variants of CxG will be briefly discussed and their relation to HPSG will be
pointed out. It is argued that lexical representations of valence are more appropriate than phrasal
approaches, which are assumed in most variants of CxG. Other areas of grammar seem to require
headless phrasal constructions (\eg the NPN construction and certain extraction constructions) and
it is shown how HPSG handle these. Derivational morphology is discussed as a further example of
an early constructionist analysis in HPSG.}
\maketitle

\begin{document}
\label{firstpage-cxg}\label{chap-cxg}
\is{Construction Grammar|(}

\noindent
This chapter deals with Construction Grammar (CxG) and its relation to HPSG. The short version of the
message is: HPSG is a Construction Grammar. It was one right from the beginning and over the years
certain aspects were adapted allowing to capture generalizations over phrasal patterns. In what
follows I will first say what Construction Grammars are (Section~\ref{sec-cxg}), I will explain why
HPSG as developed in \citew{ps,ps2} was a Construction Grammar and how it was changed to become even
more Constructive (Section~\ref{sec-inheritance}). Section~\ref{sec-valence} deals with so"=called
argument structure constructions, which are usually dealt with by assuming phrasal constructions in CxG and explains why this is
problematic and why lexical approaches are more appropriate. Section~\ref{sec-cxg-morphology} explains
Construction Morphology, Section~\ref{sec-phrasal} shows how cases that should be treated phrasally
can be handled in HPSG. Section~\ref{sec-summary} sums up the paper.

\section{What is Construction Grammar?}
\label{sec-cxg}

The first question to answer in a chapter like this is: what is Construction Grammar? While it is
relatively clear what a Construction is, the answer to the question regarding Construction Grammar
is less straight-forward. Section~\ref{sec-def-construction} provides the definition for the term
\emph{Construction} and Section~\ref{sec-tenets} states the tenets of CxG and discusses to what
extent the main frameworks currently on the market adhere to them.

\subsection{The notion Construction}
\label{sec-def-construction}

\citet[\page 5]{Goldberg2006a} defines \isi{Construction} as follows:
\begin{quote}
Any linguistic pattern is recognized as a construction as long as some aspect of its form or
function is not strictly predictable from its component parts or from other constructions recognized
to exist. In addition, patterns are stored as constructions even if they are fully predictable as
long as they occur with sufficient frequency. \citep[\page 5]{Goldberg2006a}
\end{quote}
She provides Table~\ref{tab-constructions} with examples for Constructions.
\inlinetodostefan{BB: I understand her view is that constructions must have a distinctive meaning. I
  believe Jackendoff argues against this, and Newmeyer and I did here: \citew{BN2009a}}
\begin{table}
\oneline{%
\begin{tabular}{ll}\lsptoprule
Word                    & e.g., \emph{tentacle}, \emph{gangster}, \emph{the} \\
Word (partially filled) & e.g., \emph{post}-N, V-\emph{ing}\\
Complex word            & e.g., \emph{textbook}, \emph{drive-in}\\
Idiom (filled)          & e.g., \emph{like a bat out of hell}\\
Idiom (partially filled) & e.g., \emph{believe} <one’s> \emph{ears/eyes}\\
Covariational Conditional &  The Xer the Yer\\
                          & (e.g., \emph{The more you watch the less you know})\\
Ditransitive              &  Subj V Obj1 Obj2\\
                          & (e.g., \emph{She gave him a kiss};\\
                          & \emph{He fixed her some fish tacos.})\\
Passive                   & Subj aux VPpp ( PPby )\\
% Subj kann bei Kontrollkonstruktionen entfallen
                          & (e.g., \emph{The cell phone tower was struck by lightening.})\\\lspbottomrule
\end{tabular}}
\caption{\label{tab-constructions}Examples of constructions, varying in size and complexity according to \citet{Goldberg2009b-u}}
\end{table}


If one just looks at the definition of Construction, all theories currently on the market could be
regarded as Construction Grammars. As Peter Staudacher\aimention{Peter Staudacher} pointed out in the discussion after a talk by
Knud Lambrecht\aimention{Knud Lambrecht} in May 2006 in Potsdam, lexical items are form-meaning pairs and the rules of
phrase structure grammars come with specific semantic components as well, even if it is just
\isi{functional application}. So, \cg, \isi{GB}, \gpsg, \treeag,
\lfg, HPSG and even \minimalism would be Construction Grammars. If one looks at the
examples of Constructions in Table~\ref{tab-constructions} things change a bit. Idioms are generally
not the focus of work in \isi{Mainstream Generative Grammar (MGG)}. MGG is usually concerned with
explorations of the so-called \isi{Core} Grammar as opposed to the \isi{Periphery}, to which the
idioms are assigned. The Core Grammar is the part of the grammar that is supposed to be acquired with
help of innate domain specific knowledge, something the existence of which Construction Grammar
denies. But if one takes \citet*{HCF2002a} seriously and assumes that only the ability to form complex
linguistic objects out of less complex linguistic objects (\isi{Merge}) is part of this innate knowledge
then the core/periphery distinction does not have much content and after all Minimalists could adopt
a version of Sag's local, selection-based analysis of idioms\is{idiom} \parencites{Sag2007a}{KSF2015a}{KM2017a}.
However, as is discussed in the next subsection, there are other aspects that really set Construction Grammar apart from
MGG.

\subsection{Basic tenets of Construction Grammar}
\label{sec-tenets}

 \citet{Goldberg2003b} names the following tenets as core assumptions standardly made in CxG.
form-meaning pairs):
\begin{description}
\item[Tenet 1] All levels of description are understood to involve pairings of form with semantic or discourse function, including morphemes or words, idioms, partially lexically filled and fully abstract phrasal patterns. (See Table 1)
\item[Tenet 2] An emphasis is placed on subtle aspects of the way we conceive of events and states of
affairs.
\item[Tenet 3] A ‘what you see is what you get’ approach to syntactic form is adopted: no underlying levels
  of syntax or any phonologically empty elements are posited. 
\item[Tenet 4] Constructions are understood to be learned on the basis of the input and general cognitive mechanisms (they are constructed), and are expected to vary cross-linguistically.
\item[Tenet 5] Cross-linguistic generalizations are explained by appeal to general cognitive constraints together with the functions of the constructions involved.
\item[Tenet 6] Language-specific generalizations across constructions are captured via inheritance networks much like those that have long been posited to capture our non-linguistic knowledge.
\item[Tenet 7] The totality of our knowledge of language is captured by a network of constructions: a ‘construct-i-con.’
\end{description}

I already commented on Tenet~1 above. Tenet~2 concerns semantics and the syntax-semantics interface,
which are part of most HPSG analyses. In what follows I want to look in more detail at the other tenets. 

\subsubsection{Surface orientation and empty elements}

Tenet~3 requires a surface-oriented approach. Underlying levels and phonologically empty elements
are ruled out. This excludes derivational models of transformational syntax assuming a D-structure
and some derived structure or more recent derivational variants of Minimalism. There was a time
where representational models of GB that did not assume a D-structure but just one structure with
traces (Koster \citeyear[\page ]{Koster78b-u}; \citeyear[\page 235]{Koster87a-u}; 
%\citealp[\page 66, Fußnote~4]{Bierwisch83a}; 
\citealp{KT91a}; \citealp[Section~1.4]{Haider93a}; 
\citealp[\page 14]{Frey93a}; \citealp[\page 87--88, 177--178]{Lohnstein93a-u}; \citealp[\page
  38]{FC94a}; \citealp[\page 58]{Veenstra98a}). Some of these analyses are rather similar to HPSG
analyses as they are assumed today \citep{Kiss95a,BvN98,Meurers2000b,Mueller2005c,MuellerGS,MuellerGermanic}. Chomsky's Minimalist work \citep{Chomsky95a-u} assumes a derivational model and comes
with a rhetoric of building structure in a bottom-up way and sending complete phases to the
interfaces for pronunciation and interpretation. This is incompatible with Tenet~3, but in principle
Minimalist approaches are very similar to Categorial Grammar, so there could be representational
approaches adhering to Tenet~3.

A comment on empty elements is in order: all articles introducing Construction Grammar state that
CxG does not assume empty elements. Most of the alternative theories do use empty elements: see
\citet{KoenigE99a-u} on Categorial Grammar, \citet*[\page 143]{GKPS85a} on GPSG, \citet[\page
  67]{Bresnan2001a} on LFG, \citew{Bender2000a} and \citew*[\page 464]{SWB2003a} on HPSG/Sign-Based
Construction Grammar. There are results from the 60s that show that phrase structure grammars
containing empty elements can be translated into grammars that do not contain empty elements
\citep*[\page 153, Lemma~4.1]{BHPS61a}. Grammars with empty elements often are more compact than
those without empty elements and express generalizations more directly. See for example
\citet{Bender2000a} for copulaless sentences in African American Venacular
English\il{English!African American Venacular} and \citet{Mueller2004e} on nounless NPs in
\ili{German}. The argument against empty elements usually refers to \isi{language acquisition}: it
is argued that empty elements cannot be learned since they are not present in the input. However,
if the empty elements alternate with visible material it can be argued that what is learned is the
fact that a certain element can be left out. What is true though is that things like empty
expletives cannot be learned since these empty elements are neither visible nor do they contribute to
meaning. Their only purpose in grammars is to keep uniformity. For example, \citet{Grewendorf93}
suggests an analysis of the passive in German that is parallel to the movement-based analysis of English
passives. In order to account for the fact that the subject does not move in German, he suggests an
empty expletive pronoun that takes the subject position and that is connected to the original
non-moved subject. Such elements cannot be acquired without innate knowledge about the IP/VP system and
constraints about the obligatory presence of subjects. The CxG criticism is justified here.

A frequent argumentation for empty elements in MGG is based on the fact that there are overt
realizations of an element in other languages (\eg object agreement\is{agreement!object} in \ili{Basque} and \isi{focus} markers in
\ili{Gungbe}). But since there is no language internal evidence for these empty elements they cannot be
learned and one would have to assume that they are innate. This kind of empty elements is rightly
rejected. 


\subsubsection{Language acquisition without the assumption of UG}

Tenet~4 and~5\is{language acquisition|(} are basically what everybody should assume in MGG if \citet*{HCF2002a} are taken seriously. Of
course this is not what is done in large parts of the filed. The most extreme variant being
\citet{CR2010a}, who assume at least 400 functional heads being part of \isi{Universal Grammar
(UG)} and being present in all grammars of all languages although sometimes
invisibly. Such assumptions beg the question why the genera of \ili{Bantu} languages should be part of our
genome\is{genetics} and how they got there. 
Researchers working on language \isi{acquisition} realized that the \isi{Principles \& Parameters}
approach \citep{Meisel95a} makes wrong predictions. They now talk about \isi{Micro-Cues} instead of
parameters \citep{Westergaard2014a} and these Micro-Cues are just features that can be
learned. However, Westergaard still assumes that the features are determined by UG, an dubious
assumption seen from a CxG perspective (and from the perspective of Hauser, Chomsky, Fitch and
genetics in general \citep{Bishop2002a}).

Note that even those versions of Minimalism that do not follow the Rizzi-style Cartographic
approaches are far from being minimalist in their assumptions. Some distinguish between strong
and weak features, some assume enumerations of lexical items from which a particular derivation
draws its input from, some assume that all movement has to be feature driven. Some assume that
derivations work in so-called phases and that a \isi{phase} once completed is ``shipped to the
interfaces''. Construction of phases is bottom up, which is incompatible with psycholinguistic
results (see also \citealt[Section~\ref{sec-minimalism-processing}]{chapter/minimalism}, Chapter~\ref{chap-minimalism} in this
volume). None of these assumptions is a not natural assumption to make from a
language acquisition point. Most of these assumptions do not have any real motivation in data, the
only motivation usually given is that they result in ``restrictive theories''. But if there is no
motivation for them, this means that the respective architectural assumptions have to be part of our
innate domain-specific knowledge, which is implausible according to \citet*{HCF2002a}.

As research in computational linguistics shows, our input is rich
enough to form classes, to determine the part of speech of lexical items and even to infer syntactic
structure thought to be underdetermined by the input. For instance, \citet{Bod2009a} shows that the
classical auxiliary inversion examples that Chomsky still uses in his \isi{Poverty of the Stimulus}
arguments (\citealp[\page 29--33]{Chomsky71a-u}; \citealp*{BPYC2011a}) can also be learned from language input available to children. See also
\citep{FPG2006a,FPAG2007a} on input-based language acquisition.

HPSG does not make any assumptions about complicated mechanisms like feature driven movement and so
on. HPSG states properties of linguistic objects like part of speech, case, gender and so on and
states relations between such features like agreement and government. In this respect it is like
other Construction Grammars and hence experimental results regarding and theories of language
acquisition can be carried over to HPSG. See also \crossrefchaptert{aquisition} on language
acquisition.\is{language acquisition|)}


\subsubsection{Inhertiance networks}
\label{sec-inheritance}

This leaves us with Tenet~6 and~7, that is \emph{inheritance networks} and the constructicon. MGG does not make reference to inheritance
hierarchies. HPSG did this right from the beginning in 1985 \citep{FPW85a} for lexical items and
since 1995 also for phrasal constructions \citep{Sag97a}. LFG rejected the use of types but used
macros in computer implementations. The macros were abbreviatory devices and did not play any role in
theoretical work. This changed in 2004 where macros were suggested in theoretical work
\citep*{DKK2004a}. And although any connection to constructionist work is vehemently denied by some
of the authors, recent work in LFG has a decidedly constructional flavor
\citep*{ADT2008a,AGT2014a}\todostefan{citep* yields wrong results here}.\footnote{
  See \citew[\page 516]{Toivonen2013a} for an explicit reference to construction-specific phrase
  structure rule in the senese of Construction Grammar.
}
LFG differs from frameworks like HPSG though in assuming a separate level of
c-structure. c-structure rules are basically context free phrase structure rules and they are not
modeled by feature value pairs (although they could be \citep{Kaplan95a}). This means that it is not
possible to capture generalizations regarding lexical items, lexical rules and phrasal
schemata. While HPSG describes all of these elements with the same inventory and hence can use
common supertypes in the description of all three, this is not possible in LFG.\todostefan{add reference}
TAG is also using inheritance in the Meta Grammar \citep{LK2017a}.

Since HPSG's lexical entries, lexical rules and phrasal schemata are all described by typed feature
descriptions one could call the set of these descriptions the \isi{constructicon}. Therefore,
tenet~7 is also adhered to. 

\subsubsection{Summary}

If all these points are taken together, it is clear that most variants of MGG are not Construction
Grammars. However, CxG had considerable influence on other frameworks so that there are
constructionist variants of LFG and TAG. HPSG in the version of \citet{Sag97a} (also called
Constructional HPSG) and the HPSG dialect Sign-Based Construction Grammar are Construction Grammars
that follow all the tenets mentioned above.

%\citet{Goldberg95a,Goldberg2006a,Michaelis2012a}
%% Historical aspects:
%% \begin{itemize}
%% \item non-locality \citep{FKoC88a}
%% \item type inheritance \citet{KF99a,Sag97a}
%% \end{itemize}

\subsection{Variants of Construction Grammar}

The previous section discussed the tenets of CxG and to what degree other frameworks adhere to them. This
section deals with frameworks that have Construction Grammar explicitly in their name. The following
variants are usually named:
\begin{sloppypar}
\begin{itemize}
\item Berkeley Construction Grammar\is{Construction Grammar!Berkeley} \citep{Fillmore88a,KF99a,FriedHSK}
\item Cognitive Construction Grammar\is{Construction Grammar!Cognitive} \citep{Lakoff87a-u,Goldberg95a,Goldberg2006a}
\item \isi{Cognitive Grammar} \citep{Langacker87a-u,Langacker2000a,Langacker2008a-u,Dabrowska2004a}
\item Radical Construction Grammar\is{Construction Grammar!Radical} \citep{Croft2001a}
\item Embodied Construction Grammar\is{Construction Grammar!Embodied} \citep{BC2005a}
\item Fluid Construction Grammar\is{Construction Grammar!Fluid} \citep{SDB2006a-u,SteelsFluid-ed-not-crossreferenced}
\item Sign-Based Construction Grammar\is{Construction Grammar!Sign-Based} \citep{Sag2010b,Sag2012a}
\end{itemize}
\end{sloppypar}

Berkely Construction Grammar, Embodied Construction Grammar, Fluid Construction Grammar, and
Sign-Based Construction Grammar are the ones that are more formal. All of these variants use feature
value pairs and are constraint-based. They are sometimes also referred to as unification-based
approaches. Berkeley Construction Grammar\is{Construction Grammar!Berkeley|(} never had a consistent formalization. The variant of
\isi{unification} assumed by \citet{KF99a} was formally inconsistent \citep[Section~2.4]{Mueller2006d} and the computation of \isi{construction-like
object}s (CLOs) suggested  by \citet{Kay2002a} did not work either \citep[Section~3]{Mueller2006d}. Berkeley Construction
Grammar was dropped by the authors, who joined forces with Ivan Sag, Tom Wasow\aimention{Tom Wasow}, and Laura
Michaelis\aimention{Laura Michaelis} and eventually
came up with an HPSG variant named \sbcg\is{Construction Grammar!Sign-Based|(} \citep{Sag2012a}. The differences between
Constructional HPSG \citep{Sag97a} and SBCG are to some extent cosmetic: semantic relations got the
suffix \type{-fr} for \emph{frame} (\type{like-rel} became \type{like-fr}), phrases were called constructions (\type{hd-subj-ph} became
\type{subj-head-cxt}) and lexical rules were called \emph{derivational constructions}.\footnote{
This renaming trick was so successful that it even confused some of the co-editors of the volume about
SBCG \citep{BS2012a-ed}. See for example \citew{Boas2014a} and the reply in \citew{MWArgStReply}.
}
While this renaming would not have changed anything in terms of expressiveness of theories, there
was another change that was not motivated by any of the tenets of Construction Grammar but rather by
the wish to get a more restrictive theory: \citet*{SWB2003a} and \citet{Sag2007a} changed the feature geometry of phrasal
signs in a way that signs do not contain daughters. The information about mother-daughter relations
is contained in lexical rules and phrasal schemata (Constructions) only. The phrasal schemata are
more like GPSG phrase structure rules in licensing a mother node when certain daughters are present
but without the daughters being part of the mother as it was common in HPSG from 1985 till
\citew*{SWB2003a}. This differs quite dramatically from what was done in Berkeley Construction
Grammar, since BCxG explicitly favored a non-local approach. Arguments were not cancelled but
passed up to the mother node. Adjuncts were passed up as well so that the complete internal
structure of an expression is available at the top-most node. The advantage of BCxG \citep*{FKoC88a} and
Constructional HPSG \citep{Sag97a} is that complex expressions (\eg idioms and other more transparent expressions
with high frequency) can be stored as chunks containing the internal structure. This is not possible
with SBCG, since phrasal signs never contain internal structures. For a detailed discussion of \sbcg
see \citew[Section~10.6.2]{MuellerGT-Eng1}.\is{Construction Grammar!Berkeley|)}\is{Construction Grammar!Sign-Based|)}

\isi{Embodied Construction Grammar} \citep{BC2005a} uses typed feature descriptions for the description of
linguistic objects and allows for discontinuous constituents. As
argued by \citet[Section~10.6.3]{MuellerGT-Eng1}, it is a notational variant of Reape-style HPSG
\citep{Reape94a} (see \citealt[Section~\ref{sec-domains}]{chapters/order}, Chapter~\ref{chap-order}
of this volume for discontinuous constituents in HPSG).

\isi{Fluid Construction Grammar} is also rather similar to HPSG. An important difference is that FCG attaches
weights to constraints, something that is usually not done in HPSG. But in principle there is
nothing that forbids to add weights to HPSG as well and in fact it has been done \citep{Brew95a,BC99a,MT2008a-u} and it should be
done to a larger extend \citep{Miller2013a}. \Citet{vanTrijp2013a} tried to show that
Fluid Construction Grammar is fundamentally different from SBCG but I think he failed in every
single respect. See \citew{MuellerFCG} for a detailed discussion, which cannot be repeated here for
space reasons.

What makes SBCG different from other Construction Grammars is that SBCG assumes a strongly
lexicalist stance \citep{SW2011a}: argument structure is encoded lexically. A ditransitive verb is a
ditransitive verb since it selects for three NP arguments. This selection is encoded in valence
features of lexical items. It is not assumed that phrasal configurations can license additional
arguments as it is in Radical Construction Grammar, Embodied Construction Grammar and in Fluid
Construction grammar. The next section discusses phrasal CxG approaches in more
detail. Section~\ref{sec-phrasal-patterns} then discusses patterns that should be analyzed phrasally
and which are problematic for entirely head-driven (or rather functor-driven) theories like \isi{Categorial Grammar},
\isi{Dependency Grammar} and \isi{Minimalism}.

%% \section{HPSG as a Construction Grammar}
%% \label{sec-hpsg-as-cxg}

%% \begin{itemize}
%% \item form-meaning pairs
%% \item type hierarchies
%% \item surface oriented
%% \end{itemize}

\section{Valence vs.\ phrasal patterns}
\label{sec-valence}

Much\is{language acquisition|(} work in Construction Grammar starts from the observation that children acquire patterns and get
more abstract leaving slots to be filled in in later acquisition stages \citep{Tomasello2003a}. The conclusion that is
drawn from this is that language should be described with reference to phrasal patterns. Most
Construction Grammar variants assume a phrasal approach to argument structure constructions \citep{Goldberg96a,Goldberg2006a,GJ2004a},
Constructional HPSG and SBCG being the two exceptions.\todostefan{Boas} 

I argued in several publications that the language acquisition facts can be explained in lexical
models as well \parencites[Section~6.3]{MuellerPersian}[Section~9]{MWArgSt}. While a pattern-based approach claims that (\mex{1}) is
analyzed by inserting \emph{Kim}, \emph{loves}, and \emph{Sandy} into a phrasal schema stating that
NP[nom] verb NP[acc] or subject verb object are possible sequences in English, a lexical approach
would state that there is a verb \emph{loves} selecting for an NP[nom] and an NP[acc] (or for a
subject and an object).
\ea
Kim loves Sandy.
\z
Since objects follow the verb in English (modulo extraction) and subjects precede the verb, the same
sequence is licensed in the lexical approach. The lexical approach does not have any problems with
accounting for patterns in which the sequence of subject, verb and object is discontinuous. For
example, an adverb may intervene between subject and verb:
\ea
Kim really loves Sandy.
\z
In a lexical approach it is assumed that verb and object may form a unit (a VP). The adverb attaches
to this VP and the resulting VP is combined with the subject. The phrasal approach has to assume
that either adverbs are part of phrasal schemata licensing cases like (\mex{0}) (see
\citew[Section~6.3.2]{Uszkoreit87a} for such a proposal in a \gpsg for \ili{German}) or that the phrasal
construction may license discontinuous\is{constituent!discontinuous} patterns. \citet[\page 170]{BC2005a} follow the latter approach and
assume that subject and verb may be discontinuous but verb and object(s) have to be adjacent. While
this accounts for adverbs like the one in (\mex{0}), it does not solve the general problem since
there are other examples showing that verb and object(s) may appear discontinuously as well:
\ea
Mary tossed me a juice and Peter a water.
\z
Even though \emph{tossed} and \emph{Peter a water} are discontinuously in (\mex{0}), they are an
instance of the ditransitive construction. The conclusion is that what has to be acquired is not a phrasal pattern but rather the fact that
there are dependencies\is{dependency} between certain elements in phrases. I return to ditransitive
constructions in Section~\ref{sec-coercion}.%
\is{language acquisition|)}

I discussed several phrasal approaches to \isi{argument structure} and showed where they fail
(\citeauthor{Mueller2006d}, \citeyear*{Mueller2006d,Mueller2006c,Mueller2007d,MuellerPersian,MWArgSt,MWArgStReply,MuellerLFGphrasal}).
Of course
the discussion cannot be repeated here but I want to repeat two points showing that lexical valence
representation is necessary. The first two\todostefan{fix me, add point} are problems that were around at GPSG times and basically
were solved by abandoning the framework and adopting a new framework which was a fusion of GPSG and
Categorial Grammar: HPSG.

\subsection{Derivational morphology and valence} 

The first argument \citep[Section~5.5.1]{MuellerGT-Eng1} is that certain patterns in derivational morphology refer to valence. For
example, the \bard productively applies to transitive verbs only, that is to verbs that govern an accusative.
\eal
\ex[]{
\gll unterstützbar\\
     supportable\\
}
\ex[*]{
\gll helfbar\\
     helpable\\
}
\ex[*]{
\gll schlafbar\\
     sleepable\\
}
\zl
\begin{sloppypar}
\noindent
Note that \bard is like passive: it surpresses the subject and promotes the accusative object: the
accusative object is the element adjectives derived with \bard predicate over. There is no argument
realized with the adjective \emph{unterstützbaren} `supportable' attaching to \emph{Arbeitsprozessen} `work.""processes' in \emph{unterstützbaren
  Arbeitsprozessen}.\footnote{
Adjectives realize their arguments preverbally in German:
\ea
\gll der [seiner Frau treue] Mann\\
     the \spacebr{}his wife faithful man\\
\glt `the man who is faithful to his wife'
\z
\emph{unterstützbaren} `supportable' does not take an argument it is a complete adjectival projection like \emph{seiner Frau treue}.
}  Hence one could not claim that the stem enters a phrasal construction with arguments and
\suffix{bar} attaches to this phrase. It follows that information about valency has to be present at
the stem. 
\end{sloppypar}

Note also that the resultative construction interacts with \bard. (\mex{1}) shows an example of this
construction in which the accusative object is introduced by the construction: it is the subject of
\emph{leer} `empty' but not a semantic argument of the verb \emph{fischt} `fishes'.
\ea
\gll Sie fischt den Teich leer.\\
     she fishes the pond empty\\
\z
So even though the accusative object is not a semantic argument of the verb, the \bard is possible
and an adjective like \emph{leerfischbar} can be derived. This is explained by lexical analyses of
the \bard and the resultative construction since if one assumes that there is a lexical item for
\stem{fisch} selecting the accusative object and the result predicate then this item may function as
the input for the \bard. See Section~\ref{sec-cxg-morphology} for further discussion of \bard and
 \citew{Verspoor97a}, \citew{Wechsler97a}, \citew{WN2001a}, \citew[Chapter~5]{Mueller2002b} for lexical analyses of the resultative construction in the framework of HPSG.

\subsection{Partial verb phrase fronting}

The second argument concerns partial verb phrase fronting \citep[Section~5.5.2]{MuellerGT-Eng1}. 
(\mex{1}) gives some examples: in (\mex{1}a) the bare verb is fronted and its arguments are realized
in the middle field, in (\mex{1}b) one of the objects is fronted together with the verb and in
(\mex{1}c) both objects are fronted with the verb.
\eal
\ex 
\gll Erzählen wird er seiner Tochter ein Märchen können.\\
     tell will he his daughter a fairy.tale can\\
\ex 
\gll Ein Märchen erzählen wird er seiner Tochter können.\\
     a fairy.tale tell will he his daughter can\\
\ex 
\gll Seiner Tochter ein Märchen erzählen wird er können.\\
     his daughter a fairy.tale tell will he can\\
\glt `He will be able to tell his daughter a fairy tale.'
\zl
The problem with sentences such as those in (\mex{0}) is that the valence requirements of the verb
\emph{erzählen} `to tell' are realized in various positions in the sentence. For fronted
constituents, one requires a rule which allows a ditransitive to be realized without its arguments
or with one or two objects. This basically destroys the idea of a fixed phrasal configuration for
the ditransitive construction and points again into the direction of dependencies\is{dependency}.

Furthermore, it has to be ensured that the arguments that are
missing in the prefield are realized in the remainder of the clause. It is not legitimate to omit
obligatory arguments or realize arguments with other properties like a different case, as the
examples in (\mex{1}) show:
\eal
\ex[]{
\gll Verschlungen hat er es nicht.\\
     devoured     has he.\NOM{} it.\ACC{} not\\
\glt `He did not devour it.'
}
\ex[*]{
\gll Verschlungen hat er nicht.\\
     devoured     has he.\NOM{} not\\
}
\ex[*]{
\gll Verschlungen hat er ihm nicht.\\
     devoured     has he.\NOM{} him.\DAT{} not\\
}
\zl
The obvious generalization is that the fronted and unfronted arguments must add up to the total
set belonging to the verb. This is scarcely possible with the rule-based valence
representation in GPSG. In theories such as Categorial Grammar\is{Categorial Grammar (CG)}, it is possible to formulate elegant analyses of (\mex{0})
\citep{Geach70a}. \citet{Nerbonne86a} and \citet{Johnson86a} both suggest analyses for sentences
such as (\mex{0}) in the framework of GPSG which
ultimately amount to changing the representation of valence information in the direction of
Categorial Grammar.
With a switch to CG-like valence representations in HPSG the phenomenon of partial verb phrase
fronting found elegant solutions \parencites[Section~4]{HoehleSpuren}{Mueller96a}{Meurers99a}.\is{partial verb phrase
  fronting|)}\todostefan{reference to UDC chapter?} 


\subsection{Coercion}
\label{sec-coercion}

An important observation in constructionist work is that in certain cases verbs can be used in
constructions that differ from the constructions they are normally used in. For example, verbs that are usually
used with one or two arguments may be used in the ditransitive construction:
\eal
\ex She smiled.
\ex She smiled herself an upgrade.
\ex He baked a cake.
\ex He baked her a cake.
\zl
The usual explanation for sentences like (\mex{0}b) and (\mex{0}d) is that there is a phrasal
pattern with three arguments into which intransitive and strictly transitive verbs may be
entered. It is assumed that the phrasal patterns are associated with a certain meaning \citep{Goldberg96a,GJ2004a}. For example,
the benefactive meaning of (\mex{0}d) is contributed by the phrasal pattern
(\citealt[]{Goldberg96a}; \citealt*[\page 81]{AGT2014a}).\todostefan{page number}

The insight that a verb is used in the ditransitive pattern and thereby contributes a certain
meaning is of course also captured in lexical approaches. \citet{BC99a} suggested a lexical
rule-based analysis mapping a transitive version of verbs like \emph{bake} onto a ditransitive one
and adding the benefactive semantics. This is parallel to the phrasal approach in that it says:
three-place \emph{bake} behaves like other three-place verbs (\eg \emph{give}) in taking three
arguments and by doing so it comes with a certain meaning (see \citealt{MuellerLFGphrasal} for a
lexical rule-based analysis of the benefactive constructions working for both English and German
despite the surface-differences of the respective languages). The lexical rule is a form-meaning pair
and hence a construction. As Croft put it 15 years ago: Lexical rule vs.\ phrasal schema is a false
dichotomy \citep{Croft2003a}.
\inlinetodostefan{BB: You quote Croft as saying that ``lexical rule vs. phrasal schema
is a false dichotomy''. This might suggest that you think there is no
real issue here, but of course you have argued that there is.}

\citet{BC99a} paired their lexical rules with probabilities to be able to explain differences in
productivity. This corresponds to the association strength that \citet[\page 141]{vanTrijp2011a} used to relate
lexical items to phrasal constructions of various kinds.

\subsection{Non-predictability of valence}

The last subsection discussed phrasal proposals of coercion that assume that verbs can be inserted
into constructions that are compatible with the semantic contribution of the verb. \citet[Section~7.4]{MWArgSt}
pointed out that this is not sufficiently constrained. Müller \& Wechsler discussed the examples in
(\mex{1}), among others:

\eal\label{depends-on-ex}
\ex John depends on Mary.  (\emph{counts, relies,} etc.)
\ex John trusts (*on) Mary.  
\zl
While \emph{depends} can be combined with a \emph{on}-PP, this is impossible for \emph{trusts}. Also
the form of the preposition of prepositional objects is not always predictable from semantic
properties of the verb. So there has to be a way to state that certain verbs go together with
certain kinds of arguments and others do not. A lexical specification of valence information is the
most direct way to do this. Phrasal approaches sometimes assume other means to establish connections
between lexical items and phrasal constructions. For instance, \citet[\page 50]{Goldberg95a} assumes
that verbs are ``conventionally associated with constructions''. The more technical work in Fluid
CxG\is{Construction Grammar!Fluid} assumes that every lexical item is connected to various phrasal constructions via coapplication
links \citep[\page 141]{vanTrijp2011a}. This is very similar to Lexicalized Tree Adjoining
Grammar\is{Tree Adjoining Grammar (TAG)}
(LTAG, \citealt*{SAJ88a-u}), where a rich syntactic structure is associated to a lexical anchor. So,
the phrasal approaches that link syntactic structure to lexical items are actually lexical
approaches as well. As in \gpsg some means makes sure that the lexical items enter into correct
constructions. In GPSG this was taken care of by a number. I already discussed the GPSG shortcomings
in previous subsections.

Concluding this section, it can be said that there has to be a connection between lexical items and
their arguments and that a lexical representation of argument structure is the best way to establish
such a relation.

%\section{HPSG as a Construction Grammar}
%\label{sec-hpsg-as-cxg}

\section{Construction Morphology}
\label{sec-cxg-morphology}

The\is{morphology|(}\is{derivation|(} first publication in \isi{Construction Morphology} was the masters thesis of \citet{Riehemann93a},
which later appeared as \citet{Riehemann98a}. Riehemann called her framework \emph{Type-Based
  Derivational Morphology} since it was written before influential work like \citew{Goldberg95a}
appeared and before the term \emph{Construction Morphology} \citep{Booij2005a} was used. Riehemann did a careful corpus study on adjective
derivations with the suffix \suffix{bar} `-able'. She noticed that there is a productive pattern
that can be analyzed by a lexical rule relating a verbal stem to the adjective suffixed with
\suffix{bar}.\footnote{
  She did not call her rule lexical rule but the difference between her template and the
  formalization of lexical rules by \citet{Meurers2001a} is the naming of the feature \textsc{morph-b}
  vs.\ \textsc{dtr}.
} The productive pattern applies to verbs governing an accusative as in (\mex{1}a) but
is incompatible with verbs taking a dative as in (\mex{1}b):
\eal
\ex[]{
\gll unterstützbar\\
     supportable\\
}
\ex[*]{
\gll helfbar\\
     helpable\\
}
\ex[*]{
\gll schlafbar\\
     sleepable\\
}
\zl
Intransitive verbs are also excluded as (\mex{0}c) shows. Riehemann suggests the schema in
(\mex{1}):\todostefan{fix page number}
\ea
\label{ex-schema-bar}
\begin{tabular}[t]{@{}l@{}}
Schema for productive adjective derivations with the suffix \suffix{bar} in\\ German according to
\citet[\page 17]{Riehemann98a}:\\*
\onems[reg-bar-adj]{
phonology  \ibox{1} + \normalfont\textit{bar}\\
morph-b    \liste{ \onems[trans-verb]{ phon \ibox{1}\\
                                        synsem|loc \ms{ cat|val|comps \sliste{ NP[\type{acc}]:\ibox{2} } $\oplus$ \ibox{3}\\              
                                                        cont|nuc \ibox{4} \ms{ act\\
                                                                               und \ibox{2}\\
                                                                             }\\
                                                      }\\
                                     } }\\
synsem|local \onems{ category \ms{ head & adj\\
                                valence & \ms{ subj & \sliste{ NP:\ibox{2} }\\
                                               comps & \ibox{3}\\
                                             }\\
                              }\\
                  content|nucleus \ms{ reln & $\Diamond$\\
                                       arg1 & \ibox{2}\\
                                       arg2 & \ibox{4}\\
                                     }\\
                }\\
}
\end{tabular}
\z
\textsc{morph-b} is a list that contains a description of a transitive verb (something that governs
an accusative object which is linked to the undergoer role \iboxb{2} and has an actor.\footnote{
  Note that the specification of the type \type{trans-verb} in the list under \textsc{morph-b} is
  redundant since it is stated that there has to be an accusative object and that there is an actor
  and an undergoer in the semantics. Depending on further properties of the grammar the
  specification of the type is actually wrong: productively derived particle verbs may be input to
  the \bard and these are not a subtype of \type{trans-verb} since the respective particle verb rule
  derives both transitive (\emph{anlachen} `laught at somebody')  and intransitive verbs
  (\emph{loslachen} `start to laugh') \citep[\page 296]{Mueller2003a}. \emph{anlachen} does not have an
  undergoer in the semantic representation suggested by \citet{Stiebels96a}. See
  \citet[\page 308]{Mueller2003a} for a version of the \bard schema that is compatible with particle verb
  formations as input.
}
The phonology of this element \iboxb{1} is combined with the suffix \suffix{bar} and forms the
phonology of the complete lexical item. The resulting object is of category \type{adj} and the
semantics of the accusative object of the input verb \iboxb{2} is identified with the one of the subject of the resulting adjective. The semantics of the
input verb \iboxb{4} is embedded under a modal operator in the semantics of the adjective. 

While the description of \bard given so far captures
the situation quite well, there are niches and isolated items that are exceptions. According to
\citet[\page 5]{Riehemann98a}, this was the case for 7\,\% of the adjectives she looked at in her
corpus study. Examples are verbs ending in \suffix{ig} like \emph{entschuldigen} `to excuse'. The
\suffix{ig} is dropped in the derivation:
\ea
\gll entschuldbar\\
     excuseable\\
\z
Other cases are lexicalized forms like \emph{essbar} `safely edible', which have a special
lexicalized meaning. Exceptions of the accusative requirement are verbs selecting a dative
(\mex{1}a), a prepositional object (\mex{1}b), reflexive verbs (\mex{1}c), and even intransitive, mono-valent verbs (\mex{1}d):
\eal
\ex 
\gll unentrinnbar\\
     inescapable\\
\ex 
\gll verfügbar\\
     available\\
\ex
\gll regenerierbar\\
     regenerable\\
\ex 
\gll brennbar\\
     inflammable\\
\zl 

To capture generalizations about productive, semi-productive and fixed patterns/items Riehemann
suggests a type hierarchy, parts of which are provided in
Figure~\ref{fig-bar-Riehemann}.\todostefan{why doesn't the figure start at the left?}
\begin{figure}
\centerfit{%
%\scalebox{0.95}{%
\begin{forest}
typehierarchy
[bar-adj
  [trans-bar-adj
    [reg-bar-adj]
    [essbar]
    [\ldots]]
  [dative-bar-adj
    [unentrinnbar]
    [\ldots]]
  [prep-bar-adj
    [verfügbar]]
  [intr-bar-adj
    [brennbar]
    [\ldots]]]
\end{forest}}
\caption{\label{fig-bar-Riehemann}Parts of the type hierarchy for \bard adapted from \citew[\page 15]{Riehemann98a}}
\end{figure}
The type \type{bar-adj} stands for all \suffix{bar} adjectives and comes with the constraints that
apply to all of them. One subtype of this general type is \type{trans-bar-adj}, which subsumes all
adjectives that are derived from transitive verbs. This includes all regularly derived \baradjs,
which are of the type \type{reg-bar-adj} but also \emph{essbar} `edible' and \emph{sichtbar} `visible'.

As this recapitulation of Riehemann's proposal shows, the analysis is a typical CxG analysis:
V-\emph{bar} is a partially filled word (see Goldberg's examples in
Table~\ref{tab-constructions}). The schema in (\ref{ex-schema-bar}) is a form-meaning
pair. Exceptions and subregularities are represented in an inheritance network.

\is{morphology|(}\is{derivation|(}

\section{Phrasal patterns}
\label{sec-phrasal-patterns}
\label{sec-phrasal}

Section~\ref{sec-valence} discussed the claim that Constructions in the sense of CxG have to be phrasal. I
showed that this is not true and that in fact lexical approaches to valence have to be preferred
under the assumptions usually made in non-transformational theories. However, there are other areas
of grammar that give exclusively head-driven approaches like \cg, \minimalism, and \dg a hard
time. In what follows I discuss the NPN construction and various forms of filler gap constructions.

\subsection{The N-P-N Construction}

\citet{Matsuyama2004a} and \citet{Jackendoff2008a} discuss the \isi{NPN Construction}, examples of
which are provided in (\mex{1}):
\eal
\ex Student after student left the room.
\ex
\label{ex-npn-iteration}
Day after day after day went by, but I never found the courage to talk to
her. \citep{Bargmann2015a}
\zl
The properties of the NPN construction (with \emph{after})  are summarized by \citet{Bargmann2015a}
in a concise way and I will repeat his examples and summarization below to motivate his analysis in (\ref{ex-npn-bragmann}).

The examples in (\mex{0}) show that the N-after-N Construction has \emph{NP distribution}.

As (\mex{1}) shows, the construction is \emph{partially lexically fixed}: \emph{after} cannot be
replaced by any other word \citep[\page 73]{Matsuyama2004a}.
\ea
Alex asked me question \{ after / * following / * succeeding \} question.
\z

The construction is \emph{partially lexically flexible}: The choice of Ns is free, except for that
the Ns must be identical (\mex{1}a), the Ns must be count nouns (\mex{1}b), Ns must be in the
singular (\mex{1}c), and the Ns must be bare (\mex{1}d).

\eal
\settowidth\jamwidth{(Ns have determiners)}
\ex[*]{
bus after car \jambox{(N1 $\neq$ N2)}
}
\ex[*]{
water after water \jambox{(Ns = mass nouns)}
}
\ex[*]{
books after books \jambox{(Ns = plurals)}
}
\ex[*]{
a day after a day \jambox{(Ns have determiners)}
}
\zl

The construction is \emph{syntactically fixed}: N-after-N cannot be split by syntactic operations as the
contrast in (\mex{1}) shows \citep{Matsuyama2004a}:
\eal
\ex[]{
Man after man passed by. 
}
\ex[*]{
Man passed by after man.
}
\zl
If extraposition of the \emph{after}-N constituent were possible, (\mex{0}b) with an extraposed
\emph{after man} should be fine but it is not, so NPN seems to be a fixed configuration.

There is a syntax-semantics mismatch:
while N-after-N is singular, syntactically as (\mex{1}) shows, it is plural semantically as
(\mex{2}) shows:
\ea
Study after study \{ reveals / *reveal \} the dangers of lightly trafficked streets.
\z
\eal
\ex John ate \{ apple after apple / apples / *an apple \} for an hour. 
\ex John ate \{ *apple after apple / *apples / an apple \} in an hour.
\zl
Furthermore there is an aspect of semantic sequentiality: N-after-N conveys a temporal or spatial
sequence: as \citet{Bargmann2015a} states the meaning of (\mex{1}a) is something like (\mex{1}b).
\eal
\ex Man after man passed by. 
\ex First one man passed by, then another(, then another(, then another(, then  \ldots{} ))). 
\zl
The Ns in the construction do not refer to one individual each, rather they contribute to a holistic meaning.

The NPN construction allows adjectives to be combined with the nouns but this is restricted.
N1 can only be preceded by an adjective if N2 is preceded by the same adjective: 
\eal
\ex[]{
bad day after bad day (N1 and N2 are preceded by the same adjective.)
}
\ex[*]{
bad day after awful day (N1 and N2 are preceded by different adjectives.) 
}
\ex[*]{
bad day after day (Only N1 is preceded by an adjective.)
}
\ex[]{
day after bad day (Only N is preceded by an adjective.)
}
\zl

Finally, \emph{after} N may be iterated to emphasize the fact that there are several referents of N as the example in (\ref{ex-npn-iteration}) shows. 


This empirical description is covered by the following phrasal construction, which is adapted from
\citew{Bargmann2015a}:\footnote{%
Jackendoff and Bargmann assume that the result of combining N, P, and N is an NP. However this is
potentially problematic as Matsuyama's example in (\mex{1}) shows \citep[\page 71]{Matsuyama2004a}:
\ea
All ranks joined in hearty cheer after cheer for every member of the royal family \ldots
\z
As Matsuyama points out the reading of such examples is like the reading of \emph{old men and women}
in which \emph{old} scopes over both \emph{men} and \emph{women}. This is accounted for in
structures like the one indicated in (\ref{ex-hearty-cheer-after-cheer}):
\ea
\label{ex-hearty-cheer-after-cheer}
hearty [cheer after cheer]
\z
Since adjectives attach to \nbars and not to NPs this means that NPN constructions should be
\nbars. Of course (\ref{ex-hearty-cheer-after-cheer}) cannot be combined with determiners, so one would have to assume that
NPN constructions select for a determiner that has to be dropped obligatorily. This is also the case
for mass nouns with a certain reading.
}
\ea
\label{ex-npn-bragmann}
%\begin{sideways}
\oneline{%
\onems{
phon \phonliste{ \ldots{} N \ldots, after, \ldots{} N \ldots }\\
ss|loc|cat \ms{ head & \ms[noun]{ count & $-$\\
                                  agr   & \type{3rdsing}\\
                                }\\
                val  & \ms{ spr & \eliste\\
                            comps & \eliste\\
                          }\\
             }\\
sr \rm $\lambda P.\exists X.|X| >1~\&~\forall x \in X$$:N'(x)~\&~\exists R^{order} \subseteq X^{2}~\&~P(x)$ \\
dtrs \liste{ 
\onems{ phon \phonliste{ \ldots{} N \ldots }\\
        ss|l|c \ms{ head & \onems[noun]{ count $+$\\
                                      agr  \type{3rdsing}\\
                                }\\
                val  & \ms{ spr   & \sliste{ Det }\\
                            comps & \eliste\\
                          }\\
             }\\
        sr $\ldots~\lambda x.N'(x)~\ldots$
},
$\left( \onems{ phon \phonliste{ after }\\
            \ldots{} head \type{prep}\\
            sr $\exists R^{order} \subseteq X^{2}$\\
             },
\onems{ phon \phonliste{ \ldots{} N \ldots }\\
        ss|l|c \ms{ head & \onems[noun]{ count  $+$\\
                                         agr   \type{3rdsing}\\
                                }\\
                val  & \ms{ spr   & \sliste{ Det }\\
                            comps & \eliste\\
                          }\\
             }\\
        sr $\ldots~\lambda x.N'(x)~\ldots$
} \right)^+$
}% list
}
}
%\end{sideways}
\z
There is a list of daughters consisting of a first daughter and an arbitrarily long list of
\emph{after} N pairs. The `+'\is{+} means that there has to be at least one \emph{after} N pair. The
nominal daughters select for a determiner via \spr, so they can be either bare nouns or nouns
modified by adjectives. The semantic representation, non-standardly represented as the value of
\textsc{sr}, says that there have to be several objects in a set X ($\exists X.|X| >1$) and for all of them the meaning
of the \nbar has to hold ($\forall x \in X:N'(x)$). Furthermore there is an order between the elements of X as stated by $\exists R^{order} \subseteq X^{2}$.


%% A further piece of data that is interesting was also provided by Matsuyama: The nouns of the NPN
%% construction may take complements. (\mex{1}) is one of his examples \citep[\page 59]{Matsuyama2004a}:
%% \ea
%% \ldots{} and he kept asking question after question about the world that lay away down the river,
%% with all its perils and marvels
%% \z
%% silly question after silly question about the ...
%% This can be fixed easily in Bargmann's NPN construction by just assuming that 

From looking at this construction it is clear that it cannot be accounted for by standard \xbar
rules. Even without requiring \xbar syntactic rules, there seems to be no way to capture these
constructions in head-based approaches like \minimalism, \cg or \dg. For simple NPN constructions
one could claim that \emph{after} is the head. \emph{after} would be categorized as 3rd singular
mass noun and select for two \nbar{}s. It would (non-compositionally) contribute the semantics stated above. But it is unclear how the general schema with arbitrarily
many repetitions of \emph{after} N could be accounted for. If one assumes that \emph{day after day}
forms a constituent, then the first \emph{after} in (\mex{1}) would have to combine an N with an NPN sequence.
\ea
day after [day [after day]]
\z
This means that we would have to assume two different items for \emph{after}: one for the
combination of \nbar{}s and another one for the combination of \nbar with NPN combinations. Note
that an analysis of the type in (\mex{0}) would have to project information about the \nbar{}s contained
in the NPN construction since this information has to be matched with the single \nbar at the
beginning. In any case a lexical analysis would require several highly idiosyncratic lexical items
(prepositions projecting nominal information and selecting items they usually do not select).
It is clear that a reduplication account of the NPN construction as suggested by
G.\ \citet{GMueller2011a} does not work since patterns with several repetitions of PN as in
(\mex{0}) cannot be accounted for as reduplication. G.\ Müller (p.\,241) stated that reduplication works
for word-size elements only and hence his account would not extend to the English examples
containing adjectives.

%% \subsection{\emph{To hell with this government!}}

%% \citet{Jacobs2008a}

%% \citet[Section~21.10.1]{MuellerGT-Eng1}

This subsection showed how a special phrasal pattern can be analyzed within HPSG. The next section
will discuss filler-gap constructions, which were analyzed as instances of a single schema by
\citet{ps2} but which were later reconsidered and analyzed as a family of subconstructions by \citew{Sag2010b}.

\subsection{Specialized sub-constructions}

HPSG took over the treatment of nonlocal dependencies from GPSG \citep{Gazdar81} (see also
\crossrefchaptert{evolution} on the history of HPSG and \crossrefchaptert{udc} on
unbounded dependencies). \citet[Chapters~4 and~5]{ps2} had
an analysis of topicalization constructions like (\mex{1}) and an analysis of relative
clauses. However, more careful examination revealed that more fine-grained distinctions have to be
made. \citet[\page 491]{Sag2010b} looked at the following examples:
\settowidth\jamwidth{(topicalized clause)}
\eal
\ex {}[My bagels,] she likes.                           \jambox{(topicalized clause)}
\ex {}[\emph{What} books] do they like?                 \jambox{(\emph{wh}-interrogative)} 
\ex (the person) [\emph{who} (\emph{se} book)] they like \jambox{(\emph{wh}-relative)} 
\ex {}[\emph{What a} play] he wrote!                    \jambox{(\emph{wh}-exclamative)} 
\ex {}[\emph{the more} books] they read \ldots          \jambox{(the-clause)}
\zl
As Sag shows, the fronted element is specific to the construction at hand:
\eal
\ex[*]{
[\emph{Which} bagels] / [\emph{Who}], she likes. \jambox{(topicalized clause)}
}
\ex[*]{
[\emph{What a} book] do they like? \jambox{(\emph{wh}-interrogative)} 
}
\ex[\%]{
the thing [[\emph{what}] they like] \jambox{(\emph{wh}-relative)}
}
\ex[*]{
[\emph{Which} bagels] / [\emph{What}] she likes! \jambox{(\emph{wh}-exclamative)} 
}
\ex[*]{
[\emph{which} books] they read, the more they learn. \jambox{(the-clause)}
}
\zl
A topicalized clause should not contain a \emph{wh} item (\mex{0}a), a \emph{wh}-interrogative should not
contain a \emph{what a} sequence as known from \emph{wh}-exclamatives (\mex{0}b) and so on.

Furthermore, some of these constructions allow non-finite clauses and others do not:
\eal
\ex[*]{
Bagels, (for us) to like. \jambox{(topicalized clause)}
}
\ex[*]{
It's amazing [what a dunce (for them) to talk to]. \jambox{(\emph{wh}-exclamative)} 
}
\ex[*]{
The harder (for them) to come, the harder (for them) to fall.  \jambox{(the-clause)}
}
\ex[]{
I know how much time (* for them) to take.  \jambox{(\emph{wh}-interrogative)} 
}
\ex[]{
The time in which (*for them) to finish.  \jambox{(\emph{wh}-relative)}
}
\zl
So there are differences as far as fillers and as far as sentences from which something is extracted
are concerned. Sag discussed further differences like invertion/""non-invertion in the clauses out of
which something is extracted. I do not repeat the full discussion here but refer the reader to the
original paper.

In principle there are several ways to model the phenomena. One could assume empty
heads as \citet[Chapter~5]{ps2} suggested for the treatment of relative clauses. Or one could assume empty
heads as they are assumed in Minimalism: certain so-called operators have features that have to be
checked and cause items with the respective properties to move\todostefan{reference}. \citet{Borsley2006a} discussed potential
analyses of relative clauses involving empty heads and showed that one would need a large number of
such empty heads and since there is no theory of the lexicon in Minimalism, generalizations are
missed (see also \crossrefchaptert{minimalism}). The alternative suggested by
\citet{Sag2010b} is to assume a general Filler-Head Schema of the kind assumed in \citew{ps2} and
then define more specific sub-constructions. To take an example, the wh-exclamative is a filler-head
structure, so it inherits everything from the more general construction, but in addition it
specifies the filler daughter to contain a \emph{what a} part and states the semantics that is
contributed by the exclamative construction.


\section{Summary}
\label{sec-summary}

This paper summarized the properties of Construction Grammar or rather Construction Grammars and
showed that HPSG can be seen as a Construction Grammar. I showed why lexical analyses of argument structure should be
preferred over phrasal ones and that there are other areas in grammar where phrasal analyses are
superior to lexical ones. I showed that they can be covered in HPSG while they are problematic for
proposals assuming that all structures have to have a head. 


\is{Construction Grammar|)}
%\section*{Abbreviations}
\section*{Acknowledgements}

I thank Bob Borsley, Rui Chaves, and Jean-Pierre Koenig for comments on the outline for this chapter and for discussion in general.

I thank Bob Borsley for comments on an earlier version of this paper.

{\sloppy
\printbibliography[heading=subbibliography,notkeyword=this] 
}
\end{document}

\if0

Stefan


Your outline made me think about what exactly construction grammar is. It seems to me it’s not a simple matter. It might be seen as any approach which rejects the idea emphasized by Chomskyans that traditional constructions are epiphenomena. Constructions appear to be just epiphenomena in any approach in which the syntax is just a few general combinatorial mechanisms, so not just P&P and Minimalism but also categorial grammar and early HPSG with its few ID schemata. This might mean that construction grammar is any framework which has more than just a few general combinatorial mechanisms. That would include HPSG since Sag (1997), but also GPSG with its numerous ID rules and classical TG with its numerous PS rules. But I assume most people wouldn’t see either GPSG and classical TG as forms of Construction Grammar.

 

This might suggest that form-meaning pairs is what’s key for construction grammar. That would include early GPSG with its pairs of syntactic and semantic rules, but not, I think, later GPSG with its few general semantic principles. It would also, I think, not include HPSG, since an HPSG phrase type may or may not have any semantic properties.

 
I’m inclined to see the most important feature of construction grammar as complex hierarchies of phrase types, allowing the grammar to capture very broad general facts, very specific facts, and everything in between. This includes HPSG since Sag (1997), but not early HPSG, and not P&P and Minimalism, categorial grammar, GPSG, or classical TG. However, I don’t know whether all frameworks calling themselves construction grammar have complex hierarchies of phrase types. If not, I don’t know how I would define construction grammar. 


best


Bob



JP:

 This is a nice outline. I will make some high level remarks in the context of Bob’s comments. It’s clear that a decision was made to focus on that part of ConsGr that is closest to HPSG. It certainly makes sense for the Handbook, but of course the center of gravity of ConsGr is not that part which was closest to HPSG around 2012. So, I wonder a little more history of what ConsGram was is needed, particularly work by Fillmore and Kay (I assume Goldberg will be discussed in section 3), which presumably should be cited. Two things in particular might be stressed:

1) Typing of constructions was not part of the original ConsGr if memory serves me right,
2) Constructions that are not of depth 1 and thus resolute non-locality as a possibility was parts and parcels of ConsGram. 

Finally, assigning meaning to phrase structure combinations (and meaning that can be quite more detailed than a standard type-logical meaning, see the Let alone paper, is a big thing of ConsGram and something that clearly can be done in HPSG (see Ginzburg and Sag), but is not always embraced. So, I remember Frank telling me LRS typically does not go for that, although he is quick to acknowledge nothing prevents it.



As for comments, I suggest you discuss SBCG, and in particular, cite
Boas & Sag 2012. I also suggest discussing Lee-Goldman 2011, a SBCG
dissertation from Berkeley available at
https://escholarship.org/uc/item/02m6b3hx

Finally, note that Adele Goldberg has now a new book  "Explain me
this: Creativity, Competition, and the Partial Productivity of
argument structure constructions" Princeton University, which you may
want to check too.



\fi



%      <!-- Local IspellDict: en_US-w_accents -->
