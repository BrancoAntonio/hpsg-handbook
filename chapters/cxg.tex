\documentclass[output=paper]{langsci/langscibook} 
\author{Stefan Müller\affiliation{Humboldt-Universität zu Berlin}}
\title{HPSG and Construction Grammar}

% \chapterDOI{} %will be filled in at production

%\epigram{Change epigram in chapters/03.tex or remove it there }
\abstract{}
\maketitle

\begin{document}

\section{What is Construction Grammar?}

\citet{Goldberg96a,Goldberg2006a,Michaelis2012a}

\begin{itemize}
\item form-meaning pairs
\item language acquisition without (much) UG
\item no empty elements
\end{itemize}

Historical aspects:
\begin{itemize}
\item non-locality \citep{FKoC88a}
\item type inheritance \citet{KF99a,Sag97a}
\end{itemize}

\section{HPSG as a Construction Grammar}

\begin{itemize}
\item form-meaning pairs
\item type hierarchies
\item surface oriented
\end{itemize}

\section{Valence vs.\ phrasal patterns}

\citet{Goldberg96a,Goldberg2006a,GJ2004a}
 
\citet{Mueller2006d,MWArgSt,MuellerLFGphrasal}

\section{Phrasal patterns}

Why more than just binary branching abstract schemata are needed:
\begin{itemize}
\item \citet{Jackendoff2008a}: NPN construction
\item \citet{Jacobs2008a}
\item \citet[Section~21.10.1]{MuellerGT-Eng1}
\end{itemize}

Specialized constructions for special cases, \eg \citew{Sag2010b}.

%\section*{Abbreviations}
\section*{Acknowledgements}

I thank Bob Borsley, Rui Chaves, and Jean-Pierre Koenig for comments on the outline for this chapter and for discussion in general.

\printbibliography[heading=subbibliography,notkeyword=this] 
\end{document}

\if0

Stefan


Your outline made me think about what exactly construction grammar is. It seems to me it’s not a simple matter. It might be seen as any approach which rejects the idea emphasized by Chomskyans that traditional constructions are epiphenomena. Constructions appear to be just epiphenomena in any approach in which the syntax is just a few general combinatorial mechanisms, so not just P&P and Minimalism but also categorial grammar and early HPSG with its few ID schemata. This might mean that construction grammar is any framework which has more than just a few general combinatorial mechanisms. That would include HPSG since Sag (1997), but also GPSG with its numerous ID rules and classical TG with its numerous PS rules. But I assume most people wouldn’t see either GPSG and classical TG as forms of Construction Grammar.

 

This might suggest that form-meaning pairs is what’s key for construction grammar. That would include early GPSG with its pairs of syntactic and semantic rules, but not, I think, later GPSG with its few general semantic principles. It would also, I think, not include HPSG, since an HPSG phrase type may or may not have any semantic properties.

 
I’m inclined to see the most important feature of construction grammar as complex hierarchies of phrase types, allowing the grammar to capture very broad general facts, very specific facts, and everything in between. This includes HPSG since Sag (1997), but not early HPSG, and not P&P and Minimalism, categorial grammar, GPSG, or classical TG. However, I don’t know whether all frameworks calling themselves construction grammar have complex hierarchies of phrase types. If not, I don’t know how I would define construction grammar. 


best


Bob



JP:

 This is a nice outline. I will make some high level remarks in the context of Bob’s comments. It’s clear that a decision was made to focus on that part of ConsGr that is closest to HPSG. It certainly makes sense for the Handbook, but of course the center of gravity of ConsGr is not that part which was closest to HPSG around 2012. So, I wonder a little more history of what ConsGram was is needed, particularly work by Fillmore and Kay (I assume Goldberg will be discussed in section 3), which presumably should be cited. Two things in particular might be stressed:

1) Typing of constructions was not part of the original ConsGr if memory serves me right,
2) Constructions that are not of depth 1 and thus resolute non-locality as a possibility was parts and parcels of ConsGram. 

Finally, assigning meaning to phrase structure combinations (and meaning that can be quite more detailed than a standard type-logical meaning, see the Let alone paper, is a big thing of ConsGram and something that clearly can be done in HPSG (see Ginzburg and Sag), but is not always embraced. So, I remember Frank telling me LRS typically does not go for that, although he is quick to acknowledge nothing prevents it.



As for comments, I suggest you discuss SBCG, and in particular, cite
Boas & Sag 2012. I also suggest discussing Lee-Goldman 2011, a SBCG
dissertation from Berkeley available at
https://escholarship.org/uc/item/02m6b3hx

Finally, note that Adele Goldberg has now a new book  "Explain me
this: Creativity, Competition, and the Partial Productivity of
argument structure constructions" Princeton University, which you may
want to check too.



\fi



%      <!-- Local IspellDict: en_US-w_accents -->
