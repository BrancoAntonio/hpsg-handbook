\documentclass[output=paper]{langsci/langscibook} 
\author{Manfred Sailer\affiliation{Goethe-Unversität Frankfurt}}
\title{Idioms}

% \chapterDOI{} %will be filled in at production

%\epigram{Change epigram in chapters/03.tex or remove it there }
\abstract{
This chapter first sketches basic empirical properties of idioms. The state of the art before the emerge of HPSG is presented, followed by a discussion of four types of HPSG-approaches to idioms. A section on future research closes the discussion.
}
\maketitle

\begin{document}
\label{chap-idioms}


%%% Index cross references
\is{MWE|see{multiword expression}}
%%%


\kommentar{
\eal
\ex 
\gll dass er dem Mann das Buch gab\\
     that he the man the book gave\\
\glt `that he gave the man the book'
\ex
\gll dass er versucht, [dem Mann das Buch zu geben]\\
     that he tried     \spacebr{}the man the book to give\\
\glt `that he tried to give the man the book'
\zl

\cite{RS2009a}

In a previous paper,
\cite{Sailer:12}, I gave an overview over the development of idiom research in HPSG. 
Since that paper was in German and published in a Festschrift, its accessibility to a wider audience is rather limited. The planned chapter will follow the structure of this earlier paper, but take into consideration important recent developments in HPSG idiom research and relate it to the development in other formal frameworks, TAG \citep{Abeille:95,Lichte:Kallmeyer:16} and Minimalism \citep{vCraenenbroeck:al:16draft}.

The basic analytic challenge of idioms for HPSG is the fact that idioms are perceived as multiword, phrasal units, while the HPSG is strongly lexical, i.e., within the standard formalization of the framework \citep{King89,Richter:04}, it is impossible to implement an \emph{en bloc}-insertion theory of idioms. I will show in this paper that HPSG-accounts moved from trying to simulate a phrasal analysis to reaching the conviction that a lexical analysis is, indeed, empirically motivated.

}

\section{Introdction}
\label{Sec-Intro}
%\section{Introduction}

In this chapter, I will use the term \emph{idiom} interchangeably with the broader terms such as \emph{phraseme}, \emph{phraseologism}, \emph{phraseological unit}, or \emph{multiword expression}. This means, that I will subsume under this notion expressions such as prototypical idioms (\bspT{kick the bucket}{die}), support verb constructions (\bsp{take advantage}), formulaic expression (\bsp{Good morning!}) and many more.%
\footnote{I will provide a paraphrase for all idioms at their first mention. They are also listed in the appendix, together with their paraphrase and a remark on which aspects of the idiom are discussed in the text.}
The main focus of the discussion will, however, be on prototypical idioms, as these have been in the center of the theoretical development.

%In the rest of this section, 
I will sketch some empirical aspects of idioms in Section~\ref{Sec-EmpiricalDomain}.
%and, then, characterize theoretical issues that arise in the formal modelling of idioms, both in general and with respect to HPSG in particular. 
In Section~\ref{Sec-Predecessors}, I will present the theoretical context within which idiom analyses arose in HPSG.
An overview of the development within HPSG will be given in Section~\ref{Sec-Analyses}. 
%Section~\ref{Sec-RecentOtherFrameworks} contains a brief sketch of the theoretical development outside HPSG. 
Desiderata for future research are mentioned in Section~\ref{Sec-WhereToGo}, before I close with a short conclusion.

%\subsection
\section{Empirical domain}
\label{Sec-EmpiricalDomain}
%Defining the empirical domain of idioms and phraseology. I will aim at a very inclusive definition, i.e., more in lines with ``phraseology'' than with ``idioms'' in the strict sense.

%I will assume the basic characterization of a phraseological unit from \cite{Fleischer97a-u} and \cite{Burger:98} as \emph{complex} units that show at least one of \emph{fixedness}, \emph{(semantic) idiomaticity}, and \emph{lexicalization}.

\is{idiomaticity|(}
In the context of the present handbook, the most useful characterization of idioms might be the definition of \emph{multiword expression}\is{multiword expression} from \cite{Baldwin:Kim:10}.
%\footnote{See also \cite{Sailer:18SemComp} for a more detailed summary of }
For them, any expression counts as a multiword expression if it is syntactically complex and shows some degree of \emph{idiomaticity} (i.e., irregularity), be it lexical, syntactic, semantic, pragmatic, or statistical.%
\footnote{In the phraseological tradition the aspect of \emph{lexicalization} is added \citep{Fleischer97a-u,Burger:98}. This means that an expression is stored in the lexicon. This criterion might have the same coverage as \emph{conventionalization} used in \cite{NSW94a}. 
These criteria are addressing the mental representation of idioms as a unit and are, thus, rather psycholinguistic in nature.}
%
\citeauthor{Baldwin:Kim:10}'s criteria can help us structure the data presentation in this section, expanding them where it seems suitable.
My expansions concern the aspects known under \emph{fixedness} in the phraseological tradition as in \cite{Fleischer97a-u}.%
\footnote{\cite{Baldwin:Kim:10} use describe idioms in terms of syntacitc fixedness, but they seem to consider it a derived notion.}

%\medskip%
\is{idiomaticity!lexical|(}
For \cite{Baldwin:Kim:10}, \emph{lexical idiosyncracy} concerns expressions with words that only occur in an idiom, so-called \emph{phraseologically bound words}, or \emph{cranberry words} \citep{Aronoff76a-u}. Examples include \bspT{make headway}{make progress}, \bspT{at first blush}{at first sight}, \bspT{in a trice}{in a moment/""very quickly}.%
\footnote{See https://www.english-linguistics.de/codii/, accessed 2018-07-25, for a list of bound words in \ili{English} and \ili{German} \citep{Trawinski:al:08lrec}.}
For such expressions, the grammar has to make sure that the bound word does not occur outside the idiom, i.e., we need to prevent combinations such as \refer{trice-ko}.

\eal \label{trice}
\ex  They fixed the problem in a trice.
\ex *It just took them a trice to fix the problem.\label{trice-ko}
\zl 

We can expand this type of idiosyncrasy to include  a second important property of idioms. 
Most idioms have a fixed inventory of words. In their summary of this aspect of idioms, \cite[\page 827--828]{Gibbs:Colston:07} include the following examples: \bsp{kick the bucket} means \bsp{die}, but \bsp{kick the pail}, \bsp{punt the bucket}, or \bsp{punt the pail} do not have this meaning. However, some degree of lexical variation seems to be allowed, as
the idiom \bspT{break the ice}{releave tension in a strained situation} can be varied into \bsp{shatter the ice}.%
\footnote{\label{fn-semmeln}While \cite{Gibbs:Colston:07}, following \cite{Gibbs:al:89}, present this example as a lexical variation, \cite[\page 85]{Glucksberg:01}, from which it is taken, clearly characterizes as having a somewhat different aspect of an ``abrupt change in the social climate''. Clear cases of synonymy under lexical substitution are found with \ili{German} \bspTL{wie warme Semmeln/Brötchen/Schrippen weggehen}{like warm rolls vanish}{sell like hotcakes} in which some regional terms for rolls can be used in  the idiom.}
%For example, \cite[\page 85]{Glucksberg:01} mentions the variation \bsp{shatter the ice} of the idiom \bspT{break the ice}{XXX}.
 So, a challenge for idiom theories is to guarantee that the 
right lexical elements are used in the right constellation.
\is{idiomaticity!lexical|)}

%\medskip%
\is{idiomaticity!syntactic|(}
\emph{Syntactic idiomaticity} is used in \cite{Baldwin:Kim:10} to describe expressions that are not formed according to the productive rules of English syntax, following \cite{FKoC88a}, such as \bspT{by and large}{on the whole/""everything considered}, \bspT{trip the light fantastic}{dance}.  

In my expanded use of this notion, this also subsumes irregularities/""restrictions in the syntactic flexibility of an idiom, i.e., the question whether an idiom can occur in the same syntactic constructions as an analogous non"=idiomatic combination. In transformational grammar, such as \cite{Weinreich:69} and \cite{Fraser:70}, lists of different syntactic transformations were compiled and it was observed that some idioms allow for certain transformations but not for others. This method has been pursued systematically in the framework of \emph{Lexicon-Grammar}\is{Lexicon-Grammar} \citep{Gross:75}.%
\footnote{See \cite{Laporte:18} for a recent discussion of applying this method for a classification of idioms.}
%
\cite{SBBCF2002a-ausgedruckt} distinguish three levels of fixedness: \emph{fixed}, \emph{semi-fixed}, and \emph{flexible}. 
Completely fixed idioms include \bsp{of course}, \bsp{ad hoc} and are often called \emph{words with spaces}\is{word with spaces}.
Semi-fixed idioms allow for morpho-syntactic variation, such as inflection. These include some prototypical idioms (\bsp{trip the light fantastic}, \bsp{kick the bucket}) and complex proper names. In English, semi-flexible idioms show inflection, but cannot easily be passivized or allow for parts of it to be topicalized, see \refer{kick-ex}.

\eal \label{kick-ex} 
\ex Alex kicked / might kick the bucket.
\ex *The bucket was kicked by Alex.
\ex *The bucket, Alex kicked.
\zl 

%For other languages, such as \ili{Dutch} and \ili{German}, \cite{Schenk:95} claims that other automatic/meaningless syntactic processes should be included as well, such as verb-second movement\is{verb second} and some types of fronting.


Flexible idioms pattern with free combinations. For them, we do not only find inflection, but also passivization, topicalization or pronominalization of parts etc. Free combinations include some prototypical idioms (\bspT{spill the beans}{reveal a secret}, \bspT{pull strings}{exert influence/""use one's connections}), but also collocations (\bsp{brush one's teeth}) and light verbs (\bsp{make a mistake}).

The assumption of two flexibility classes is not uncontroversial: 
\cite{Horn:03} distinguishes two types among what \cite{SBBCF2002a-ausgedruckt} consider flexible idioms. 
\cite{Fraser:70} assumes six flexibility classes, looking at a wide range of syntactic operations.
\cite{Ruwet:91} takes issue with the cross-linguistical applicability of the classification of syntactic operations. Similarly, \cite{Schenk:95} claims that for languages such as \ili{Dutch} and \ili{German}, automatic/meaningless syntactic processes other than just inflection are possible for semi-fixed idioms, such as verb-second movement\is{verb second} and some types of fronting.

The analytic challenge of syntactic idiomaticity is to capture the difference in flexibility in a non"=ad hoc way. It is this aspect of idioms that has received particular attention in Generative linguistics, but also in HPSG.%
\is{idiomaticity!syntactic|)}

%\medskip%
\is{idiomaticity!semantic|(}
\emph{Semantic idiomaticity} may sound pleonastic, as, traditionally, an expression is called idiomatic if it has a conventional meaning that is different from its literal meaning. 
Since I use the terms idiom and idiomaticity in a broader sense, the qualification \emph{semantic} idiom(aticity) is needed. 

One challenge of the modelling of idioms is to capture the relation between the literal and the idiomatic meaning of an expression.
\cite{Gibbs:Colston:07} give an overview over 
psycholinguistic research on idioms. Whereas it was first assumed that speakers would compute the literal meaning of an expression and, then, derive the idiomatic meaning, evidence has been accumulated that the idiomatic meaning is accessed directly.

\cite{WSN84a-u} and \cite{NSW94a} explore various semantic relations for idioms, in particular \emph{decomposability} \is{decomposability} and \emph{transparency}\is{transparency}.
An idiom is \emph{decomposable} if its idiomatic meaning can be distributed over its component parts in such a way that we would arrive at the idiomatic meaning of the overall expression if we interpreted the syntactic sturcture on the basis of such a meaning assignment. 
The idiomatic meaning of the expression \bsp{pull strings} can be decomposed by interpreting \bsp{pull} as \bsp{exploit/use} and \bsp{strings} as \bsp{connections}. 
%To make this criterion non"=arbitrary, it is now common to require that there be empirical support for such a decomposition. In English, we can often insert an adjective 
The expressions \bsp{kick the bucket} and \bspT{saw logs}{snore} are not decomposable.

An idiom is \emph{transparent} if there is a synchronically accessible relation between the literal and the idiomatic meaning of an idiom. 
For some speakers, \bsp{saw logs} is transparent in this sense, as the noise produced by this activity is similar to a snoring noise. 
For \bsp{pull strings}, there is an analogy to a puppeteer controlling the puppets' behavior by pulling strings. A non"=transparent idiom is called \emph{opaque}. 

Some idioms do not show semantic idiomaticity at all, such as collocations\is{collocation} (\bsp{brush one's teeth}) or support verb constructions (\bsp{take a shower}). 
Many body-part expressions such as \bspT{shake hands}{greet} or \bspT{shake one's head}{decline/""negate} constitute a more complex case they describe a conventionalized activity and denote the social meaning of this activity \citep{Burger:76}.

In addition, we might need to assume a \emph{figurative} interpretation. For some expressions, in particular proverbs\is{proverb} or cases like 
\bsp{take the bull by the horns}) we might get a figurative reading rather than an idiomatic reading. 
%
\cite{Glucksberg:01} explicitly distinguishes between idiomatic and figurative interpretations. In his view, the above-mentioned case of \bsp{shatter the ice} would be a figurative use of the idiom \bsp{break the ice}. 
While there has been a considerable amount of work on figurativity in psycholinguistics, the integration of its results into formal linguistics is still a desideratum.%
\is{idiomaticity!semantic|)}

%\medskip%
\is{idiomaticity!pragmatic|(}
\emph{Pragmatic idiomaticity} covers expressions that have a \emph{pragmatic point} in the terminology of \cite{FKoC88a}. These include complex formulaic expressions (\bsp{Good morning!}). There has been little work on this aspect of idiomaticity in formal phraseology.
\is{idiomaticity!pragmatic|)}

%\medskip%
\is{idiomaticity!statistical|(}
The final type of idiomaticity is \emph{statistical idiomaticity}. 
Contrary to the other idiomaticity criteria, this is a usage-based aspect. If we find a high degree of co-occurrence of a particular combination of words that is idiosyncratic for this combination, we can speak of a statistical idiomaticity. This category includes \emph{collocations}\is{collocation}. \cite{Baldwin:Kim:10} mention \bsp{immaculate performance} as an example. Collocations are important in computational linguistics and in foreign-language learning, but their status for theoretical linguistics and for a competence-oriented framework such as HPSG is unclear. 
\is{idiomaticity!statistical|)}

%\bigskip%
This discussion of the various types of idiomaticity shows that idioms do not form a homogeneous empirical domain but are rather defined negatively. 
This leads to the basic analytical challenged of idioms: while the empirical domain is defined by  absence of regularity in at least one aspect, idioms largely obey the principles of grammar. 
In other words, there is a lot of regularity in the domain of idioms, while any approach still needs to be able to model the irregular properties. 
%I have tried to introduce the notions that are most commonly used in HPSG research and to identify the analytical problems related to them. 


\is{idiomaticity|)}

%\subsection{Language-theoretical interest in idioms}
%\label{Sec-TheoreticalInterest}

%\begin{itemize}
%\item between rule-based and idiosyncratic
%\item compositional challenge and collocational challenge
%\end{itemize}


\section{Predecessors to HPSG analyses of idioms}
\label{Sec-Predecessors}


In this section, I will sketch the theoretical environment within which HPSG and HPSG analyses of idioms have emerged.

The general assumption about idioms in Generative grammar is that they
must be represented as a complex phrasal form-meaning unit. 
Such units are inserted \emph{en bloc}\is{en bloc insertion} into the structure rather than built by syntactic operations.
This view goes back to \cite[\page 190]{Chomsky:65}. 
With this unquestioned assumption, arguments for or against particular analyses can be constructed. 
To give just one classical example, \cite{Chomsky81a} uses the passivizabilty of some idioms as an argument for the existence of Deep Structure, i.e.\@ a structure on which the idiom is inserted holistically. 
%
\cite{Ruwet:91} and \cite{NSW94a} go through a number of such lines of argumentation showing their basic problems. 

The holistic view on idioms is most plausible for idioms that show many types of idiomaticity at the same time, though it becomes more and more problematic if only one or only a few types of idiomaticity are attested.
HPSG is less driven by analytical pre-decisions than other frameworks, nonetheless, idioms have been used to motivate assumptions about the architecture of linguistic signs.

\cite{WSN84a-u} and \cite{NSW94a} are probably the two most influential papers in formal phraseology in the last decades. 
%These papers have also shaped the analysis of idioms in \emph{Generalized Phrase Structure Grammar} \citep{GKPS85a}\is{Generalized Phrase Structure Grammar} and, consequently in HPSG. 
While there are many aspects of \cite{NSW94a} that have not been integrated into the formal modelling of idioms, 
there are at least two insights that have been widely adapted in HPSG.
First, not all idioms should be represented holistically. 
Second, the syntactic flexibility of an idiom is related to its semantic decomposability. In fact, they state this last insight even more generally:%
\footnote{Aspects of this approach are already present in \cite{Higgins:74} and \cite{Newmeyer:74}.}

\ea \label{NSW-quote} \citet[\page 531]{NSW94a}:
\begin{quotation}
We predict that the syntactic flexibility of a particular idiom will ultimately be explained in terms of the compatibility of its semantics with the semantics and pragmatics of various constructions.
\end{quotation}
\z 

%One of the theoretical predecessors of HPSG is \emph{Generalized Phrase Structure Grammar} \citep{GKPS85a}\is{Generalized Phrase Structure Grammar}. 


\cite{WSN84a-u} and \cite{NSW94a} propose a simplified first approach to a theory that would be in line with this quote. They argue that, for English, there is a correlation between syntactic flexibility and semantic decomposability in that non"=decomposable idioms are only semi-flexible, whereas decomposable idioms are flexible, to use our terminology from Section~\ref{Sec-EmpiricalDomain}. 
\is{Generalized Phrase Structure Grammar|(}
This idea has been directly encoded formally in the idiom theory of
\cite{GKPS85a}, who define the framework of 
\emph{Generalized Phrase Structure Grammar}\is{Generalized Phrase Structure Grammar} (GPSG).

\cite{GKPS85a} assume that  non"=decomposable idioms are inserted into sentences \emph{en bloc}\is{en bloc insertion}, i.e.\@ as a fully specified syntactic trees which are assigned the idiomatic meaning holistically. This means that the otherwise strictly context-free grammar of GPSG needs to be expanded by adding a (small) set of larger trees. 
Since non"=decomposable idioms are inserted as units, their parts cannot be accessed for syntactic operations, such as passivization or movement. Consequently, the generalization about semantic non"=decomposability and syntactic fixedness of English idioms from \cite{WSN84a-u} is implemented directly.

Decomposable idioms are analyzed just as free combinations in syntax. The idiomaticity of such expressions is achieved by two assumptions: First, there is lexical ambiguity, i.e. for an idiom like \bsp{pull strings}, the verb \bsp{pull} has both a literal meaning and an idiomatic meaning. Similarly for \bsp{strings}.
Second, \cite{GKPS85a} assume that lexical items are not necessarily translated into total functions but can be partial functions. Whereas the literal meaning of \bsp{pull} might be a total function, the idiomatic meaning of the word would be a partial function that is only defined on elements that are in the denotation of the idiomatic meaning of \bsp{strings}. This analysis predicts syntactic flexibility for decomposable idioms, just as proposed in \cite{WSN84a-u}.


\is{Generalized Phrase Structure Grammar|)}



%Related to this general dilemma are two more concrete analytical challenges, which \cite[\page 12]{Bargmann:Sailer:18} call the \emph{compositional challenge} and the \emph{collocational challenge}. The compositional challenge consists in associating a sequence of words with a non"=literal, idiomatic, meaning. The collocational challenge consists in making sure that the components of an idiom all occur together in the right constellation.


%\section{Predecessors to HPSG analyses of idioms}
%\label{Sec-Predecessors}

%\begin{itemize}
%\item Generalized Phrase Structure Grammar \citep{GKPS85a}
%\item Semi-formal, influential papers \citep{WSN84a-u,NSW94a}
%\item Construction Grammar \citep{FKoC88a}
%\end{itemize}


\section{HPSG-analyses of idioms}
\label{Sec-Analyses}

%\subsection{HPSG-specific research questions}

\kommentar{
\begin{itemize}
\item Phrasal entities in a lexical framework? i.e., basic problem of phrasal vs.\@ lexical analyses in a framework like HPSG.
\end{itemize}
}

HPSG does not make a core-periphery distinction. Consequently, idioms belong to the empirical domain to be covered by an HPSG grammar.
Nonetheless, idioms are not discussed in \cite{ps2} and their architecture of grammar does not have a direct place for an analysis of idioms.%
\footnote{This section follows the basic structure and argument of \cite{Sailer:12} and \cite{Richter:Sailer:14}.} 
They situate all idiosyncrasy in the lexicon, which consists of lexical entries for words. 
\marginpar{Crossref to lexicon chapter}
Every word has to satisfy a lexical entry and all principles of grammar, see Chapter~\crossrefchaptert{chap-lexicon}.%
\footnote{The lexicon can often be expressed by a \emph{Word Principle}, a constraint on words that contains a disjunction of all lexical entries.\is{Word Principle}\is{lexicon}.}
%
All properties of a phrase can be inferred from the properties of the lexical items occurring in the phrase and the constraints of grammar. 

In their grammar, \cite{ps2} adhere to  \emph{Strict Locality Hypothesis} (SLH),\is{locality!Strong Locality Hypothesis}\is{locality|(} i.e., all lexical entries describe leaf nodes in a syntactic structure, all phrases are constrained by principles that only refer to local (i.e., \type{synsem}) properties of the phrase and to local properties of its immediate daughters. This hypothesis is summarized in \refer{slh}.

\vbox{
\ea Strong Locality Hyphothesis\label{slh} (SLH)\\
The rules and principles of grammar are statements on a single node of a linguistic structure of on nodes that are immediately dominated by that node.
\z 

}

This precludes any purely phrasal approaches to idioms. 
Following the heritage of GPSG\is{Generalized Phrase Structure Grammar}, we would assume that all regular aspects of linguistic expressions can be handled by mechanisms that follow the SLH, 
whereas idiomaticity would be a range of phenomena that may violate it. 
It is, therefore, remarkable that a grammar framework that denies a core-periphery distinction would start with a strong assumption of regularity. 

This is in sharp contrast to the basic motivation of Construction Grammar\is{Construction Grammar}, which assumes that constructions can be of arbitrary depth and of an arbitrary degree of idiosyncrasy. 
\cite{FKoC88a} use idiom data and the various types of idiosyncrasy discussed in Section~\ref{Sec-EmpiricalDomain}
as an important motivation for this assumption. 
To contrast this position clearly with the one taken in \cite{ps2}, I will state the \emph{Strong Non"=locality Hypothesis}\is{locality!Strong Non"=locality Hypothesis} (SNH) in \refer{snh}. 

\vbox{
\ea Strong Non"=locality Hypothesis (SNH)\label{snh}\\
The internal structure of a construction can be arbitrarily deep and show an arbitrary degree of irregularity at any substructure.
\z

}


\marginpar{Crossref to formal background chapter}
The actual formalism used in \cite{ps2}, \cite{King89} -- see Chapter \crossrefchaptert{chap-formal-background} -- does not require the strong versions of the locality and the non"=locality hypotheses, but is compatible with a weaker versions. I will call these the \emph{Weak Locality Hypothesis} (WLH),\is{locality!Weak Locality Hypothesis} and the 
\emph{Weak Non"=locality Hypothesis} (WNH), see \refer{wlh} and \refer{wnh} respectively.

\vbox{
\ea  Weak Locality Hypothesis (WLH)\label{wlh}\\
At most the highest node in a structure is licensed by a rule of grammar or a lexical entry.
%The rules and principles of grammar can constrain the internal structure of a linguistic sign at arbitrary depth, but each sign needs to be licensed independently.
\z 

}

According to the WLH, just as in the SLH, each sign needs to be licensed by the lexicon and/or the grammar. 
This precludes any \emph{en bloc}-insertion analyses, which would be compatible with the SNH.
%
According to the WNH, in line with the SLH, a sign can, however, impose further constraints on its component parts, that may go beyond local (i.e., \type{synsem}) properties of its immediate daughters.

\ea Weak Non"=locality Hypothesis (WNH)\label{wnh}\\
The rules and principles of grammar can constrain -- though not license -- the internal structure of a linguistic sign at arbitrary depth.
\z 




\kommentar{
For this reason, HPSG approaches have attempted to argue that there is more regularity than meets the eye and, basically, try to treat idioms very much like free combinations. 
This is also the perspective in which idioms have been used to argue for or against a particular organization of linguistic signs: attributes and sorts can be motivated just by their necessity for an analysis of idioms. 
}

In this section, I will review four types of analyses developed within HPSG, in a mildly synchronic order:
First, I will discuss a conservative extension of \cite{ps2} for idioms \citep{KE94a} that sticks to the SLH. 
Then, I will look at attempts to incorporate constructional ideas more directly, i.e., ways to include a version of the SNH. 
The third type of approach will exploit the WLH. Finally, I will summarize recent apporoaches, which are, again, emphasizing the locality of idioms.



\is{locality|)}

\subsection{Early lexical approaches}
\label{Sec-EarlyLexical}

\kommentar{
\begin{itemize}
\item \cite{KE94a}
%\cite{Krenn:Erbach:94}
\end{itemize}

\cite{SS2003a}
}


\cite{KE94a}, based on \cite{Erbach92a}, present the first comprehensive HPSG account of idioms. 
They look at a wide variety of different types of German idioms, including support verb constructions. 
They only modify the architecture of \cite{ps2} marginally and stick to the Strict Locality Hypothesis. 
They base their analysis on the apparent correlation between syntactic flexibility and semantic decomposability from \cite{WSN84a-u,NSW94a}. Their analysis is a representational variant of the analysis in \cite{GKPS85a}.

To maintain the SLH, \cite{KE94a} assume that the information available in syntactic selection is slightly richer than what has been assumed in \cite{ps2}:
First, they use a lexeme-identification feature, \attrib{lexeme}, which is located inside the \attrib{index} value and whose value is the semantic constant associated with a lexeme. 
Second, they include a feature \attrib{theta-role}, whose value indicates which thematic role a sign is assigned in a structure. In addition to standard thematic roles, they include a dummy value \type{nil}.
Third, as the paper was written in the transition phase between \cite{ps} and \cite{ps2}, they assume that the selectional attributes contain complete \type{sign} objects rather than just \type{synsem} objects. 
Consequently, they can assume selection for phonological properties and internal constituent structure, which we could consider a violation of the SLH. 

The effect of these changes in the analysis of idioms can be seen in \refer{ke-spill} and \refer{ke-kick}. In \refer{ke-spill}, I sketch the analysis of syntactically flexible, decomposable idiom, \emph{spill the beans}.
There are individual lexical items for the idiomatic words. 


\eal % Analysis of \emph{spill the beans} in the spirit of \cite{KE94a}
\label{ke-spill}
\ex 
\ms{phon & \phonliste{spill}\\
synsem & \ms{cat & \ms{subcat & \liste{NP, NP\ms{lexeme & beans\_i}}}\\
            cont & \ms{rel & spill\_i}
}}
\ex 
\ms{phon & \phonliste{beans}\\
synsem & \ms{content & \ms{index & \ms{lexeme & beans\_i}}}}
\zl 

The \attrib{lexeme} values of these words can be used to distinguish them from their ordinary, non"=idiomatic, homonyms. 
Each idiomatic word comes with its idiomatic meaning, which models the decomposability of the expression. 
The lexical items satisfying the entries in \refer{ke-spill} can undergo lexical rules such as passive. 

The idiomatic verb \emph{spill} selects an NP complement with the \attrib{lexeme} value \type{beans\_i}. 
The lexicon is built in such a way that no other word selects for this \attrib{lexeme} value. 
This models the lexical fixedness of the idiom.

The choice of putting the lexical identifier into the \attrib{index} guarantees that it is shared between a lexical head and its phrase, which allows for syntactic flexibility inside the NP. 
Similarly, the information shared between a trace and its antecedent contains the \attrib{index} value. Consequently, participation in unbounded dependency constructions is equally accounted for.
Finally, since a pronoun has the same \attrib{index} value as its antecedent, pronominalization is also possible. 
%


I sketch the analysis of a non"=decomposable, fixed idioms, \emph{kick the bucket}, in \refer{ke-kick}. 
In this case, there is only a lexical entry of the syntactic head of the idiom, the verb \emph{kick}. 
It selects the full phonology of its complement. This blocks any syntactic processes inside this NP. It also follows that the complement cannot be realized as a trace, which blocks extraction. The special \attrib{theta-role} value \type{nil} will be used to ensure that the non"=applicability of lexical rules 


\ea % Analysis of \emph{spill the beans} in the spirit of \cite{KE94a} 
\label{ke-kick}
\ms{phon & \phonliste{kick}\\
synsem & \ms{cat & \ms{subcat & \liste{NP, 
                                 NP\ms{phon & \phonliste{the, bucket}\\
                                       theta-role & nil
                        }}}\\
             cont & \ms{rel & die}}
}
\z 


With this analysis, \cite{KE94a} capture the both, the idiosyncratic aspects and the regularity, of idioms. 
They show how it generalizes to a wide range of idiom types. 
There are, however a number of problems. I will just mention few of them here.

There are two problems for the analysis of non"=decomposable idioms. 
First, the approach is too restrictive with respect to the syntactic flexibility of \emph{kick the bucket}, as it excludes cases such as \emph{kick the social/figurative bucket}, which have are discussed in  \cite{Ernst:81}. 
Second, it is built on equating the class of non"=decomposable idioms with that of semi-fixed idioms. As shown above, this cannot be maintained. 

There are also some problems with the \textsc{lexeme}-value selection. The index-identity between a pronoun and its antecedent would require that the subject of the relative clause\is{relative clause} in \refer{strings-relcl} has the same \attrib{index} value as the head noun \emph{strings}. However, the account of the lexical fixedness of idioms is built on the assumption that no verb except for the idiomatic \emph{pull} selects for an argument with \attrib{lexeme} value \type{strings\_i}.%
\footnote{\cite{Pulman:93} discusses the analogous problem for the denotational theory of \cite{GKPS85a}.}

\ea \label{strings-relcl}
Parky pulled the strings that got me the job.
\citep[137]{McCawley:81}
\z 

The analytic ingredients of \cite{KE94a} constitute the basis of later HPSG analyses. In particular, a mechanism for lexeme-specific selection has been widely assumed in most approaches. The attribute \attrib{theta-role} can be seen as a simple for of an \emph{inside-out} mechanism\is{inside-out constraint}, i.e., as a mechanism of encoding information about the larger structure within which a sign appears. 

%\cite{SS2003a}: inside out, 
%\cite{Sag2012a}

\subsection{Phrasal approach}
\label{Sec-Phrasal}

\marginpar{Crossref to constructional HPSG chapter}
\is{constructional HPSG|(}
With the advent of constructional analyses within HPSG, starting with \cite{Sag97a}, it is natural to expect phrasal accounts of idioms to emerge as well, as idiomaticity is a central empirical domain for Construction Grammar\is{Construction Grammar}, see Chapter~\crossrefchaptert{chap-cxg}. 
\marginpar{Crossref to CxG Chapter}
In this version of HPSG, there is an elaborate type hierarchy below \type{phrase}. 
\cite{Sag97a} also introduces \emph{defaults}\is{default} into HPSG, which play an important role in the treatment of idioms in constructional HPSG.
The clearest phrasal approach to idioms can be found in \cite{Riehemann2001a}, which incorporates insights from earlier publications such as \cite{Riehemann97a} and \cite{RB99a}.
%
\marginpar{crossref to semantics chapter}
The overall framework of \cite{Riehemann2001a} is constructional HPSG with \emph{Minimal Recursion Semantics}\is{Minimal Recursion Semantics} (MRS) \citep{CFMRS95a-u,CFPS2005a}, see also Chapter~\crossrefchaptert{chap-semantics}.

For Riehemann, idioms are phrasal units. 
Consequently, she assumes a subtype of \type{phrase} for each idiom, such as \type{spill-beands-idiomatic-phrase} or \type{kick-bucket-idiomatic-phrase}.
The proposal in \cite{Riehemann2001a} is at the same time phrasal and obeys the SLH. To achieve this, \cite{Riehemann2001a} assumes and attribute \attrib{words}, whose value contains all words dominated by a phrase. This makes it possible to say that a phrase of type \type{spill-beans-idiomatic-phrase} dominates the words \emph{spill} and \emph{beans}. This is shown in the relevant type constraint for the idiom \emph{spill the beans} in \refer{sr-spillbeans}.%
\footnote{The percolation mechanism for the feature \attrib{words} is rather complex. In fact, in \cite[Section 5.2.1]{Riehemann2001a} the idiom-specific words appear within an \attrib{c-words} value, the other words dominated by the idiomatic phrase in the value of an attribute \attrib{other-words}, which together form the value of \attrib{words}. While all the values of these features are subject to local percolation principles, the fact that entire words are percolated undermines the locality intuition behind the SLH.}


\marginpar{change \attrib{words} value to set!}
\vbox{
\ea Constraint on the type \type{spill-beans-idiomatic-phrase} from \citet[185]{Riehemann2001a}:\label{sr-spillbeans}\\
\ms[spill-beans-ip]{
words & \liste{
\ms[i\_spill]{\ldots liszt & \liste{
\ms[i\_spill\_rel]{undergoer & \ibox{1}}}} 
\srdefault 
\ms{\ldots liszt & \liste{\type{\_spill\_rel}}},
\\
\ms[i\_bean]{\ldots liszt & \liste{
\ms[i\_bean\_rel]{inst& \ibox{1}}}} 
\srdefault 
\ms{\ldots liszt & \liste{\type{\_bean\_rel}}}, \ldots 
}
}
\z 
}

The \attrib{words} value of the idiomatic phrase contains at least two elements, the idiomatic words of type \type{i\_spill} and \type{i\_beans}. 
The special symbol ``\srdefault'' used in this constraint expresses a default\is{default}. It says that the idiomatic version of the word \emph{spill} is just like its non"=idiomatic homonym, except for the parts specified in the left-hand side of the default. 
In this case, the type of the words and the type of the semantic predicate contributed by the words are changed. 
\cite{Riehemann2001a} only has to introduce the types for the idiomatic words in the type hierarchy but need not specify type constraints on the individual idiomatic words, as these are constrained by the default statement within the constraints on the idioms containing them.


As in the account of \cite{KE94a}, the syntactic flexibilty of the idiom follows from its free syntactic combination and the fact that all parts of the idiom are assigned an independent semantic contribution. The lexical fixedness is a consequence of the requirement that particular words are dominated by the phrase, namely the idiomatic versions of \emph{spill} and \emph{beans}.

The appeal of the account is particularly clear in its application to non"=de\-com\-posable, semi-flexible idioms such as \emph{kick the bucket} \citep[\page 212]{Riehemann2001a}. 
For such expressions, the constituting idiomatic words are assumed to have an empty semantics and the semantics of the idiom is contributed as a constructional semantic contribution by only by the idiomatic phrase. 
Since the \attrib{words} list contains entire words, it is also possible to require that the idiomatic word \emph{kick} be in active voice and/or that it takes a complement compatible with the description of the idiomatic word \emph{bucket}.
This analysis captures the syntactically regular internal structure of this type of idioms, and is compatible with the occurrence of modifiers such as \emph{proverbial}. At the same time, it prevents passivization and excludes extraction of the complement.

Riehemann's approach clearly captures the intuition of idioms as phrasal units much better than any other approach in HPSG. 
However, it faces a number of problems.
Frist, the integration of the approach with constructional HPSG is done in such a way that the phrasal types for idioms are cross-classified in complex type hierarchies with the various syntactic constructions in which the idiom can appear. 
This allows Riehemann to account for idiosyncratic differences in the syntactic flexibility of idioms, but the question is whether such an explicit encoding misses generalizations that should follow from indepedent properties of the components of an idiom and/or of the syntactic construction -- in line with the quote from \cite{NSW94a} in \refer{NSW-quote}.


Second, the mechanism of percolating dominated words to each phrase is not compatible with the intuitions of most HPSG researchers. 
Since no empirical motivation for such a mechanism outside idioms is provided in \cite{Riehemann2001a}, this idea has not been pursued in other papers. 

Third, the question of how to block the occurrence of idiomatic words outside idioms is not solved in \cite{Riehemann2001a}, i.e., while the idiom requires the presence of particular idiomatic words, the occurrence of these words is not restricted.%
\footnote{Since the problem of free occurrences of idiomatic words is not an issue for parsing, versions of Riehemann's approach have been integrated in practical parsing systems \citep{Villavicencio:Copestake:02}, see Chapter~\crossrefchaptert{chap-cl}. 
Similarly, the approach to idioms sketched in \cite{Flickinger:15Slides2} 
is part of a system for parsing and machine translation. Idioms in the source language are identified by bits of semantic representation -- analogous to the elements in the \attrib{words} set. This approach, however, does not constitute a theoretical modelling of idioms within one language.}
\marginpar{Crossref to compling chapter}


%\bigskip%
Before closing this subsection, I would like to point out that 
\cite{Riehemann2001a} and \cite{RB99a} are the only HPSG papers on idioms that address the question of statistical idiomaticity\is{statistical idiomaticity}, based on the variationist study in \cite{Bender2000a}. 
In particular, \citet[\page 297--301]{Riehemann2001a} proposes phrasal constructions for collocations even if these do not show any lexical, syntactic, semantic, or pragmatic idiosyncrasy but just a statistical co-occurrence preference. 
She extends this into a larger plea for an \emph{experience-based HPSG}\is{experience-based HPSG}. 
%
\cite{Bender2000a} discusses the same idea under the notions of \emph{minimal} versus \emph{maximal} grammars, i.e., grammars that are as free of redundancy as possible to capture the grammatical sentences of a language with their correct meaning versus grammars that might be open to an connection with usage-based approaches\is{usage-based grammar} to language modelling.
\citet[\page 292]{Bender2000a} sketches a version of HPSG with frequencies/probabilities attached to lexical and phrasal types.%
\footnote{A so-far unexplored solution to the problem free occurrence of idiomatic words within an experience-based version of HPSG could be to assign the type \type{idiomatic-word} an extremely low probability of occurring. This might have the effect that such a word can only be used if it is explicitly required in a construction. However, note that neither defaults\is{default} nor probabilities are well-definied part of the formal foundations of theoretical work on HPSG, see Chapter~\crossrefchaptert{chap-formalbackground}.}
\marginpar{Crossref to formal background chapter}

\is{constructional HPSG|)}

\subsection{Mixed lexical and phrasal approaches}
\label{Sec-Mixed}

While \cite{Riehemann2001a} proposes a parallel treatment of decomposable and non"=decomposable idioms -- and of flexible and semi-flexible idioms, the devision between fixed and non"=fixed expressions is at the core of another approach, the \emph{two-dimensional theory of idioms}\is{two-dimensional theory of idioms}. This approach was first outlined in \cite{Sailer2000a} and referred to under this label in \cite{Richter:Sailer:09,Richter:Sailer:14}. It is intended to combine constructional and collocational approaches to grammar.

The basic intuition behind this approach is that signs have internal and external properties. 
All properties that are part of the feature structure of a sign are called \emph{internal}. 
Properties that relate to larger feature structures containing this sign are called its \emph{external} properties. 
The approach assumes that there is a notion of \emph{regularity} and that anything diverging from it is \emph{idiosyncratic} -- or idiomatic, in the terminology of this chapter. 

This approach is another attempt to reify the GPSG\is{Generalized Phrase Structure Grammar} analysis within HPSG.
\cite{Sailer2000a} follows the distinction of \cite{NSW94a} into non"=decomposable and non"=flexible idioms on the one hand and decomposable and flexible idioms on the other. The first group is considered internally irregular and receives a constructional analysis in terms of a \emph{phrasal lexical entry}\is{phrasal lexical entry}. The second group is considered to consists of independent, smaller lexical units that show an external irregularity in being constrained to co-occur within a larger structure. 
Idioms of the second group receive a collocational analysis. The two types of irregularity are connected by the  \emph{Predictability Hypothesis}, given in \refer{PredHypo}.

\ea Predictability Hypothesis \citep[\page 366]{Sailer2000a}:\label{PredHypo}\\
For every sign whose internal properties are fully predictable, the distributional
behavior of this sign is fully predictable as well.
\z 


In the most recent version of this approach, \cite{Richter:Sailer:09}, there is a feature \attrib{coll} defined on all signs. 
The value of this feature specifies the type of internal irregularity. 
The authors assume a cross-classification of regularity and irregularity with respect to syntax, semantics, and phonology -- ignoring pragmatic and statistical (ir)regularity in their paper. 
Every basic lexical entry is defined as completely irregular as its properties are not predictable. 
Fully regular phrases such as \emph{read a book} have a trivial value of \attrib{coll}. 
A syntactically internally regular but fixed idiom such as \emph{kick the bucket} is classified as having only semantic irregularity, whereas a syntactically irregular expression such as \emph{trip the light fantastic} is of an irregularity type that is a subsort of syntactic and semantic irregularity, but not of phonological irregularity.
Following the terminology of \cite{FKoC88a}, this type is called \type{extra-grammatical-idiom}.
%
The phrasal lexical entry for \emph{trip the light fantastic} is sketched in  \refer{rs-trip}.

\ea Phrasal lexical entry for the idioms \emph{trip the light fantastic}:\label{rs-trip}\\
\ms[phrase]{
phon & \ibox{1} $\oplus$ \phonliste{the, light, fantastic}\\
syns & \ms{loc  \ms{cat  \ms{head & \ibox{4}\\
                        listeme & trip-the-light-fantastic\\
                        val & \ibox{5} \ms{subj & \liste{\ibox{2} \ms{\ldots index & \ibox{3}}}}}\\
                     cont \ms[trip-light-fant]{danser & \ibox{3}}
                     }}\\
dtrs & \ms[headed-structure]{head-dtr &
        \ms{phon & \ibox{1}\\
            \ldots cat & \ms{head & \ibox{4} verb\\
                             listeme & trip\\
                            val & \ibox{5} \ms{subj & \liste{\ibox{2}}\\
                            comps & \eliste}
                            }}}\\
coll & extra-grammatical-idiom
%\ms[extra-grammatical-idiom]{req & \eliste}
}
\z 

In \refer{rs-trip}, the constituent structure of the phrase is not specified, but the phonology is fixed, with the exception of the head daughter's phonological contribution. This accounts for the syntactic irregularity of the idiom. The semantics of the idiom is not related to the semantic contributions of its components, which accounts for the semantic idiomaticity.

\cite{Soehn2006a} applies this theory to German\il{German}. He solves the problem of the relatively large degree of flexibility of non"=decomposable idioms in German
by using underspecified descriptions of the constituent structure dominated by the idiomatic phrase.

For decomposable idioms, the two-dimensional theory assumes a collocational component. This component is integrated into the value of an attribute \attrib{req}, which is only defined on \type{coll} objects of one of the irregularity types. 
This encodes the Predictability Hypothesis.
%
The most comprehensive version of this collocational theory is given in \cite{Soehn:09}, summarizing and extending ideas from \cite{Soehn2006a} and \cite{richter-soehn:2006}. 
Soehn assumes that collocational requirements can be of various types: 
a lexical item can be constrained to co-occur with particular \emph{licensers} (or collocates). These can be other lexemes, semantic operators, or phonological units. In addition, the domain within which this licensing has to be satisfied is specified in terms of syntactic barriers, i.e., syntactic nodes dominating the externally irregular item.

To give an example, the idiom \emph{spill the beans} would be analyzed as consisting of two  idiomatic words \emph{spill} and \emph{beans} with special \attrib{listeme} values \type{spill-i} and \type{beans-i}. The idiomatic verb \emph{spill} imposes a lexeme-selection on its complement. The idiomatic noun \emph{beans} has a non"=empty \attrib{req} value, which specifies that it must be selected by a word with \attrib{listeme} value \type{spill-i} within the smallest complete clause dominating it.

%\cite{Richter:Sailer:09,Richter:Sailer:14} look at idioms with 

%\bigskip
The two-dimensional approach suffers from a number of weaknesses. 
First, it presupposes a notion of regularity. This assumption is not shared by all linguists.
Second, the criteria whether an expression should be treated constructionally or collocationally are not always clear. Idioms with irregular syntactic structure need to be analyzed constructionally, but this is less clear for non"=decomposable idioms with regular syntactic structure such as \emph{kick the bucket}.
Here, the approach inherits the weakness of \cite{WSN84a-u} equating syntactic flexibility and semantic decomposability.

%\begin{itemize}
%\item \cite{Sailer2000a}, \cite{Soehn2006a}, \cite{Richter:Sailer:09}
%\end{itemize}

\subsection{Recent lexical approaches}
\label{Sec-RecentLexical}

\cite{KSF2015a} marks an important re-orientation in the analysis of idioms: the lexical analysis is extended to all syntactically regular idioms, i.e., to both decomposable (\emph{spill the beans}) and non"=decomposable idioms (\emph{kick the bucket}).% 
\footnote{This idea has been previously expressed within a more generative grammar perspective in \cite{Everaert:10}. 
}

Though \cite{KSF2015a} use Sign-based Construction Grammar\is{Sign-based Construction Grammar}, I consider it legitimate and important to include their analysis here.
%
\cite{KSF2015a} achieve a lexical analysis of non"=decomposable idioms by two means: (i) an extension of the HPSG selection mechanism, (ii) the assumption of semantically empty idiomatic words. 

As in previous accounts, the relation among idiom parts is established through lexeme-specific selection, using a feature \attrib{lid} (for: \emph{lexical identifier}). 
The authors assume that there is a difference between idiomatic and non"=idiomatic \attrib{lid} values. 
Only heads that are part of idioms themselves can select for idiomatic words. 
%Quote: Ordinary, non"=idiom predicators are lexically specified as requiring all members of their VAL list to be nonidiomatic.

For the idiom \emph{kick the bucket}, \cite{KSF2015a} assume that all meaning is carried by the lexical head, an idiomatic version of \emph{kick}, whereas the other two words, \emph{the} and \emph{bucket} are meaningless. 
This meaninglessness allows Kay et al.\@ to block the idiom from occurring in constructions which reuqire meaningful constituents, such as questions, \emph{it}-clefts, middle and others. 
To exclude passivization, the authors assume that English passive cannot apply to verbs selecting a semantically empty direct object.

The approach in \cite{KSF2015a} is a recent attempt to maintain the SLH as much as possible. 
Since the SLH has been a major conceptual motivation for SBCG\is{Sign-based Construction Grammar}, \citeauthor{KSF2015a}'s paper is an important contribution to show the empirical robustness of this assumption.

%\medskip%
\cite{Bargmann:Sailer:18} propose a similar lexical approach to non"=de\-com\-pos\-able idioms. 
They take as their starting point the syntactic flexibility of semantically non"=decomposable idioms in Engish and, in particular, in German.
There are two main differences between \citeauthor{KSF2015a}'s paper and \citeauthor{Bargmann:Sailer:18}'s: (i), \citeauthor{Bargmann:Sailer:18} assume a collocational rather than a purely selectional mechanism to capture lexeme-restrictions of idioms, and (ii), they propose a redundant semantics rather than an empty semantics for idiom parts in non"=decomposable idioms. In other words, \cite{Bargmann:Sailer:18} propose that both \emph{kick} and \emph{bucket} contribute the semantics of the idiom \emph{kick the bucket}. 
\citeauthor{Bargmann:Sailer:18} argue that the semantic contiributions of parts of non"=decomposable, syntactically regular idioms are the same across languages, whereas the differences in syntactic flexibility are related to the different syntactic, semantic, and pragmatic constraints imposed on various constructions. 
To give just one example, whereas passive subjects in German are almost non restricted, there are strong discourse-structural constraints on passive subjects in English.

Both \cite{KSF2015a} and \cite{Bargmann:Sailer:18} attempt to derive the (partial) syntactic inflexibility of non"=decomposable idioms from independent properties of the relevant constructions. 
As such, they subscribe to the programmatic statement of \cite{NSW94a} from \refer{NSW-quote} above.  
In this respect, the extension of the lexical approach from decomposable idioms to all syntactically regular expressions has been a clear step forward. 

%\bigskip%
\cite{Findlay:17} provides a recent discussion and criticism of lexical approaches to idioms in general, which applies in particular to non"=decomposable expressions. 
His reservations comprise the following points. 
First, there is a massive proliferation of lexical entries for otherwise homophonous words. Is is unclear, for example, if a separate definite article is needed for each idiom which contains one, i.e., it might turn out that we need different lexical entries for the word \emph{the} in \emph{kick the bucket}, \emph{shoot the breeze}, and \emph{shit hits the fan}. 
Second, the lexical analysis does not represent idioms as units, which might make it difficult to connect their theoretical treatment with processing evidence. Findlay refers to psycholinguistic studies, such as \cite{Swinney:Cutler:79}, that point to a faster processing of idioms than of free combinations. 


\kommentar{
\begin{itemize}
\item \cite{KSF2015a}, \cite{Sag2012a}
\item \cite{Bargmann:Sailer:18}
\end{itemize}
}

\kommentar{
\section{Recent developments in other frameworks}
\label{Sec-RecentOtherFrameworks}

The interest in idioms varies 

\begin{itemize}
\item TAG: \cite{Lichte:Kallmeyer:16}
\item Minimalism: \cite{vCraenenbroeck:al:16draft}, \cite{Everaert:10}
\item LFG: \cite{Findlay:17}
\end{itemize}
}

\section{Where to go from here?}
\label{Sec-WhereToGo}

The final section of this article contains short overviews over research that has been done in areas of phraseology that are outside the main thread of this chapter. I will also identify desiderata. 


%In this final section, I would like to point to directions for future research. 
\subsection{Neglected phenomena}
\label{Sec-Neglected}

Not all types of idioms  and not all types of idiomaticity mentioned in Section \ref{Sec-EmpiricalDomain} have received an adequate treatment in the (HPSG) literature.
I will briefly look at three empirical areas that deserve more attention: neglected types of idiom variation, phraseological patterns, and the literal and non"=literal meaning components of idioms.

%\bigskip%Variation
Most studies on idiom variation have looked at verb- and sentence-related syntactic constructions, such as passive and topicalization. 
However, not much attention has been payed to lexical variation in idioms. This is illustrated by the following examples from \citet[\page 184, 191]{Richards:01}. 

\eal  \label{creeps}
\ex The Count gives everyone the creeps.
\ex You get the creeps (just looking at him).
\ex I have the creeps.
\zl 

In \refer{creeps}, the alternation of the verb seems to be very systematic -- and has been used by \cite{Richards:01} to motivate a lexical decomposition of these verbs.
A similar argument has been made in \cite{Mateu:Espinal:07} for Catalan. 
We are lacking systematic, larger empirical studies of this type of substitution, and it would be important to see how it can be modeled in HPSG. 
One option would be to capture the \emph{give}--\emph{get}--\emph{have} alternation(s) with lexical rules. Such lexical rules would be different from the standard cases, however, as they would change the lexeme itself rather than just alternating its morpho-syntactic properties or its semantic contribution.

In the case mentioned in footnote \ref{fn-semmeln}, the alternation consists in substituting a word with a (near) synonym and keeping the meaning of the idiom intact. Again, HPSG seems to have all the required tools to model this phemonenon -- for example, by means of hierarchies of lexical-id values. 
However, the extent of this phenomenon across the set of idioms is not known empirically. 

%\medskip%
In the domain of syntactic variation, the nominal domain has not received the attention it might deserve yet. 
There is a well-known variation with respect to the marking of possession within idioms. 
This has been documented for English\il{English} in \cite{Ho:15}, for Modern Hebrew\il{Hebrew} in \cite{Almog:12}, for Modern Greek and German\il{German} in \cite{Markantonatou:Sailer:16}. 
In German, we find a relatively free alternation between a plain definite and a possessive, see \refer{ex-verstand}. This is, however, not possible with all idioms,  \refer{ex-frieden}.

\eal \label{ex-verstand-herz}
\ex 
\gll Alex hat den / seinen Verstand verloren.\\
Alex has the {} his mind lost\\
\glt `Alex lost his mind.'\label{ex-verstand}
\ex 
\gll Alex hat *den / ihren Frieden mit der Situation gemacht.\\
     Alex has the {} her peace with the situation made\\
\glt `Alex made her peace with the situation.'\label{ex-frieden}
\zl 


We can also find a free dative in some cases, expressing the possessor. 
In \refer{ex-herz}, a dative possessor may co-occur with a plain definite or a coreferential possessive determiner, in \refer{ex-augen} only the definite article but not the possessive determiner is possible.  


\eal \label{ex-herz-augen}
\ex 
\gll Alex hat mir das / mein Herz gebrochen.\\
Alex has me.\textsc{dat} the {} my heart broken\\
\glt `Alex broke my heart.'\label{ex-herz}
\ex 
\gll Alex sollte mir lieber aus den / *meinen Augen gehen.\\
Alex should me.\textsc{dat} rather {out of} the {} my eyes go\\
\glt `Alex should rather disappear from my sight.'\label{ex-augen}
\zl 

While they do not offer a formal encoding, \cite{Markantonatou:Sailer:16} observe that a particular encoding of possession in idioms is only possible if it would also be possible in a free combination. However, an idiom my be idiosyncratically restricted to a subset of the realizations that would be possible in a corresponding free combination. A formalization in HPSG might consist of a treatment of possessively used definite determiners, combined with an analysis of free datives as an extension of a verb's argument structure.

 

%\bigskip%Patterns
Related to the question of lexical variation are \is{phraseological patterns}\emph{phraseological patterns}, i.e., very schematic idioms 
in which the lexical material is largely free. Some examples 
of phraseological patterns are
  the \emph{Incredulity Response Construction} as in \emph{What, me worry?} \citep{Akmajian:84,Lambrecht:90}, 
or the \emph{What's X doing Y?}-construction \citep{KF99a}.
Such patterns are of theoretical importance as they typically involve a non"=canonical syntactic pattern. 
The different locality and non"=locality hypotheses introduced above make different predictions. 
\cite{FKoC88a} have presented such constructions as a motivation for the non"=locality of constructions, i.e., as support of a SNH. However, \cite{KF99a} show that a lexical analysis might be possible for some cases at least. 
\citeauthor{KF99a} provide a detailed lexical analysis of the \emph{What's X doing Y?}-construction. 

\cite{Borsley:04} looks at another phraseological pattern, the \emph{the X-er the Y-er}-construction, or \emph{comparative correlative construction}.
Borsley analyzes this construction by means of two special (local) phrase structure types: one for the comparative \emph{the}-clauses, and one for the overall construction. He shows that (i), the idiosyncrasy of the construction concerns two levels of embedding and is, therefore, non"=local, however, (ii),
a local analysis is still possible. This approach raises the question as to whether the WNH is empirically vacuous since we can always encode a non"=local construction in terms of a series of idiosyncratic local construction. 
Clearly, work on more phraseological patterns is needed to assess the various analytical options and their consequences for the architecture of grammar.


%\bigskip%Literal
A major charge for the conceptual and semantic analysis of idioms is the interaction between the literal and the idiomatic meaning. 
I presented the basic empirical facts in Section \ref{Sec-EmpiricalDomain}. 
All HPSG approaches to idioms so far basically ignore the literal meaning.
This position might be justified, as  an HPSG grammar should just model the structure and meaning of an utterance and need not worry about the meta-linguistic relations among different lexical items or among different readings of the same (or a homophonous) expression.
This is an important conceptual point that immediately provides possibilities to connect HPSG research to other disciplines and/or frameworks, such as cognitive linguistics, such as \cite{Dobrovolskij:Piirainen:05}, and psycholinguistics.



\kommentar{
\begin{itemize}
\item Variation in nominal parts/possessive idioms \citep{Bond:al:15}
\item Literal and non"=literal meaning components (Bargmann, Gehrke \& Richter on \emph{conjunction modification}, \citet{Hoeksema:Sailer:12}, \ldots).
\item Phraseological patterns
\end{itemize}
}

\subsection{Challenges from other languages}
\label{Sec-OtherLanguages}

The majority of work on idioms in HPSG has been done on English and German. 
%This led to a limitation of the possible phenomena that can be studied on idioms. 
As discussed in Section \ref{Sec-RecentLexical}, the recent trend in HPSG idiom research necessitates a detailed study of individual syntactic structures. 
Consequently, the restriction on two closely related languages limits the possible phenomena that can be studied on idioms. 
It would be essential to expand the empirical coverage of idiom analyses in HPSG to as many different languages as possible. 
The larger degree of syntactic flexibility of French, German, and Dutch idioms \citep{Ruwet:91,NSW94a,Schenk:95} has led to important refinements of the analysis in \cite{NSW94a} and, ultimately, to the lexical analyses of all syntactically regular idioms. 

Similarly, the above-mentioned data on possessive alternations only become prominent when languages beyond English are taken into account. The languages mentioned above \refer{ex-herz-augen} all show the type of external possessor classified as a European areal phenomenon in \cite{Haspelmath:99}. 
It would be important to look at idioms in languages with other types of external possessors.


In a recent paper, \cite{Sheinfux:al:18} provide data from Modern Hebrew\il{Hebrew} that show that opacity and figurativity of an idiom are decisive for its syntactic flexibility rather than decomposability.
This result stresses the importance of the literal reading for an adequate account of the syntactic behavior of idioms. 
%
It shows that the inclusion of other languages can cause a shift of focus to other types of idioms or other types of idiomaticity. 

To add just one more example, HPSG(-related) work on Persian such as \cite{MuellerPersian-unlinked} and \cite{Samvelian:Faghiri:16} establishes a clear connection between complex predicates and idioms. 
Their insights might also lead to a reconsideration of the similarities between light verbs and idioms, as already set out in \cite{KE94a}.



\kommentar{
The majority of work on idioms in HPSG has been done on English and German, with a new line of research on Modern Hebrew\il{Hebrew} \citep{Sheinfux:al:15,HMW2016a-u}. 
Work on complex predicates in Persian\il{Persian} \citep{MuellerPersian-unlinked,Samvelian:Faghiri:16}, establishes a clear connection between the idioms and complex predicates.
}


As far as I can see, the following empirical phenomena have not been addressed in HPSG-approaches to idioms as they do not occur in the main object languages for which we have idiom analyses, i.e.\@ English and German. They are, however, common in other languages: the occurrence of clitics in idioms (found in Romance and Greek); aspectual alternations in verbs (Slavic and Greek); argument alternations other than passive (such as anti-passive, causative, inchoative etc
(in part found in Hebrew and addressed in \cite{Sheinfux:al:18}); 
displacement of idiom parts into special syntactic positions (focus position in Hungarian). 

Finally, so far, idioms have usually been considered as either offering irregular structures or as being more restricted in their structures than free combinations. In some languages, however, we find archaic syntactic structures and function words in idioms that do not easily fit these two analytic options. To name just a few, \cite{Lodrup:09} argues that Norwegian\il{Norwegian} used to have an external possessor construction similar to that of other European languages, which is only conserved in some idioms. Similarly, Dutch\il{Dutch} has a number of archaic case inflections in multiword expressions \citep[\page 129]{Kuiper:18}, and there are archaic forms in Modern Greek\il{Greek} multiword expressions. It is far from clear what the best way would be to integrate such cases into an HPSG grammar. 


\kommentar{
In other formal frameworks, other languages have received more systematic attention. 

To mention just a few, there are many analyses within versions of minimalism.  
\cite{Mateu:Espinal:07} and other work of these authors on Catalan. \cite{vCraenenbroeck:al:16draft} discuss idioms from Dutch\il{Dutch} and Dutch dialects in a minimalist framework. 

For many other languages, there are computational treatments of various types of phraseological units, which have, however, not necessarily had a
}

\kommentar{
\begin{itemize}
\item Work by Espinal on Spanish and Catalan in comparison to English
\item Hebrew \citep{HMW2016a-u}
\item German (if not already discussed enough in earlier sections)
\end{itemize}
}

\kommentar{
\subsection{Methodological considerations}
\label{Sec-Methods}

Brief overview of some corpuslinguistic and psycholinguistic results and the question what they could contribute to the question of modeling idioms in HPSG.
}

\section{Conclusion}
\label{Sec-Summary}

Idioms are among the topics in linguistics for which HPSG-related publications have had a clear impact on the field and have been widely quoted across frameworks.
This handbook article aimed at providing an overview over the development of idiom analyses in HPSG. There seems to be a clear development towards ever more lexical analyses, starting from the holistic approach for all idioms in Chomsky's work, to a lexical account for all syntactically regular expressions. 
However, it is very likely that phrasal analyses are going to experience a comeback in the near future. 

The sign-based character of HPSG seems to be particularly suited for a theory of idioms as it allows to take into consideration syntactic, semantic, and pragmatic aspects and to use them to constrain the occurrence of idioms appropriately.



\kommentar{
\section{Where we came from} 
Phasellus maximus erat ligula, accumsan rutrum augue facilisis in. Proin sit amet pharetra nunc, sed maximus erat. Duis egestas mi eget purus venenatis vulputate vel quis nunc. Nullam volutpat facilisis tortor, vitae semper ligula dapibus sit amet. Suspendisse fringilla, quam sed laoreet maximus, ex ex placerat ipsum, porta ultrices mi risus et lectus. Maecenas vitae mauris condimentum justo fringilla sollicitudin. Fusce nec interdum ante. Curabitur tempus dui et orci convallis molestie \citep{Chomsky:57}.

\begin{table}
\caption{Frequencies of word classes}
\label{tab:1:frequencies}
 \begin{tabular}{lllll} 
  \lsptoprule
            & nouns & verbs & adjectives & adverbs\\ 
  \midrule
  absolute  &   12 &    34  &    23     & 13\\
  relative  &   3.1 &   8.9 &    5.7    & 3.2\\
  \lspbottomrule
 \end{tabular}
\end{table}

Sed nisi urna, dignissim sit amet posuere ut, luctus ac lectus. Fusce vel ornare nibh. Nullam non sapien in tortor hendrerit suscipit. Etiam sollicitudin nibh ligula. Praesent dictum gravida est eget maximus. Integer in felis id diam sodales accumsan at at turpis. Maecenas dignissim purus non libero scelerisque porttitor. Integer porttitor mauris ac nisi iaculis molestie. Sed nec imperdiet orci. Suspendisse sed fringilla elit, non varius elit. Sed varius nisi magna, at efficitur orci consectetur a. Cras consequat mi dui, et cursus lacus vehicula vitae. Pellentesque sit amet justo sed lectus luctus vehicula. Suspendisse placerat augue eget felis sagittis placerat. 

\ea
\gll cogito                           ergo      sum\\  
     think.\textsc{1sg}.\textsc{pres} therefore \textsc{cop}.\textsc{1sg}.\textsc{pres}\\ 
\glt `I think therefore I am.'
\z

Sed cursus eros condimentum mi consectetur, ac consectetur sapien pulvinar. Sed consequat, magna eu scelerisque laoreet, ante erat tristique justo, nec cursus eros diam eu nisl. Vestibulum non arcu tellus. Nunc dignissim tristique massa ut gravida. Nullam auctor orci gravida tellus egestas, vitae pharetra nisl porttitor. Pellentesque turpis nulla, venenatis id porttitor non, volutpat ut leo. Etiam hendrerit scelerisque luctus. Nam sed egestas est. Suspendisse potenti. Nunc vestibulum nec odio non laoreet. Proin lacinia nulla lectus, eu vehicula erat vehicula sed. 

}

\section*{Appendix: List of used idioms}

Some idioms do not show semantic idiomaticity at all, such as collocations\is{collocation} (\bsp{brush one's teeth}) or support verb constructions (\bsp{take a shower}). 
Many body-part expressions such as \bspT{shake hands}{greet} or \bspT{shake one's head}{decline/""negate} constitute a more complex case they describe a conventionalized activity and denote the social meaning of this activity \citep{Burger:76}.

\subsection*{English}
\begin{tabular}{@{}lll}
idiom & paraphrase & comment\\\hline
break the ice & \myappcolumn{4.5cm}{relieve tension in a strained situation} & non"=decomposable\\
brush one's teeth & \myappcolumn{4.5cm}{clean one's teeth with a tooth brush}
& collocation, no idiomaticity\\
give so the creeps & make so feel uncomfortable & systematic lexical variation\\
Good morning! &(morning greeting) & formulaic expression\\
immaculate performance & perfect performance & statistical idiomaticity\\
in a trice & in a moment & bound word: \emph{trice}\\
kick the bucket & die & non"=decomposable\\
make headway & make progress & bound word: \emph{headway}\\
pull strings & \myappcolumn{4.5cm}{exert influence/""use one's connections} & flexible\\
saw logs & snore & \appc{transparent, non"=decomposable, semi-flexible}\\
shake hands & greet & body-part expression\\
shake one's head & decline/""negate & \appc{body-part expression, possessive idiom}\\
take a shower & clean oneself using a shower & collocation, light verb consstruction\\
%
shake hands & greet & body-part expression\\
% 
shake one's head & decline/""negate & body-part expression\\
%
shit hit the fan & there is trouble & \appc{subject as idiom component, transparent/figurative, non"=decomposable}\\
shoot the breeze & chat & non"=decomposable\\
spill the beans & reveal a secret & flexible\\
take a shower & to shower & collocation\\
\appc{take the bull by the horns} & \appc{approach a problem directly}
& figurative expression\\
trip the light fantactic & dance & syntactically irregular\\


\end{tabular}

\subsection*{German}
\begin{tabular}{@{}ll@{}l@{}l}
idiom & gloss & translation & comment\\\hline
%
 \appc{den/seinen Verstand verlieren}
 & \appc{the/""one's mind lose}
 & \appc{lose one's mind}
 & \appc{alternation of possessor marking}
 \\
 %
 \appc{jm das Herz brechen} & so the heart break & break s.o.'s heart
 & \appc{dative possessor and possessor alternation}\\
 %
 \appc{jm aus den Augen gehen} & so out of the eyes go
 & \appc{disappear from s.o.'s sight} &
 \appc{dative possessor, restricted possessor alternation}\\
 %
 \appc{seinen Frieden machen mit}
 & \appc{one's peace make with}
 & \appc{make one's peace with}
 & \appc{no possessor alternation possible}\\
 %
 \appc{wie warme Semmeln/Brötchen/Schrippen weggehen}
 & \appc{like warm rolls vanish} & sell like hotcakes & 
 \appc{parts can be exchanged by synonyms}
 \\
\end{tabular}


\section*{Abbreviations}

\begin{tabular}{ll}
GPSG & Generalized Phrase Structure Grammar \citep{GKPS85a}\\
MRS & Minimal Recursion Semantics \citep{CFPS2005a}\\
MWE & multiword expression\\
SLH & Strong Locality Hypothesis, see page \pageref{slh}\\
SNH & Strong Non"=locality Hypothesis, see page \pageref{snh}\\
WLH & Weak Locality Hypothesis, see page \pageref{wlh}\\
WNH & Weak Non"=locality Hypothesis, see page \pageref{wnh}\\
\end{tabular}

\section*{Acknowledgements}

I have perceived Ivan A.\@ Sag\ia{Sag, Ivan A.} and his work wiht various colleagues as a major inspiration for a lot of my own work on idioms and multiword expressions. 
This is clearly reflected in the structure of this paper, too. 
I apologize for this bias, but I think it is legitimate within an HPSG handbook.
%
I am grateful to Stefan Müller for comments on the outline of this chapter. 


{\sloppy
\printbibliography[heading=subbibliography,notkeyword=this] }
\end{document}
