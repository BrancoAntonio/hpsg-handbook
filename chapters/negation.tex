\documentclass[output=paper]{langsci/langscibook} 
\author{Jong-Bok Kim\affiliation{Kyung Hee University, Seoul}}
\title{Negation}

% \chapterDOI{} %will be filled in at production

\epigram{Change epigram in chapters/03.tex or remove it there }
\abstract{Change the  abstract in chapters/03.tex \lipsum[3]}
\maketitle

\begin{document}
\label{chap-negation}

%\if 0

{\avmoptions{center}

\section{Modes of Expressing Negation}

Each language has its own way of expressing negation
with grammatical restrictions in its surface realizations.
This
chapter aims to provide an investigation of morpho-syntactic aspects
of negation in natural languages, encompassing both empirical and
theoretical issues  concerning negation as well as related phenomena
in question.

%In a typological study of sentential negation,  Dahl (1979) has
%identified three major ways of expressing negation in natural
%%languages as a morphological category on verbs, as an auxiliary
%verb, and as an adverb-like particle.

There are four main ways of expressing negation in the
languages: morphological negative,
auxiliary negative verb, adverbial negative, and clitic-like negative (see \citet{Dahl:79}, \citet{Payne:85}, and \citet{Dryer:05}).
Each of these types is illustrated in the following:

\eal
\ex\label{1a} Turkish:\\
\gll Ali  elmalar-i  ser-me-di-$\emptyset$. \\
Ali apples-\textsc{acc}  like-\textsc{neg}-\textsc{pst}-\textsc{3sg} \\
\glt `Ali didn't like apples.'

\ex\label{1b} Korean:\\
\gll sensayng-nim-i o-ci anh-usi-ess-ta \\
teacher-\textsc{hon}-\textsc{nom} come-\textsc{conn} \textsc{neg}-\textsc{hon}-\textsc{pst}-\textsc{decl} \\
\glt `The teacher didn't come.'

\ex \label{1c} French:\\
\gll Dominique (n')\'{e}crivait pas de lettre.\\
     Dominique wrote \textsc{neg} of letter \\
\glt `Dominiquedid not write a letter.'

\ex \label{1d} Italian:\\
\gll Gianni non legge articoli di sintassi. \\
Gianni \textsc{neg} reads articles of syntax \\
\glt `Gianni doesn't read syntax articles.'
\zl
%
As given in (\ref{1a}), languages like Turkish
have typical examples of morphological negatives where
negation is expressed by an inflectional category realized on the
verb by affixation. Meantime, languages like Korean
 employ a negative auxiliary verb as in (\ref{1b}).\footnote{Korean
 is peculiar in that it has two ways to
 express sentential negation: a negative auxiliary (a long form
 negation)  and a morphological negative (a short form negation)
 for sentential negation. See \citet{Kim:00,Kim:16} and references therein for detail.}
  The negative auxiliary
 verb here is marked with the basic verbal categories such as agreement, tense, aspect, and mood, while the main verb remains in an invariant, participle form. The third major way of expressing negation is to use an adverb-like
particle. This type of negation, forming an independent word, is prevalent in English and French, as given in (\ref{1c}). In these languages, negative markers behave like adverbs in their ordering with respect to the verb. The fourth
type is to introduce a clitic-like element in
expressing sentential negation. The negative marker in Italian in (\ref{1d}), preceding a finite verb like other types of clitics in the language,
belongs to this type.

This chapter provides a construction-based HPSG analysis of these four main types of negation we find in natural languages  and and further answers the
following three questions:

\begin{itemize}
\item What are the main ways of expressing sentential
negation or negating a sentence or clause?

\item What are the distributional possibilities of
negative markers in
relation to other main constituents of the sentence?

\item What do the answers to these two questions imply for
the theory of grammar?
\end{itemize}

\noindent
This chapter addresses these questions, based on empirical data,
theoretical issues, and analyses of negation.

\section{Derivational Views}

The derivational view has claimed that the positioning of all of the
four types of negatives is basically determined by the interaction of movement
operations, a rather large set of functional projections including NegP,
and their hierarchically fixed organization.

%The most influential, representative work is Pollock (1989, 1994), which
%notes

English and French display systematic differences with respect
to negation, adverb position, and subject-aux inversion, as illustrated
in the following:
% grammar have received considerable attention in the recent
% syntactic literature. Central to this inquiry has been the
% following set of contrasts:
%


\bigskip

\noindent {\bf Position of Negation}:

\eal\label{exe:1}
\ex[*] {Kim likes not Lee.
}
\ex[] {Kim does not like Lee.
}
\zl

\eal
\ex{
\gll Robin  n'aime    pas       Stacey. \\
     Robin  (n')likes \textsc{neg} Stacey \\
\glt `Robin does not like Stacey.'
}
\ex[*]{Robin ne pas aime Stacey.}
\zl

\noindent {\bf Position of Adverbs}:

\eal
\ex[*]{Kim kisses often Lee.}
\ex[]{Kim often kisses Lee.}
\zl

\eal
\ex[*]{Robin embrasse souvent Stacey.}
\ex[]{Robin souvent embrasse Stacey.}
\zl

\noindent {\bf Subject-Verb Inversion in Questions}:\index{inversion}

\eal
\ex[*]{Likes he Sandy?}
\ex[]{Does he like Sandy?}
\zl


\eal
\ex[*]{Likes Lou Sandy?}
\ex[]{Aime-t-il Sandy?}
\zl




The examples illustrate that in English, the negator \emph{not}
and adverb \emph{often} need to precede a main verb, while in French,
the corresponding negator \textit{pas} and adverb \textit{souvent} follow
a main verb. In addition, only French allows the main verb inversion.
Drawing on the earlier insights of \citet{Emonds:78},
\citet{Pollock:89,Pollock:94} and a number of subsequent researchers
have interpreted these contrasts as providing critical motivation for
the process of head movement and the existence of functional
categories such as MoodP, TP, AgrP, and NegP (see \citet{Belletti:90}, \citet{Zanuttini:91,Zanuttini:97,Zanuttini:01}, \citet{Chomsky:91,Chomsky:93,Chomsky:95}, \citet{Lasnik:95}, \citet{Haegeman:95,Haegeman:97}, \citet{Vikner:97}, \citet{Zeijlstra:15}, inter alia).
It has been widely
accepted that the variation between French and English illustrated
here can be explained only in terms of the respective properties
of verb movement and its interaction with a view of clause
structure organized around functional projections.

For example, in \citet{Pollock:89}'s proposal, all verbs in French
move to a higher structural position, whereas this is possible in
English only for the auxiliaries \emph{have} and \emph{be}, as
shown in Figure~\ref{fig:1}.

\begin{figure}
	\begin{subfigure}[b]{0.48\textwidth}
\centering
		\begin{forest}
			[TP
				[Tns,name=Tns]
				[NegP
					[pas]
				[Neg$'$
					[Neg [ne]]
					[AgrP
						[Agr [t,name=t]]
						[VP
							[V[all verbs,name=verbs]]
							[\dots]]]]]]
			\draw[->,dotted] (verbs) to[out=south west,in=south] (t);
			\draw[->,dotted] (t) to[out=west,in=south] (Tns);
		\end{forest}
		\caption{French}
	\end{subfigure}
	\hfill
	\begin{subfigure}[b]{0.48\textwidth}
\centering
		\begin{forest}
			[TP
				[Tns,name=Tns]
				[NegP
					[Neg[not]]
					[AgrP
						[Agr[t,name=t]]
						[VP
						[V[have/be,name=have]]
						[\dots]]]]]
			\draw[->,dotted] (have) [out=south west,in=south] to (t);
			\draw[->,dotted] (t) [out=west,in=south] to (Tns);
		\end{forest}
		\caption{English}
	\end{subfigure}
	\caption{Add caption}\label{fig:1}
\end{figure}

Why does V-movement happen when it does? This question
has been answered in diverse (and sometimes inconsistent) ways in
the literature (see \citet{Pollock:89,Pollock:94,Pollock:97a,Pollock:97b}, \citet{Vikner:94,Vikner:97}). In \citet{Pollock:89},
it is the strength of the Agr feature that  determines V-movement: unlike
French, English non-auxiliary verbs  cannot undergo V-movement because Agr in
French is `transparent'  (or `strong') whereas Agr in English is `opaque' (or
`weak'). The  richness of French verbal morphology is assumed to provide
the  motivation for the strength of French Agr, in consequence of which the
raised verb in French can transmit theta roles to its arguments through AGR,
thus avoiding any violation of the theta criterion.  But the weakness of
English Agr (assumed to follow from the paucity of English verbal
morphology) is what blocks lexical verbs from assigning theta roles once
they have moved to Tns. Hence movement of a theta-assigning verb in English
would result in a violation of the theta criterion.

The basic spirit of Pollock's analysis---that `morphology determines syntactic movement'---has
remained essentially unchanged for the last decades though what triggers V-movement has varied
considerably in subsequent work (see, among others, \citet{Zanuttini:01}, \citet{Bo:14},
\citet{Zeijlstra:15}).  As far as we are aware, there is no agreed upon movement-based analysis of
either the English or French systems. In fact, as \citet{Lasnik:00}) stresses, the Minimalist
Program as articulated in \citet{Chomsky:93,Chomsky:95,Chomsky:00-mini-inq} not only fails to deal with the
ungrammaticality of simple examples like \textit{*John left not} or \textit{*John not left}, it also
provides no basis for explaining the French/English contrasts in adverb position discussed by
\citet{Pollock:89} and \citet{Cinque:99} (e.g., \textit{embrasse souvent} vs.\ \textit{often
  kisses}).\index{interpretable}


The derivational view summarized here has focused on adverbial negatives
in English and French. This view with movement operations in the
 hierarchy of functional projections has been extended to account for the other types of negation as well, which we will note in due course.

\section{A Construction-based HPSG Analysis}

Departing from the derivational view, we herein offer an alternative construction"=based view in which the distributional possibilities of negatives
are drawn from the interplay among the lexical properties of each negative marker
and from the interaction of elementary, independently motivated
morphosyntactic and valence properties of syntactic heads, and constructional
properties (see \citet{Kim:00}, \citet{KS:02}, \citet{Crowgey:12}).


\subsection{Morphological Negation}

As noted earlier, languages like Turkish and Japanese employ morphological negation in which the negative marker behaves like a suffix. Consider
Turkish and Japanese examples:
% again:



\eal
\ex
\gll Git-me-yece\~{g}-$\varnothing$-im \\
     go-\NEG-\FUT-\COP-\textsc{1sg} \\
\glt `(I) will not come.'
\ex
\gll kare-wa kinoo kuruma-de ko-na-katta. \\
     he-\textsc{tpc} yesterday car-\textsc{inst} come-\textsc{neg}-\textsc{pst} \\
\glt `He did not come by car yesterday.'
\zl




\noindent
As the examples illustrate, the sentential negation of Turkish and Japanese employ
morphological suffixes  \suffix{me} and \suffix{na},
respectively.
It is possible to state the ordering
of these morphological negative markers in derivational
or syntactic terms. But it is too strong a claim to
take the negative suffix \suffix{me} or \suffix{na}  to be an independent syntactic element,
and to attribute its positional possibilities to syntactic constraints
such as verb movement and other configurational notions (see \citet{kelepir} for
Turkish and \citet{Kato:97,Kato:00} for Japanese).
%Kelepir 2001
%Japanese and Turkish show other clear examples of morphological negation.
%
In these languages, the negative morpheme acts just like
other verbal inflections in numerous respects.
%
%\enumsentence{
%\shortex{4}
%
%{T\"{u}rk-les-tir-il-me-mis-ler-den-siniz.}
%{turk-become-CAUS-PASS-NEG-PSP-PLUR-ABL-COP}
%{`You are of those who didn't have themselves Turkified.' (van
%Schaaik 1994:39)}}
%
%\noindent
%
%
The morphological status of
these negative markers comes from their morphophonemic alternation.
For example, the vowel of the Turkish negative suffix \suffix{me} shifts from open to closed when followed by the
future suffix, as in \emph{gel-mi-yecke} `come-\NEG-\FUT'.  Their
strictly fixed position also indicates their morphological
constituenthood. Though these languages allow rather a free permutation of
syntactic elements (scrambling), there exist strict ordering restrictions among
verbal suffixes including the negative suffix, as can be seen from
the following examples:

\eal
\ex
\gll tabe-sase-na-i/*tabe-na-sase-i \\
     eat-\textsc{caus}-\textsc{neg}-\textsc{npst} \\

\ex
\gll tabe-rare-na-katta/*tabe-na-rare-katta \\
     eat-\textsc{pass}-\textsc{neg}-\textsc{pst} \\

\ex
\gll tabe-sase-rare-na-katta/*tabe-sase-na-rare-katta \\
     eat-\textsc{caus}-\textsc{pass}-\textsc{neg}-\textsc{pst} \\
\zl

\noindent
The ordering of the negative morpheme is a matter of morphology.
If it were a syntactic concern, then
the question would arise as to why
there is an obvious contrast in the ordering principles
of morphological and syntactic constituents, i.e., why the ordering
rules of morphology are distinct from the ordering rules of syntax. The
simplest explanation for this contrast is to accept
the view that morphological constituents including the negative marker
are formed in the lexical component and hence have no syntactic
status (see \citet{Kim:00} for detailed discussion).

This being noted, it is more reasonable to assume that the placement of a
negative morpheme is regulated by morphological principles, i.e.\ by
the properties of the morphological negative morpheme itself. In the
construction-based HPSG, we could take this as an inflectional
construction.  The negative marker, as we have seen in Turkish and Japanese, is realized as a suffix
attached to the verb root. The resulting combination is not
a word-level entity but a verb stem to which an aspectual or tense marker can be attached. We could thus take such a morphological process as an inflectional
one. For instance, Figure~\ref{fig:2} could be a morphological
construction in Turkish.\footnote{See \citet{Sag:12} and \citet{Hilpert:16} for a construction-based approach to
inflectional as well as derivational processes.}



%\newpage
\begin{figure}
	\centering	
	%\vspace{-.5cm}	
	\begin{forest}
		[
		\begin{avm}
			\[\tpv{v-neg-stem}\\
			form & \< \textbf{F}\textsubscript{NEG} (X)\>\\
			%%
			syn|head|pos & \tpv{verb}\\
			%%
			sem|frames & \< \[\tpv{neg-fr}\\
			arg \@{1} \] \>
			\]	
		\end{avm}
		%%%%%%
		[
		\begin{avm}
			\[ \tpv{v-lxm}\\
			form & \< X \>\\
			syn|head|pos & verb\\
			sem|frames & \<\@{1} \>
			\]
		\end{avm}
		]
		]
	\end{forest}
	\caption{Negative-Infl Construction ($\uparrow$\emph{infl-cxt})}\label{fig:2}
\end{figure}	

This inflectional construction ($\uparrow$\type{infl-cxt}) allows us to generate a Turkish inflection construct like \textit{ser-me} `like-\NEG' (in (\ref{1a})) from the v-lexeme \textit{ser-} with the change in the root's meaning into a sentential negation. The morphological function F\sub{\textsc{neg}} could ensure that the vowel of the negative morpheme \textit{me} is subject to phonological changes depending on its environment. If it is followed by a consonant-initial morpheme, it undergoes vowel harmony with the vowel in the preceding syllable (e.g., \textit{yika-n-ma-di} `wash-\REFL-\NEG-\PST'). If it is followed by a vowel-initial morpheme, its vowel drops (gel-m-iyor `come-\NEG-\PROG') (see \citet{kelepir}).\footnote{As
 for a way of capturing the ordering of suffixes within this kind of system,
 see \citet{Kim:16}.}


The construction-based analysis sketched here
  has been built upon the
thesis that autonomous (i.e.\ non-syntactic) principles govern the
distribution of morphological elements \citet{BM:95}.
The position of the morphological negation is simply
defined in relation to
the verb stem it attaches to. There are no syntactic operations such
as head-movement or multiple functional projections in forming
a verb with the negative marker.



\subsection{Negative Auxiliary Verb}

Another way of expressing sentential negation, as noted earlier, is to employ
a negative auxiliary
verb. Head-final languages like Korean and Hindi employ
negative auxiliary verbs. Consider a Korean example:

%\footnote{But in some SOV
%languages, the position of the negative auxiliary is preverbal.
%Consider Hindi examples from Mohanan 1995.
%
%\enumsentence[i]{
%\shortex{4}
%{anil & kitaab\~{e} & nah\~{\i}\~{\i} & becegaa.}
%{Anil-N &  book-N-PL(M) & not & sell-FU.M}
%{`Anil will not sell the books.'}}
%}



\ea
\gll John-un ku chayk-ul ilk-ci anh-ass-ta. \\
     John-\textsc{tpc} that book-\textsc{acc} read-\textsc{conn} \textsc{neg}-\textsc{pst}-\textsc{decl}  \\
\glt `John did not read the book.'
\z

\noindent
The negative auxiliary in head-final languages
typically appears
clause-finally, following the invariant form of the main verb.
In head-initial SVO languages, however, the negative auxiliary
almost invariably occurs immediately before the lexical verb
(see \citet{Payne:85}). Finnish exhibits this property \citep{Mitchell:91}:

\ea
\gll Min\"{a} e-n puhu-isi. \\
     I-\textsc{nom} \textsc{neg}-\textsc{1sg} speak-\textsc{cond} \\
\glt `I would not speak.'
\z


\noindent
These negative auxiliaries have syntactic status: they can be
inflected, above all. Like other verbs, they can be marked
with verbal inflections such as agreement, tense, and mood.

In dealing with auxiliary negative constructions,
most of the derivational approaches have
followed Pollock's and Chomsky's analyses in factoring out functional
information carried by lexical items into
various different phrase-structure nodes (see, among others, \citet{Hagstrom:97,Hagstrom:02}, \citet{Han:07} for Korean and \citet{Vasishth:00} for Hindi).
%
%PAUL HAGSTROM
%Journal of East Asian Linguistics 11, 211?242, 2002.
%? 2002 Kluwer Academic Publishers. Printed in the Netherlands.
%IMPLICATIONS OF CHILD ERRORS FOR THE
%SYNTAX OF NEGATION IN KOREAN
%Journal of East Asian Linguistics 11, 211?242, 2002.
%? 2002 Kluwer Academic Publishers. Printed in the Netherlands.
%JOURNAL ARTICLE
%V-Raising and Grammar Competition in Korean: Evidence from Negation and %Quantifier Scope
%Chung-hye Han, Jeffrey Lidz and Julien Musolino
%
This derivational view has
been appealing, in that one identical structure could explain
different types of negation.   However,
problems have arisen from the fact that it misses the basic properties
of this type of negation which, for example, differentiate it from
morphological negation (i.e., double negation, lexical
idiosyncrasies, phonological restriction, etc).\footnote{See \citet{Nino:94} for arguments against a derivational analysis
for Finnish negative auxiliary such as that of \citet{Mitchell:91}.}


In the construction-based HPSG analysis, the
negative auxiliary is taken to be an independent lexical
verb whose functional information is not distributed
over different phrase structure nodes, but incorporated into
its precise and enriched lexical entry and an independently
motivated construction for other types of auxiliary verbs.
The Korean negative auxiliary displays all the key properties of auxiliary verbs in the language. For instance, the typical auxiliary verbs as
well as the negative auxiliary all require the preceding main verb to be marked with a specific verb form (\vform), as illustrated
in the following:





\eal
\ex\label{14a}
\gll ilk-ko/*ci siph-ta. \\
     read-\textsc{conn} would.like-\textsc{decl} \\
\glt `(I) would like to read.'

\ex\label{14b}
\gll ilk-ci anh-ass-ta. \\
     read-\textsc{conn} not-\textsc{pst}-\textsc{decl} \\
\glt `(I) did not read.'
\zl


The auxiliary verb \textit{siph-} in (\ref{14a}) requires the
\textit{-ko} marked main verb while the negative auxiliary
 verb \textit{anh-} in (\ref{14b}) asks for the \textit{-ci} marked main verb.




In terms of syntactic structure, there
are two possible analyses.  One is to assume that the negative auxiliary
takes a VP complement and the other is to claim that it forms a verb complex
with the preceding main verb, as represented in Figures~\ref{fig:3a} and~\ref{fig:3b}, respectively (\citep{Kim:16}).

\begin{figure}
	\begin{subfigure}[b]{0.48\textwidth}
\centering
		\begin{forest}
%		sm edges
			[VP
				[VP
					[ \dots\ ] 
					[V {[\textsc{vform} \type{ci}]}
					]
					]
				[V {[\textsc{aux $+$}]}
					[anh-ta\\ \textsc{neg-decl}]
				]
			]	
		\end{forest}
	\caption{Add caption}\label{fig:3a}
		\end{subfigure}	
\hfill
	\begin{subfigure}[b]{0.48\textwidth}
\centering
		\begin{forest}
%		sm edges
			[VP, s sep=1cm
				[ \dots\ ]
				[V
					[V {[\textsc{vform} \type{ci}]}
						[\dots]]
					[V {[\textsc{aux $+$}]}
						[anh-ta\\ \textsc{neg-decl}]]]]
		\end{forest}
	\caption{Add caption}\label{fig:3b}	
		\end{subfigure}
	\caption{Add caption}
\end{figure}

The distributional properties of the negative auxiliary in the language, however, support
 the complex predicate structure (cf.\ Figure~\ref{fig:3b}) in which the negative auxiliary verb
forms a syntactic/semantic unit with the preceding main verb.
For instance, no adverbial expression, including
a parenthetical adverb, can intervene between
the main and auxiliary verb, as illustrated by the
following Korean example:

\ea
\gll Mimi-nun (yehathun) tosi-lul (yehathun) ttena-ci (*yehathun) anh-ass-ta. \\
     Mimi-\textsc{tpc} anyway city-\textsc{acc} anyway leave-\textsc{conn} anyway \textsc{neg}-\textsc{pst}-\textsc{decl} \\
\glt `Anyway, Mimi didn't leave the city.'
\z
%
Further, in an elliptical construction, a verb
complex always occurs together:



\eal\ex \gll Kim-i hakkyo-eyse pelsse tolawa-ss-ni? \\
Kim-\textsc{nom} school-\textsc{src} already return-\textsc{pst}-\textsc{que} \\
\trans`Did Kim return from school already?'

\ex \gll ka-ci-to anh-ass-e \\
go-\textsc{conn}-\textsc{del} \textsc{notp}-\textsc{pst}-\textsc{decl} \\
\trans`(He) didn't even go.'

\ex[*] {ka-ci-to. go-\textsc{conn}-\textsc{del}}

\ex[*] {anh-ass-e \textsc{neg}-\textsc{pst}-\textsc{decl}}
\end{xlist} \end{exe}
%
Neither the main verb nor the auxiliary verb alone can serve
as the fragment answer to the polar question. The two verbs
must occur together.

These constituent tests
indicate that the negative auxiliary forms
a syntactic unit with a preceding main verb in Korean.
Following \citet{Bratt:96}, \citet{Chung:98}, and \citet{Kim:16},
we then could assume that
an auxiliary verb forms a complex predicate, licensed by
the following construction:
%
%
%What this implies is that in addition to the Subject-Predicate
%and Head-Complement Constructions, Korean employs
%another construction rule that allows the combination of two lexical-level
%expressions, as stated in the following:%\index{hd-lex-cxt@\textsl{hd-lex-cxt}}





\ea
\label{hd-lex-cxt}
\hd-lite:\\
\begin{myavm}\small
\[\type{hd-lex-cxt }\\
\COMPS\  \textit{L}\]    $\Rightarrow$ \@1 \[\LEX\  +\\
                                  \COMPS\  \textit{L}\], H \[%\AUX\ \; +\\
                                           \COMPS\  \q<\@1\q>\]
\end{myavm}
\z


\noindent   This construction rule means that a lexical head
expression combines with its lexical (\lite) complement. When this combination happens,
there is a kind of argument composition: the \COMPS\  value (\textit{L}) of this
lexical complement is passed up to the resulting mother.
The constructional constraint thus induces the effect of argument composition at syntax,
as illustrated by the following example:\footnote{The V$'$ is just a
notational variant to indicate that it is a syntactic complex predicate.}

\begin{figure}
	\begin{forest}
		[V$'$\\
		\begin{avm}
			\[\tp{hd-lex-cxt}\\
			head & \@{3}\\
			lex & $+$\\
			comps & \<\@{2}\,NP\>\]
		\end{avm}, l sep*=3
			[\ibox{1}\,V\\
			\begin{avm}
				\[head
				\[vform & ci\\
				lex & $+$\\
				comps & \<\@{2}\,NP\>\]\] 
			\end{avm}, edge label={node[midway,left,outer sep=1.5mm,]{Lexical Arg.}}
				[ilk-ci\\read-\textsc{conn},tier=word]]
			[V\\
			\begin{avm}
				\[head & \@{3}\\
				comps & \<\@{1}\,V\>\]
			\end{avm}, edge label={node[midway,right,outer sep=1.5mm,]{H}}
					[anh-ass-ta\\ \textsc{neg-pst-decl},tier=word]]]
	\end{forest}
\caption{Add caption}
\end{figure}

The auxiliary verb \emph{anh-ass-ta} `\NEG-\PST-\DECL' combines with the matrix verb \textit{ilk-ci} `read-\conn',
forming a well-formed head-lex construct.\footnote{The negative auxiliary
verb selects two arguments, a subject and a main verb. See \citet{Kim:16} for
a detailed analysis.}
Note that the resulting construction inherits the
\COMPS\ value from that of the lexical complement \textit{ilk-ci} `read-\conn' through the operation of argument composition.
It is the \hd-lite\ that licenses
the combination of an auxiliary verb with its main verb, while
inheriting the main verb's complement value as argument composition. The present system thus allows the argument composition at the syntax level, rather than
in the lexicon.

One important property of the auxiliary construction is that there is no limit for auxiliary verbs to
occur in sequence as long as each combination observes
the morphosyntactic constraint on the preceding expression. Consider
the following:



\eal\ex \gll sakwa-lul mek-ci anh-ta. \\
apple-\textsc{acc} eat-\textsc{conn} \textsc{neg}-\textsc{decl} \\

\ex \gll sakwa-lul mek-ko siph-ci anh-ta. \\
apple-\textsc{acc} eat-\textsc{conn} wish-\textsc{conn} \textsc{neg}-\textsc{decl} \\

\ex \label{20c} \gll sakwa-lul mek-ko siph-e ha-ci anh-ta. \\
apple-\textsc{acc} eat-\textsc{conn} wish-\textsc{conn} do-\textsc{conn} \textsc{neg}-\textsc{decl} \\

\ex \gll sakwa-lul mek-ko siph-e ha-key toy-ci anh-ta. \\
apple-\textsc{acc} eat-\textsc{conn} wish-\textsc{conn} do-\textsc{conn} become-\textsc{conn} \textsc{neg}-\textsc{decl} \\
\end{xlist} \end{exe}



  As seen from each of these examples, we can add one more auxiliary verb to the
  existing construction, with an appropriate
  connective marker on the preceding one. Theoretically, there is no upper limit to the possible number  of auxiliary
  verbs we can add.

  Within the present complex-predicate analysis with
  the supposition of \hd-lite\ in the language, we could
  license all these examples. Figure~\ref{fig-apple-eat-wish-do} is
  a simplified structure for (\ref{20c}):

\begin{figure}
	\begin{forest}
		sm edges
		[VP
			[\ibox{2}\,NP
				[sakwa-lul;apple-\textsc{acc},roof]]
			[V$'$\\
			\begin{avm}
				\[head & verb\\
				comps & \<\@{2}\>\]
			\end{avm}
				[\ibox{5}\,V$'$\\
				\begin{avm}
					\[vform & ci\\
					comps & \<\@{2}\>\]
				\end{avm}
						[\ibox{3}\,V$'$\\
						\begin{avm}
							\[vform & ae\\
							comps & \<\@{2}\>\]
						\end{avm}
							[\ibox{7}\,V\\
							\begin{avm}
								\[vform & ko\\
								comps & \<\@{2}\>\]
							\end{avm}
								[mek-ko;eat-\textsc{conn}]]
							[V\\
							\begin{avm}
								\[vform & ae\\
								comps & \<\@{7}\>\]
							\end{avm}
								[siph-e;wish-\textsc{conn}]]]
						[V\\
						\begin{avm}
							\[vform & ci\\
							comps & \<\@{3}\>\]
						\end{avm}
							[ha-ci;do-\textsc{conn}]]]
					[V\\
					\begin{avm}
						\[comps & \<\@{5}\>\]
					\end{avm}
						[anh-ass-ta;\textsc{neg-pst-decl}]]]]
	\end{forest}
\caption{Add caption}\label{fig-apple-eat-wish-do}
\end{figure}
%
%
The bottom structure indicates that the auxiliary verb \textit{siph-e} wish-\conn forms a
\hd-lite\ through the combination with the main verb \textit{mek-ko} eat-\conn.
This resulting complex predicate, which
is still a \LEX\ expression, inherits the main verb's \COMPS\ value
as well as the \textit{ae} \VFORM\ head feature from the auxiliary.\footnote{Instead,
we can adopt the feature \textsc{light} to a lexical expression, as suggested for
the French Auxiliary Construction by \citet{AG:97}.} Meanwhile,
the auxiliary verb \textit{ha-ci} also requires a \LITE\  level
expression with the \VFORM\ value {\textit{ae}}, combining
with the preceding complex predicate in a legitimate way.
This combination, forming a \hd-lite, again inherits  the \COMPS\
value. The final negative
auxiliary then combines this resulting complex predicate,
yielding a final complex predicate that can combine with the object. Each combination thus
forms a well-formed complex predicate, licensed by the lexical projection
of each auxiliary verb and the \hd-lite.

The present analysis has taken the negative auxiliary \textit{ahn-ta} \NEG-\DECL
to select the main verb and form a verb complex with it.
This verb complex treatment has been supported from
constituent tests including
 adverb intervention and elliptical constructions. Further, the analysis,
exploiting the mechanism of argument composition,
allows us to capture the properties of this negative
auxiliary.
%, and provides a simple and straightforward explanation for
%phenomena such as aspect selection.
The conclusion we can draw from here is that the
 distribution of a negative auxiliary verb is determined by
independent constructional constraints
that regulate the placement of other
similar verbs.



\subsection{Adverbial Negation}

\subsubsection{Two Key Factors}

The third main type of negation is
the adverbial negative marker which most of the Indo-European
languages employ. There are two main factors
that determine the position of an adverbial negative: finiteness of
the verb and its intrinsic properties, namely, whether it is an auxiliary
or main verb (see \citet{Kim:00}, \citet{KS:02}).\footnote{German also
employs an adverbial negative \emph{nicht}, which behaves quite
differently from the negative in English and French. See \citet{MuellerGT-Eng1}
for a detailed review of the previous, theoretical analyses of German negation.}

%\subsubsection{Finiteness vs.\ Non-finiteness}

The first crucial factor that affects  the position of adverbial
negatives in English and French concerns the finiteness of the main verb.
French shows us how the finiteness of a verb influences the
surface position of the negative marker \textit{pas}.



\eal\ex[]{
\gll Robin  n'aime  pas  Stacey. \\
     Robin  (n')likes  \textsc{neg} Stacey \\
\trans`Robin does not like Stacey.'
}
\ex[*]{
Robin ne pas aime Stacey.
}
\zl


\eal\ex[]{
\gll Ne  pas  parler   Fran\c{c}ais  est  un  grand d\'{e}savantage  en ce cas. \\
ne \textsc{neg}  to.speak  French  is  a great disadvantage  in this case \\
\trans `Not to speak French is a great disadvantage in this case.'
}
\ex[*]{
Ne  parler  pas  Fran\c{c}ais  est  un  grand d\'{e}savantage en ce cas.
}
\zl

\noindent
The negator \textit{pas} cannot precede the finite verb
but must follow it. But its placement with respect to
the nonfinite verb is the reverse image. The negator \textit{pas}
should precede the infinitive verb.
English is not exceptional in this respect (\citet{Baker:89,Baker:91}, \citet{Ernst:92}).
The negation \textit{not} precedes an infinitive verb, but cannot follow
a finite main verb.\footnote{A similar contrast between finiteness and
nonfiniteness can be observed in the Scandinavian language like Norwegian (see \citet{Platzack:86}, \citet{HP:88}, and \citet{Vikner:94,Vikner:97}).}

\eal\ex[]{
\gll Jon  skj\o nte   aldri  dette sp\o rsm\.{a}let. \\
Jon  understood  never this question \\
\trans `John never understood this question.'
}
\ex[]{
\gll Han  hadde  foresatt  seg  aldri  a sla  hunden. \\
He   had  decided  himself  never  to beat  the.dog \\
\trans `He had decided himself never to beat the dog.'
}
\zl

\begin{exe}
\ex\label{eng-fin-neg} \begin{xlist}
\ex[]{
Kim does not like Lee.
}
\ex[*]{
Kim not likes Lee.
}
\ex[*]{Kim likes not Lee.
}
\zl

\pfix
\pfix\pfix

\begin{exe}
\ex\label{fr-fin-neg} \begin{xlist}
\ex[]{
Kim is believed [not [to like Mary]].
}
\ex[*]{
Kim is believed to [like not Mary].
}
\zl



%
%
%
%
%
%\subsubsection{An Intrinsic Property of the Verb}

The second important factor that determines the position of adverbial
negatives concerns the presence of an auxiliary or main verb.
Modern English displays a clear example where this
intrinsic property of the verb influences the position of
the English negator \textit{not}: the negator cannot follow
a finite main verb but when the finite verb is an auxiliary verb,
this ordering is possible.

\eal
\ex[*]{
Kim left not the town.
}
\ex[]{
Kim has not left the town.
}
\ex[]{
Kim is not leaving the town.
}
\zl

\noindent
The placement of \textit{pas} in French infinitival
clauses also illustrates that the intrinsic property of
the verb affects the position of the adverbial negative \textit{pas}:

\eal
\ex[]{
Ne pas avoir de voiture dans cette ville rend la vie difficile. \\
`Not to have a car in this city makes life difficult.'
}
\ex[] {
N'avoir pas de voiture dans cette ville rend la vie difficile.
} \label{28b}
\zl

\eal
\ex[]{
Ne pas \^{e}tre triste est une condition pour chanter des chansons. \\
`Not to be sad is a prerequisite condition for sining songs.'
}
\ex[]{
N'\^{e}tre pas triste est une condition pour chanter des chansons.
} \label{29b}
\zl

\noindent
The negator \textit{pas} can either follow or precede the infinitive
auxiliary verb in French, though the acceptability of the
ordering in (\ref{28b}) and (\ref{29b}) is restricted to certain conservative
varieties.



%But only the postverbal position is possible for \textit{pas} when the verb is the %finite form.
%
%\eenumsentence{
%\item\shortexnt{1}
%{Jean n'a pas une voiture.}
%{`John does not have a car.'}
%\item
%{\bad Jean ne pas a une voiture.}}


In capturing the distributional behavior of such negatives
in English and French, as we have noted earlier,
the derivational view (exemplified by \citet{Pollock:89} and \citet{Chomsky:91})
has relied on the notion of verb
movement and functional projections.  The most appealing aspect of this
view (initially at least) is that it can provide an analysis of the
systematic variation
between English and French. By simply assuming that the
two languages have different scopes of verb movement -- in English
only auxiliary verbs move to a higher functional projection whereas
all French verbs undergo the same process, the derivational
view could explain why the French negator \textit{pas} follows
a finite verb, unlike the English negator.  In order for this system to succeed,
nontrivial complications are required in the basic components of the
grammar, e.g rather questionable subtheories.
For example, the introduction of Pollock's theta and quantification
theories has been necessary to account for the obligatory verb
movement.\footnote{His theta theory says only nonthematic verbs move
up to the higher functional position, whereas his quantification
theory says [$+$fin] is an operator that must bind a variable.}
However, when these subtheories interact with each other,
they bring about a `desperate' situation, as \citet[398]{Pollock:89} himself concedes: his quantification theory forces
all main verbs in English to undergo verb movement, but his
theory blocks this. This contradictory outcome has forced him to adopt
an otherwise unmotivated mechanism, a dummy
nonlexical counterpart of \textit{do} in English (which \citet{Chomsky:89} tries
to avoid by adopting the notion of LF re-raising).
Leaving the plausibility of this mechanism aside,  as
discussed by \citet{Kim:00} and \citet{KS:02},
a derivational analysis such as that of \citet{Pollock:89}
fails to allow for all the distributional possibilities of
English and French negators as well as adverb positioning in
various environments

\iffalse{
In capturing the interaction with auxiliary verbs, derivational analyses have chosen the direction of generating
auxiliaries and main verbs in different positions. For example,
\citet{Pollock:89}'s system for English auxiliaries posits
various different positions for different verbs: main
verbs and \textit{have} and \textit{be} under V within the VP,
\textit{do} under Agr, modals such as \textit{will, may}, and \textit{can}
under T.\footnote{See \citet{Ouhalla:91}'s system in which
all auxiliaries are generated under the head of AspP.} But for French,
all verbs, whether
auxiliary or main verbs, are generated under V.
This contrast does not seem to be unreasonable, considering that in
modern French no syntactic phenomenon clearly distinguishes auxiliary
verbs and main verbs. Leaving aside the question of why the two
typologically related languages have such different ways of generating
verbs including auxiliaries, Pollock's system has suffered
from problems in capturing the distribution of \textit{not} and
\textit{pas} in \emph{have}/\emph{avoir} and modal constructions.}\fi
%as well as the properties of \textit{have/avoir}.
%This has led the system to introduce rather weakly motivated
%and questionable assumptions, e.g.\ an exotic structure for the main
%verb usage of \textit{have/avoir}




%\footnote{This
%distinction has been required
%in Pollock's theory since [+fin] tense requires verb
%movement to Tense, prohibits affix movement in French and turns {\it
%not} into a block for affix movement
%in English, but [-fin] does not require verb movement, does not prohibit affix
%movement and allows \textit{not} not to count as a block for affix
%movement. See Pollock (1989:391--395) for further details
%}.

\subsubsection{Constituent Negation in English and French}


The construction-based, lexicalist analysis we offer here also recognizes
the fact that finiteness plays a crucial role in
determining the distributional possibilities of negative
adverbs. Its main explanatory resource
has basically come from the proper lexical specification of these negative
adverbs. The lexical specification that \emph{pas} and
\emph{not} both modify nonfinite VPs has sufficed to predict their
occurrences in nonfinite clauses.



When English \textit{not} negates an embedded constituent, it behaves
much like the negative adverb \textit{never}. The similarity between {\it
not} and \textit{never} is particularly clear in nonfinite verbal
constructions (participle, infinitival and bare verb phrases), as
illustrated in (\ref{30}) and (\ref{31}) (\citet{Klima:64}, \citet{Baker:89,Baker:91}).

\eal\label{30}
\ex[]{
Kim regrets [never [having read the book]].
}
\ex[]{
We asked him [never [to try to read the book]].
}
\ex[]{
Duty made them [never [miss the weekly meeting]].
}
\zl

\eal\label{31}
\ex[]{
Kim regrets [not [having read the book]].
}
\ex[]{
We asked him [not [to try to read the book]].
}
\ex[]{
Duty made them [not [miss the weekly meeting]].
}
\zl

\noindent
French \textit{ne-pas} is no different in this regard.  \textit{Ne-pas} and
certain other adverbs precede an infinitival VP:



\eal
\ex[]{
\gll
[Ne  pas  [repeindre    sa    maison]]  est   une n\'{e}gligence. \\
ne  not   paint     one's  house    is  a    negligence \\
\trans `Not to paint one's house is negligent.'
}
\ex[]{
\gll
[R\'{e}guli\`{e}rement   [repeindre    sa   maison]]   est  une  n\'{e}cessit\'{e}. \\
regularly        to.paint    one's   house    is    a   necessity \\
}
\zl


To account for these properties, we regard \textit{not} and \textit{ne-pas} not as
heads of their own functional projection, but rather as adverbs that modify
nonfinite VPs. The lexical entries for \textit{ne-pas} and \textit{not} include the
information shown in (\ref{c-neg}).\footnote{Here we assume that both languages
distinguish between \textit{fin(ite)} and \textit{nonfin(ite)} verb forms, but that
certain differences exist regarding lower levels of organization. For example,
\type{prp} (\type{present participle}) is a subtype of \type{fin} in French,
whereas it is a subtype of \type{nonfin} in English.

%In \ex{1}, VP[\type{nonfin}]:\lower4pt\hbox{\begin{avm}\@2\end{avm}}
%abbreviates a nonfinite VP whose CONTENT value is
%\lower4pt\hbox{\begin{avm}\@2\end{avm}}.  Similar abbreviations are
%used throughout.

For ease of exposition, we will not treat cases where the negation modifies
something other than VP, e.g.\ adverbs (\textit{not surprisingly}), NPs (\textit{not
many students}), or PPs (\textit{not in a million years}). Our analysis
can accommodate such cases by generalizing the \textsc{mod} specification in
the lexical entry for \textit{not}.}

\ea
\label{c-neg}
\begin{avm} \avml
 %\textit{not}/\textit{ne-pas} \\
 \[\FORM\ \q<\normalfont\textit{not}/\textit{ne-pas}\q>\\
\SYN\|\HEAD\ \[\POSP\ \type{adv}\\
               \MOD\ \q<VP [\type{nonfin}]: \@2\q>\]\\
  \SEM\ \[\FRAMES\ \<\[\type{neg-fr}\\
                       \ARGa\ \@2\]\>\]
  \]\avmr\end{avm}
\z



\noindent %[JB: begins] CONT is added
The lexical entry in (\ref{c-neg}) specifies that
\textit{not} and \textit{ne-pas} modifies a nonfinite VP and that this
modified VP serves as the semantic argument of the negation.
%[JB: ends]
This simple lexical specification correctly describes the
distributional similarities between English \emph{not} and French
\emph{ne-pas}: neither element can separate an infinitival verb
from its complements.\footnote{The exception to this
generalization, namely cases where \textit{pas} follows an auxiliary
infinitive (\textit{n'avoir pas d'argent}), is discussed in section
5.2 below.} And both \emph{ne-pas} and \emph{not}, like other
adverbs of this type, precede the VPs that they modify:

\eal
\ex[]{
\gll
[Ne  pas  \ssub{VP[\type{inf}]}[parler  fran\c{c}ais]]  est  un grand d\'{e}savantage  en ce cas. \\
ne  not  \hspace{0.5in}to.speak French  is  a great disadvantage  in this case \\
} \label{34a}
\ex[*]{
Ne  parler  pas  fran\c{c}ais  est  un  grand d\'{e}savantage en ce cas.
} \label{34b}
\zl



\eal
\ex[] {
[Not [speaking English]] is a disadvantage.
} \label{35a}
\ex[*] {
[Speaking not English] is a disadvantage.
} \label{35b}
\zl



%\item{\bad Lee likes not Kim.}}
%
%\eenumsentence{
%\item{Lee is believed [not $_{VP[inf]}$[to like Kim]].}
%\item{\bad Lee is believed to $_{VP[inf]}$[like not Kim].}}

\noindent Independent principles guarantee that modifiers of this
kind precede the elements they modify, thus ensuring the
grammaticality of (\ref{34a}) and (\ref{35a}), where \textit{ne-pas} and \textit{not} are used as VP[\type{nonfin}] modifiers.
(\ref{34b}) and (\ref{35b}) are ungrammatical, since
the modifier fails to appear in the required position---i.e.\
before all elements of the nonfinite VP.

The lexical properties of \textit{not} thus ensures that it cannot
modify a finite VP, as shown in (\ref{36}), but it can modify any
nonfinite VP:
%, as is clear from the examples in \ex{2}:

\eal\label{36}
\ex[*]{
Pat [not \ssub{VP[fin]}[left]].
}
\ex[*]{
Pat certainly [not \ssub{VP[fin]}[talked to me]].
}
\ex[*]{
Pat [not \ssub{VP[fin]}[always agreed with me]].
}
\zl



%\eenumsentence{
%\item{I saw Pat acting rude and [not $_{VP[prp]}$[saying hello]].}
%\item{I asked him to [not $_{VP[\type{bse}]}$[leave the bar]].}
%\item{Their having [not $_{VP[\type{psp}]}$[told the truth]] was upsetting.}}

\noindent And much the same is true for French, as the
following contrast illustrates:

\eal
\ex[*]{
\gll Robin  [(ne) pas \ssub{VP[\type{fin}]}[aime  Stacey]]. \\
Robin  [(ne) not \hspaceThis{\ssub{VP[\type{fin}]}[}likes Stacey] \\
}
\ex[]{
Il veut [ne pas publier dans ce journal]. \\
`He wants not to publish in this journal.'
}
\zl


Note that head-movement transformational analyses stipulate: (1) that negation
is generated freely, even in preverbal position in finite clauses and (2) that
a post-negation verb must move leftward because otherwise some need would be
unfulfilled---the need to bind a tense variable, the need to overcome some
morphological deficiency with respect to theta assignment, etc. On our
account, no such semantic or morphosyntactic requirements are stipulated;
instead, what is specified is a lexical selection property. There is no a
priori reason, as far as we are aware, to prefer one kind of stipulation over
the other. It should be noted, however, that our proposal only makes reference
to selectional properties that are utilized elsewhere in the grammar.


\subsubsection{Sentential Negation in English}


As just
illustrated, the analysis of \emph{not} and \emph{ne-pas} as nonfinite VP modifiers provides a straightforward explanation for much of their distribution. We may simply assume that French and English have essentially the
same modifier-head construction and that \textit{not} and \textit{ne-pas} have
near-identical lexical entries. With respect to negation in finite clauses,
however, there are important difference between English and French.

It is a general fact of French that \emph{pas} must follow the finite verb, in
which case the verb optionally bears negative morphology (\textit{ne}-marking):

\eal
\ex[]{
Dominique (n')aime pas Alex.
}
\ex[*]{
Dominique pas aime Alex.
}
\zl

\noindent
In English, \textit{not} must follow the finite verb, which
must in addition be an auxiliary verb:

\eal
\ex[]{
Dominique does not like Alex.
}
\ex[*]{ Dominique not does like Alex.
}
\ex[*]{ Dominique likes not Alex.
}
\zl

In contrast to the distribution of \textit{not}
 in nonfinite clauses as constituent negation, its distribution
 in finite clauses concerns sentential
 negation.
 The need to distinguish the two types of negation comes from scope
possibilities in an example like (\ref{not-two}) \citep{Klima:64},\citep{Baker:89}, and \citep{Warner:00}.

\ea[]{\label{not-two} The president could not approve the bill.
}
\z
%
Negation here could have the two different scope readings
paraphrased in (\ref{41}).


\eal\label{41}
\ex[]{
It would be possible for the president not to approve the bill.
}
\ex[]{
It would not be possible for the president to approve the bill.
}
\zl
%
The first interpretation is constituent negation; the second is
sentential negation. As noted, sentential \emph{not} may not modify a finite
VP, different from the adverb \textit{never}:



\eal
\ex[]{
Lee never/*not left.\ \ \ \ (cf.\ Lee did not leave.)
}
\ex[]{
Lee will never/not leave.
}
\zl
%
The contrast in these two sentences
shows one clear difference between \emph{never}
and \emph{not}. The negator \emph{not} cannot
precede a finite VP though it can freely occur
as a nonfinite VP modifier.
%, a
%property further illustrated by the following examples:
%
%\ees{\item John could [not [leave town]].
%
%\item John wants [not [to leave town]].}
%
%\ees{\item \bad John [not [left town]].
%
%\item \bad John [not [could
%leave town]].}

Another distributional difference between \emph{never} and \emph{not} is found in
the VP ellipsis construction.  Observe the following
contrast:

\eal
\label{vpe-not-ex}\ex[]{
Mary sang a song, but Lee never did \trace.
}
\ex[*]{
Mary sang a song, but Lee did never \trace.
}
\ex[]{
Mary sang a song, but Lee did not \trace.
}
\zl


%
The data here indicate that \emph{not} behaves differently from
adverbs like \emph{never} in finite contexts, even though the two
behave alike in nonfinite contexts. The adverb \emph{never} is a true
diagnostic of a VP-modifier, and we use contrasts between \emph{never} and \emph{not} to reason about what the properties of
the negator \emph{not} must be.

%\iffalse{

We saw the lexical representation for constituent negation
\emph{not} in (\ref{c-neg}) above.
 Sentential \emph{not} typically appears linearly in the
same position -- following a finite auxiliary verb -- but shows
different syntactic properties (while
 constituent negation need not follow an auxiliary
 as in \emph{Not eating gluten is dumb}).
 As a way
 to deal with the sentential negation in English,
  we follow \citet{Bresnan:01} and \citet{KM} in assuming that the sentential negation forms a unit with the preceding
finite auxiliary verb.  This can be supported from
the fact that English sentential negation requires the proximity of
a finite auxiliary or modal auxiliary on its left and that it can function as synthetic negation as \emph{n't}. That is, the auxiliary and the negator \emph{not} are fused into a single lexical
expression through contraction, as in \emph{won't, can't},
and so forth.\footnote{\citet{ZP:83} note that the contracted
negative \emph{n't} more closely
resembles word inflection than it does a `clitic' or `weak' word of
the kind that often occurs in highly entrenched word sequences
(e.g., \emph{Gimme}!). For example, as Zwicky and Pullum observe,
\emph{won't} is not the fused form one would predict based on
the pronunciation of the word \emph{will}, and such idiosyncrasies are far more characteristic of inflectional endings than clitic words.}



 With this assumption, the
present analysis, in particular, assumes that the combination of a finite auxiliary verb with the sentential
negation \emph{not} is licensed by the \hd-lite\ (similar to the one in Korean), which licenses
the combination of two lexical expressions such as verb and particle (e.g., \emph{figure out, give
  up}, etc), as well (see \citet{KM}). The construction, along with the assumption that the
sentential negator \emph{not} bears the \LEX\ feature, projects a structure like in Figure~\ref{fig:6}.

\begin{figure}
	\begin{forest}
		sm edges
		[VP, s sep=1cm
			[V$'$\\
			\begin{avm}
				\[\tp{hd-lex-cxt}\\
				aux & $+$\]
			\end{avm}
				[V
					[could]]
				[Adv {[\textsc{neg} $+$]}
				[not]]]
			[VP
				[leave town,roof]]]
	\end{forest}
\caption{Add caption}\label{fig:6}
\end{figure}

%
Just as a particle combines with the preceding main verb, forming a
head-lex structure,  expressions like
the negator \emph{not}, \emph{too, so} and \emph{indeed} combine with a
preceding auxiliary verb:

\eal
\ex[]{ Kim will not read it.
}
\ex[]{
Kim will too/so/indeed read it.
}
\zl
%
Expressions like \emph{too} and \emph{so} are used to
reaffirm the truth of the sentence in question and
follow a finite auxiliary verb.  We assume that the negator and these reaffirming expressions form
a unit with the finite auxiliary, resulting in a lexical-level construction.

Since the sentential negator is not a modifier of
the following VP-type expression,
we take it to be selected by a finite auxiliary verb, as a main verb selects a particle.
This means a finite auxiliary verb (\type{fin-aux}) can be projected into a corresponding
\NEG-introducing auxiliary verb (\type{neg-fin-aux}), as in Figure~\ref{fig:7}.
%
%What this implies is that the sentential negator \emph{not}, just like particles,
%bears the feature \LEX, so that it can be selected by a finite auxiliary verb:

\begin{figure}
	\begin{forest}
		[\begin{avm}
			\[\tp{neg-fin-aux}\\
			syn|head & \[aux & $+$\\
			vform & fin\\
			neg & $+$\]\\
			arg-st & \<\@{1}\,XP{,} $\textnormal{Adv}$ \[lex & $+$\\neg & $+$\]{,} \@{2}\,XP\>\]
		\end{avm}
			[\begin{avm}
				\[\tp{fin-aux}\\
				syn|head & \[aux & $+$\\
				neg & $+$\]\\
				arg-st & \<\@{1}\,XP{,} \@{2}\,XP\>\]
			\end{avm}]]
	\end{forest}
\caption{Negative Auxiliary Construction ($\uparrow$\type{post-infl-cxt})}\label{fig:7}
\end{figure}
%
This is a post-inflection construction that allows for words to
be derived from other words. We can take this mother-daughter relation as a kind of derivation whose input is a finite auxiliary verb (daughter)
and whose output is a neg-finite auxiliary (\type{fin-aux} $\rightarrow$ \type{neg-fin-aux}). That is, the finite auxiliary verb selecting just a
complement XP can be projected into a \NEG\ finite auxiliary that selects the negator
as its additional lexical complement that bears the feature \NEG\ as well
as the feature \LEX.
%%What we can observe here is the addition of a \NEG\ element in the \ARG-ST.
%
%
The output construction then licenses the structure in Figure~\ref{fig:8} for
sentential negation.
%This construction rule licenses a lexical head to combine with an expression (like particles)
%bearing the feature LEX, forming another lexical expression.

\begin{figure}
	\begin{forest}
		sm edges
		[VP\\
		\begin{avm}
			\[vform & fin\\
			aux & $+$\\
			comps & \<~~\>\]
		\end{avm}, s sep=1cm
			[V\\
			\begin{avm}
				\[\tp{hd-lex-cxt}\\
				vform & fin\\
				aux & $+$\\
				comps & \<\@{3}\,VP\>\]
			\end{avm}
				[V\\
				\begin{avm}
					\[vform & fin\\
					aux & $+$\\
					comps & \<\@{2}\,{[\textsc{neg} $+$]}{,} \@{3}\,VP\>\]
				\end{avm}
					[could]]
				[\ibox{2}\,Adv
					[not]]]
			[\ibox{3}\,VP
				[leave town,roof]]]
		\end{forest}
\caption{Add caption}\label{fig:8}
\end{figure}

As shown here, the negative finite auxiliary verb \type{could} selects two complements, the negator
\emph{not} and the VP \emph{leave town}. The finite auxiliary then first combines with the negator,
forming a well-formed head-lex construct. This construct then can combine with a VP complement,
forming a Head"=Complement construct.

By treating \emph{not} as both a modifier (constituent negation)
and a lexical complement (sentential negation), we can
account for
the scope differences in (\ref{not-two}) as well as double
negation examples like the following:

\eal
\ex[]{
You [must not] simply [not work].
}

\ex[]{
He [may not] just [not have been working].
}
\zl
%

In addition, the analysis can account for various other phenomena
including VP ellipsis we discussed in (\ref{vpe-not-ex}). The point
was that unlike \emph{never}, the sentential negation can
host a VP ellipsis.  The VP ellipsis after \emph{not} is
possible, given that any VP complement can be unexpressed, leaving
the sentential complement intact:\todostefan{rephrase, reference to figure}

\begin{figure}
	\begin{forest}
		sm edges
		[V$'$
			[V\\
			\begin{avm}
				\[head|aux & $+$\\
				val & \[spr & \<\@{1}\>\\
					comps & \<\@{2}\,$\textnormal{Adv}${[neg $+$]}\>\]\\
				arg-st & \<\@{1}{,} \@{2}{,} VP{[\tpv{bse}]}\>\]
				\end{avm}
					[could]]
			[\ibox{2}\,Adv\\
				\begin{avm}\[neg & $+$\]\end{avm}
					[not]]]
	\end{forest}
\caption{Add caption}
\end{figure}
%
As represented here, the auxiliary verb \emph{could} forms a
well-formed head"=complement construct with \emph{not} while its
VP[\emph{bse}] is unrealized (see \citet{Kim:00}, \citet{KS:08} for
detail.).

The sentential negator \emph{not} can `survive' VPE because it can be
licensed in the syntax as the complement of an auxiliary, independent
of the following VP.  However, an adverb like \emph{never} is only
licensed as a modifier of VP. Thus if the VP were elided, we would have the hypothetical
structure like the one in Figure~\ref{fig-could-never}:

\begin{figure}
	\begin{forest}
		sm edges
		[VP
			[V{[\textsc{aux $+$}]}
				[could]]
			[*VP
				[Adv{[\textsc{mod} $\langle$VP$\rangle$]}
					[never]]]]
	\end{forest}
\caption{Add caption}\label{fig-could-never}
\end{figure}

Here, the adverb \emph{never} modifies a VP through the feature MOD,
which guarantees that the adverb requires the head VP that it
modifies. In an ellipsis structure, the absence of such a VP means
that there is no VP for the adverb to modify.  In other words, there
is no rule licensing such a combination -- predicting the
ungrammaticality of
*\emph{has never}\index{adverb},  as opposed to \emph{has
not}.


\subsubsection{Sentential Negation in French}


My analysis in which the negator \emph{not} and \emph{pas} are taken
to modify a nonfinite VP and select it through the head feature
MOD, provides us with a clean and simple way of accounting for
their distribution in infinitive clauses. But at stake is
their placement in finite clauses:

\eal
\ex[]{
Lee does not like Kim.
}
\ex[*]{
Lee not likes Kim.
}
\ex[*]{
Lee likes not Kim.
}
\zl

\eal
\ex[*]{
\gll Robin  ne [pas \ssub{VP[\type{fin}]}[aime  Stacey]]. \\
Robin  ne  \textsc{neg}  \hspace{35pt}likes  Stacey \\
}
\ex[]{
\gll Robin  (n')aime  pas  Stacey. \\
Robin  likes  \textsc{neg}  Stacey \\
}
\zl



\noindent
Unlike the English negator \emph{not}, \emph{pas} must follow the
finite verb. Such a distributional contrast has motivated verb
movement analyses \citep[see][]{Pollock:89,Zanuttini:01}.
%.  As noted previously, the derivational view addresses this
%variation on the basis of verb movement and the notion of functional
%projections (see section 4.2.4 for further discussion of derivational
%analyses such as that of Pollock (1989)).

By contrast, the present analysis is cast
in terms of a lexical rule that maps a finite verb into a verb
with a certain adverb like \emph{pas} as an additional complement, as
I did for English \emph{not}.  The idea of converting modifiers into
complements has been independently proposed by \citet{Miller92d-u} and
\citet{AG:94} for French adverbs including
\emph{pas} also.  Building upon this
previous work, I also assume that the modifier \emph{pas} can
be converted to a syntactic complement of a
finite verb for French via the lexical rule given
in Figure~\ref{fig:11}.\footnote{Following \citet{Miller92d-u}, I take \emph{ne} to
be an inflectional affix which can be optionally realized
in the output of the lexical rule in Modern French.}

\begin{figure}
	\begin{forest}
		[\begin{avm}
			\[\tp{neg-fin-v}\\
			form & \<(\tpv{ne})\ $+$ \@{2} \>\\
			syn|head & \[vform & fin\\
			neg & $+$\]\\
			arg-st & \<\@{1}\,XP\>\ $\oplus$ \<$\textnormal{Adv}_\textnormal{I}$\>\ $\oplus$ L \]
		\end{avm}
			[\begin{avm}
				\[\tpv{fin-v}\\
				form & \<\@{2}\>\\
				syn|head|vform & fin\\
				arg-st &  \<\@{1}\,XP\>\ $\oplus$ L\]
			\end{avm}]]
	\end{forest}
\caption{Negative Verb Construction in French ($\uparrow$\type{post-infl-cxt})}\label{fig:11}
\end{figure}

% \VAL\|\COMPS\ \q<\[\LEX\ +\\
%                   \NEG\ +\], \@1XP\q>\]\]
%\end{avm}} \medskip
%
%
%\noindent
The post-inflection construction allows us
to build a negative verb from a finite verb in French.
That is, a finite verb can give rise to a negative finite
verb that selects an Adv\ssub{I}
adverb including \emph{pas} as the second
argument.
Adv\ssub{I} includes only a small subset of French negative adverbs such
as \emph{pas}, \emph{plus} `no more', \emph{jamais} `never', and
\emph{point} `not'. This derivational construction has a semantic effect: the
negative verb taking \emph{pas} as an additional argument takes the meaning of
the input verb (\ibox{2}) as its argument.

One direct consequence of adopting this construction-based approach
is that it systematically expands the set of basic lexical entries.
For example, the construction maps lexical entries like
\emph{aime} into its negative counterpart \emph{(n')aime}, as shown
in Figure~\ref{fig:12}.

\begin{figure}
	\begin{forest}
		[\begin{avm}
			\[\tp{neg-fin-v}\\
			form & \<$\textnormal{(n')aime}$\>\\
			syn|head & \[vform & fin\\
			neg & $+$\]\\
			arg-st & \< \@{1}\,NP{,} $\textnormal{Adv}_\textnormal{I}$ \[lex & $+$\\
				neg & $+$\]{,} \@{2}\,NP\>\]
		\end{avm}
			[\begin{avm}
				\[\tpv{fin-v}\\
				form & \<$\textnormal{aime}$\>\\
				syn|head|vform & fin\\
				arg-st & \< \@{1}\,NP{,} \@{2}\,NP\>\]
			\end{avm}]]
	\end{forest}
\caption{Post-Inflection of \emph{(n')aime}}\label{fig:12}
\end{figure}

This output verb \type{neg-fin-v} then allows the negator \emph{pas} to function
as the complement of the verb \emph{naime}, as represented in Figure~\ref{fig:13}.

\begin{figure}
	\begin{forest}
		sm edges
		[VP\\
		\begin{avm}
			\[head|form & fin\\
			subj & \< \@{1}\,NP\>\\
			comps & \<~~\> \]
		\end{avm}, l sep*=1.5
			[V\\
			\begin{avm}
				\[\tpv{neg-fin-v}\\
				head|vform & fin\\
				subj & \< \@{1}\,NP\>\\
				comps & \< \@{2}\,$\textnormal{Adv}_\textnormal{I}${,} \@{3}\,NP\>\\
				arg-st & \< \@{1}\,NP{,} \@{2}\,$\textnormal{Adv}_\textnormal{I}${,} \@{3}\,NP\>\]
			\end{avm}
				[n'aime]]
			[\ibox{2}\,Adv\textsubscript{I}
				[pas]]
			[\ibox{3}\,NP
				[Stacy]]]
	\end{forest}
\caption{Add caption}\label{fig:13}	
\end{figure}



%\noindent
The analysis also explains the position of \emph{pas} in
finite clauses:

\eal
\ex[*]{
Jean  ne [pas \ssub{VP[\type{fin}]}[aime  Jan]].
} \label{56a}
\ex[]{
Jean \ssub{VP[\type{fin}]}[\ssub{V[\type{fin}]}[(n')aime] \ssub{Adv}[pas] \ssub{NP}[Jan]].
}
\zl

\noindent
The placement of \emph{pas} in (\ref{56a}) is unacceptable since
\emph{pas} here is used not as a nonfinite VP modifier, but as
a finite VP modifier. But due to the
present analysis which allows \textit{pas}-type negative adverbs
to function as the complement of a finite verb,
\emph{pas} can function as
the sister of the finite verb
\emph{aime}.
%\footnote{Of course, this
%word ordering
%conforms to the independent LP rule that a lexical head precedes
%all complements.}

Given that the conditional, imperative, and subjunctive,
and even present participle verb forms in French are finite, the
construction
analysis also predicts that \emph{pas} cannot precede any of these verb
forms:


\eal
\ex[]{
Si j'avais de l'argent, je ne ach\`{e}terais pas. \\
`If I had money, I would not buy a car.'
}
\ex[*]{
Si j'avais de l'argent, je ne pas ach\`{e}terais.
}
\zl

\eal
\ex[]{
Ne mange pas ta soupe.  \\
`Don't eat your soup!'
}
\ex[*]{
Ne pas mange ta soupe.
}
\zl

\eal
\ex[]{
Il est important que vous ne r\'{e}pondiez pas. \\
`It is important that you not answer.'
}
\ex[*]{
Il est important que vous ne pas r\'{e}pondiez.
}
\zl

\eal
\ex[]{
Ne parlant pas Fran\c{c}ais, Stacey avait des difficult\'{e}s. \\
`Not speaking French,  Stacey had difficulties.'
}
\ex[*]{
Ne pas parlant Fran\c{c}ais, Stacey avait des difficult\'{e}s.
}
\zl


Another important consequence of the present construction-based analysis
is that it allows us to reduce the parametric differences between
French and English negation to be a matter of lexical properties.
The negators \emph{not} and \emph{pas} are identical in that they both are
VP[\type{nonfin}] modifying adverbs. But they are different with respect to
which verbs can select them as complements.
A comparison between the French Negative Construction and
the English  Negative Construction shows that \emph{not} can be the
complement of a finite auxiliary verb, whereas \emph{pas} can be the
complement of a finite verb.  So the only difference
is the morphosyntactic value [\AUX\ $+$] and this induces
the difference in positioning the negators.

This surface-oriented approach is in a sense similar to
Pollock's viewpoint that the verb's finiteness plays a
crucial role in the distribution of adverbs and negation. But
there is one fundamental difference. I claim that it is not the interaction
of verb movement and his subtheories  such as the theta theory
and `quantification theory'  but the morphosyntactic value (\VFORM\
value) of the verb and lexical rules that affect the position of
adverbs including \emph{pas} and \emph{not}. All surface structures are directly
generated by X$'$ theory without movement. The conclusion we can draw from these observations
is that the position of adverbial negatives is determined not by
the respective
properties of verb movement, but by their lexical
properties, the morphosyntactic (finiteness) features of the verbal head,
and independently motivated Linear Precedence constraints.



%
%In the nonderivational analysis sketched here, the required
%notion was the independently motivated morphosyntactic feature AUX
%(motivated from NICE constructions in English and possibly from
%AUX-to-COMP and clitic climbing in old French).
%Interacting with the notion of conversion, this elementary
%morphosyntactic feature has been able to capture the
%effects of the verb's intrinsic property in determining
%the positioning of the negative markers \emph{pas} and \emph{not}.



%The key fact is that the English negative adverb \emph{not} leads a double life: one as a
%nonfinite VP modifier, marking constituent negation, and the other
%as a complement of a finite auxiliary verb, marking sentential
%negation.\index{nonfinite}\index{negation} Constituent negation
%is the name for a construction where negation combines with some
%constituent to its right, and negates exactly that constituent (see Kim and Sag 2002, Kim and Sells %2008):

%
%The English negative adverb \emph{not} leads a double life: one as a
%nonfinite VP modifier, marking constituent negation, and the other
%as a complement of a finite auxiliary verb, marking sentential
%negation.\index{VP!nonfinite}\index{negation} Constituent negation
%is the name for a construction in which negation combines with some
%constituent to its right, and negates exactly that constituent.



\subsection{Clitic-like Negative Verb}

As we have seen,  the negative markers \emph{non} and \emph{no} are
the main way of expressing negation in Italian and Spanish.
These negative markers always precede the main verb, whether finite or
non-finite, as can be observed from the repeated Italian
examples:\footnote{Languages like Welsh also employ a clitic-like
negative. See \citet{BJ:00} for detailed discussion
of Welsh negation.}



\eal
\ex[]{
\gll
Gianni non legge articoli di sintassi. \\
Gianni   \textsc{neg}   reads  articles  of  syntax \\
\trans`Gianni doesn't read syntax articles.'
}
\ex[]{
\gll
Gianni  vuole  che  io  non  legga  articoli  di  sintassi. \\
Gianni  wants  that  I   \textsc{neg}   read  articles  of  syntax. \\
}
\ex[]{
\gll
Non   leggere  articoli di sintassi   \`{e}  un vero peccato. \\
\textsc{neg}  to.read  articles of syntax   is  a real shame \\
}
\ex[]{
\gll
Non  leggendo  articoli di sintassi,  Gianni  trova  la linguistica  noiosa. \\
\textsc{neg}   reading  articles of syntax,  Gianni  finds  {} linguistics  boring \\
}
\zl


Various properties of the negator \emph{non} in Italian
and \emph{no} in Spanish are shared with those of pronominal
clitics. Like pronominal clitics, the Italian negator \emph{non} must
occupy the pre-auxiliary position:

\eal
\ex[]{
\gll
Gianni non ha parlato. \\
Gianni \textsc{neg} has talked \\
\trans `Gianni has not talked.'
}
\ex[*]{
Gianni ha non parlato. \ \ \ \citep[12]{Belletti:90}
}
\zl

However, one key difference from pronominal clitics is
that negators \emph{non} in Italian and \emph{no} in Spanish can appear alone,
especially in ellipsis-like constructions.
Consider Spanish examples from \citet{Lopez:94}.




\eal
\ex[*]{
\gll
Juan no ha comido, pero Susana ha. \\
Juan \textsc{neg} has eaten but Susana has \\
\trans `John has not eaten, but Susana has.'
}
\ex[*]{
\gll Juan ha comido, pero Susana no ha. \\
Juan has eaten but Susana \textsc{neg} has \\
}
\ex[]{
\gll Juan ha comido pero Susana no. \\
Juan has eaten but Susana \textsc{neg} \\
}
\zl

%As noted, any theory needs to capture the dual properties
%of \emph{non} (clitic and non-clitic properties). But
%neither Analysis A, nor Analysis B, nor Analysis C, appears to satisfy
%this without significant adjustments.



The derivational view again attributes the distribution of this
negative marker to the reflex of verb movement and functional
projections (see \citet{Belletti:90},
\citet{Zanuttini:91}). This line of analysis also appears to be persuasive
in that the different scope of verb movement application could explain
the observed variations among typologically and genetically related
languages. Such an analysis, however,
  fails to capture unique properties of clitic-like negators
  from inflectional negators, negative auxiliaries, or adverb negatives.

The analysis which I defend here is to take \emph{non}
to be an independent lexical head element though it is a clitic.
This claim follows the spirit of  \citet{Monachesi:93,Monachesi:98}'s analysis claiming that there are two types of clitics, affix-like
clitics and word-like clitics: pronominal clitics belong to the
former, whereas the bisyllabic clitic \emph{loro} `to-them' to the
latter. The present analysis thus suggests that \emph{non} also belongs
to the latter group.\footnote{But one main difference between
\emph{non} and \emph{loro} is that \emph{non} is a head
element, whereas \emph{loro} is a complement XP. See
\citet{Monachesi:93,Monachesi:98} for further discussion of the
behavior of \emph{loro} and its treatment.} One key difference from
pronominal clitics is thus that it functions as an independent word.
%
%Like Analysis B,
Treating \emph{non} as
a word-like element will allow us to capture its word-like
properties such as the possibility of stress on the negator and
its separation from the first verbal element. But it is not a
phrasal modifier, but a clitic, which combines with
the following main verb. Adopting the treatment of
English expressions like \emph{than, as, of, a/an, the} as functor expressions
that select a head expression \citep{Eynde:07,Sag:12}, the present analysis takes
\emph{non} as a functor, as represented by
the following lexical specifications:



\ea
Lexical specifications for \emph{non}: \\
\begin{avm}
\[\FORM\ \q<\normalfont non\q>\\
  \SYN\ \[\HEAD\|\POSP\ \type{clitic}\\
        \SEL\ \<\[\HEAD\|\POSP\ \type{verb}\\
                 \FRAMES\ \q<\@1\q>\]\>\]\\
  \SEM\ \[\FRAMES\ \<\[\type{neg-fr}\\
                       \ARGa\ \@1\]\>\]\]

\end{avm}
\z



%
%Note that the \POSP\ value of the negator.
%Its \POSP\ value is in a sense undetermined, but structure-shared
%with the \POSP\ value of its verbal complement. Its head value is thus
%determined by what it combines with. When it combines with a finite
%verb, it will be a finite verb. When it combines with an
%infinitive verb, it will be an infinitive verb.
%
%
This lexical entry roughly corresponds to the entry for
Italian auxiliary verbs (and restructuring verbs with clitic climbing),
in that the negator selects for a verbal complement. The combination
of these two expressions is licensed by the \textsc{head-functor construction}:

\ea\label{hd-functor-cxt}
\textsc{head-functor construction}:\\
\begin{myavm}\small
\[\type{hd-functor-cxt }\\
 \SEM\ \@2\]~$\Rightarrow$ \[\SEL\ \q<\@1\q>\\
                                             \SEM\ \@2\], \@1\,H[\POSP\ \type{verb}]
                                           \end{myavm}
\z

The construction allows a functor expression to combine with a head
expression, whose resulting semantics is identical to the functor. This
then licenses the combination of the clitic \emph{non} with the following
verb. In order to see how
this system works, let us consider an example where
the negator combines with a transitive verb as in
(\ref{66}).


\ea[]{
\gll Gianni non legge articoli di sintassi. \\
     Gianni \textsc{neg} reads articles of syntax \\
\glt `Gianni doesn't read syntax articles.'
}\label{66}
\z



\noindent
The negator \emph{non} combines with the finite verb \emph{legge},
whose lexical entry is given in (\ref{67}):

%\pagebreak
\ea\label{67} Lexical specifications for the word \emph{legge}:\\
\begin{avm}
\[\FORM\ \q<\normalfont legge\q>\\
  \SYN\ \[\HEAD\|\POSP\ \type{verb}\\
        \VAL\ \[\SUBJ\ \q<\@1\,NP\q>\\
              \COMPS\ \q<\@2\,NP\q>\]\]\\
  \ARG-ST\ \q<\@1\,NP, \@2\,NP\q>\\
  \SEM\ \[\FRAMES\ \<\[\type{legge-fr}\]\>\]
                       \]
\end{avm}
\z
%

This lexical construction will license the following structure, interacting
with the \textsc{head-functor construction} (cf. Figure~\ref{fig:14}).\todostefan{add glosses to tree}

\begin{figure}
	\begin{forest}
		sm edges
		[VP\\
		\begin{avm}
			\[vform & fin\\
			comps & \<~~\>\]
		\end{avm}, s sep=1cm
			[V$'$\\
			\begin{avm}
				\[\tp{hd-functor-cxt}\\
				vform & fin\\
				comps & \< \@{2}\,NP\>\]
			\end{avm}
				[V\\
				\begin{avm}
					\[sel & \< \@{1}\,V\>\]
				\end{avm}
					[non]]
				[\ibox{1}\,V\\
				\begin{avm}
					\[vform & fin\\
				 	comps & \< \@{2}\,NP\>\]
				 \end{avm}
			 		[legge]]]
			 [\ibox{2}\,NP
			 	[articoli di sintassi,roof]]]
	\end{forest}
\caption{Add caption}\label{fig:14}
\end{figure}

The \textsc{hd-functor construction} licenses the combination
of the functor \emph{non} with the following finite verb. The resulting combination inherits the subcategorization value of the head
verb \emph{legge}, but the meaning is identical
to that of the functor.

Given that a functor in Italian precedes the head it selects,
the functor treatment of \emph{non} can easily account for the
fact that the negator \emph{non} precedes an auxiliary
verb either in finite or infinitive clauses, but cannot
follow it in either clause-type .




\eal
\ex[]{
\gll
Maria   non  ha sempre  pagato  le tasse. \\
Maria  \textsc{neg} has always  paid  the taxes \\
\trans `Maria hasn't always paid taxes.'}
\ex[*]{
Maria ha sempre non pagato le tasse.  \citep[123]{Zanuttini:91}}
\zl

\eal
\ex[]{
\gll Gianni sostiene di non essere uscito.\\
Gianni claims to \textsc{neg}  have gone.out \\
\trans `Gianni claims not to have gone out.'}
\ex[*]{
\gll Gianni  sostiene   di   essere   non   uscito. \\
Gianni   claims   to   have   \textsc{neg}  gone.out \ \ \ \citep[90]{Belletti:90} \\
}
%\ex
%\gll *Gianni  sostiene   di   essere   non   uscito. \\
%Gianni   claims   to   have   \textsc{neg}  gone.out (Belletti 1990:90)\\
\zl

In the nonderivational, lexicalist analysis just sketched here,
the negator is taken to be a functor clitic that combines
with a following verb. This analysis
not only allows us to capture its dual properties -- clitic-like and
word-like properties, but also
correctly predicts the positioning of \emph{non} in various contexts.
The conclusion we can draw from Italian type of sentential negation
is that the distribution of a clitic-like negator is determined
in relation to the head that this negative selects.


\section{Concluding Remarks}

%In the previous two sections, I have provided my
%answers to the two main questions in this study.
%Then, the remaining question is what these answers
%imply for the theory of grammar.

The types of negation we have seen are identical
in that they negate a sentence or clause in the given language.
Does this entail that there is a universal functional category Neg
that, interacting with other
grammatical constraints such as movement operations, allows all their
distributional possibilities?  My answer to this question is no.



One of the most attractive consequences of the
derivational perspective has been that one uniform category,
given other syntactic operations and constraints,
explains the derivational properties of all types of negation
in natural languages, and further can provide a surprisingly
close and parallel structure among languages, whether typologically
related or not. However, this line of thinking, first of all, runs the risk of
missing the peculiar properties of each type of
negation. Each individual language has its own
way of expressing negation, and further has
its own restrictions in the surface realizations of negation which
can hardly be reduced to one uniform category.
%
%The only thing that seems to be universal about negation
%is that negation yield new sentences or clauses,
%operating on sentences or clauses.
The supposition of a uniform syntactic category
notion abstracts away only the
common denominator  (presumably a semantic notion), sweeping the
particular lexical or syntactic characteristics of each type of negation
under the carpet. This uniform NegP analysis has eventually forced
nontrivial complications of other grammatical components
in order to allow all the surface
possibilities of each type of negation. For instance,
by placing morphological negation in the realm of syntax, we
miss the fundamental generalization that the formation and
distribution of word and sentence are subject to different principles
and operations and correspond to different modules of the grammar,
namely, morphology and syntax.

%The types of negation we have observed in this research can be
%re-classified into two groups: morphological and syntactic
%negation.  For a morphological treatment of
%a negator, I have taken into consideration
%factors such as prosodic unity of negation
%and verb, morphophonemic alternation, and wordhood
%tests. For a syntactic treatment, I have taken into consideration factors
%such as prosodic independence, wordhood tests,
%carrying inflectional affixes by itself, independent occurrence in
%syntax, and  so forth.  Within the lexical integrity
%principle I defend here, this means the distribution of a morphological
%negative is regulated by morphological principles, whereas that of a
%syntactic negative marker is governed by syntax principles.
%This is what we have found.

In the nonderivational analysis,  there is no uniform
syntactic element, though a certain universal aspect of
negation does exist, viz.\ its semantic contribution.
%Each type of negation exists as a distinct category.
Languages appear to employ various possible
ways of negating a clause or sentence. Negation can
be realized as different morphological and syntactic categories.
By admitting morphological and syntactic categories,
we have been able to capture their idiosyncratic properties in a
simple and natural manner. Further this theory has been built upon
the lexical integrity principle, the thesis that the principles that govern the
composition
of morphological
constituents are fundamentally different from the principles that
govern sentence structures. The obvious advantage of
this perspective is that it can capture the distinct properties of
morphological and syntactic negation, and also of their distribution,
in a much more complete and satisfactory way.

%When compared with the derivational analyses put forth so far, it seems to
%be far more economical to discard the uniform functional category
%Neg, deep structure, and transformational component, and predict most of
%the positional possibilities of each type of negation from
%its own lexical properties and `surface structure constraints'.

One can view the difference between the derivational view
and the nonderivational, lexicalist view as a matter
of a different division of labor. In the derivational view
the syntactic components of grammars bear almost all the
burden of descriptive as well as explanatory resources.
But in the nonderivational view,  it is both the morphological
and syntactic components that carry the
burden.  It is true that a derivational
grammar whose chief explanatory resources are functional projections
including NegP and syntactic movement, also has
furthered our understanding of negation and
relevant phenomena in certain respects.
But in so doing it has also brought other complexities into the basic
components of the grammar. The present research strongly suggests
that a more conservative division of labor between morphology and syntax is
far more economical and feasible.

}%end avmoptions{center}

%\fi
{\sloppy
\printbibliography[heading=subbibliography,notkeyword=this] }
\end{document}
