\documentclass[output=paper]{langsci/langscibook} 
\author{%
	Markus Steinbach\affiliation{Georg-August-Universität Göttingen}%
	\lastand Anke Holler\affiliation{Georg-August-Universität Göttingen}%
}
\title{Sign languages}

% \chapterDOI{} %will be filled in at production

\epigram{Change epigram in chapters/03.tex or remove it there }
\abstract{Change the  abstract in chapters/03.tex \lipsum[3]}
\maketitle

\begin{document}

\section{Introduction} 
Duis pulvinar lacus id gravida ornare. Phasellus eu mauris sed tortor maximus condimentum ultrices in leo. Donec non erat nec nulla ullamcorper ornare sed id ex. Integer risus mauris, aliquet vel aliquam sed, feugiat quis nisi. Suspendisse quis nunc a turpis porttitor mollis. In luctus nulla id nunc dapibus, id rhoncus lorem pretium. Nunc eget fringilla velit, semper commodo diam. Suspendisse odio odio, euismod ac ornare sed, tincidunt ac arcu. Pellentesque vitae fringilla orci. Donec faucibus metus dui, nec iaculis purus pellentesque sit amet. Sed fermentum lorem non augue cursus, eu accumsan risus ullamcorper. Suspendisse rhoncus magna vitae enim pellentesque, eget porttitor quam finibus. Nunc ultricies turpis at quam vehicula, at tempus justo molestie. Proin convallis augue ut turpis cursus rhoncus. Donec sed convallis justo. Sed sed massa pharetra ex aliquet eleifend. 
\isi{finality} 



{\avmoptions{center}
\begin{avm} 
\[\tp{some-type}\\
feat-a & \@{10}\[\tp{type-a}\\ feat-aa & type-aa\\
feat-ab & \<\[synsem\|loc\|cat\|head & type-aba\\ feat-abc \tpv{type-abc}\],
                                  \textup{NP}\>\]\\
     feat-b & \@{10}type-b\]
\end{avm}}


\begin{figure}
\centering
\begin{forest}
typehierarchy
[\type{sign}
  [\type{word}]
  [\type{phrase} 
    [\type{non-headed-phrase}]
    [\type{headed-phrase} [\type{head-complement-phrase}]]]]
\end{forest}
\caption{\label{fig-type-sign}Type hierarchy for \type{sign}}
\end{figure}%

 
\section*{Abbreviations}
\section*{Acknowledgements}

\printbibliography[heading=subbibliography,notkeyword=this] 
\end{document}



Inf marker as Argument, Partikle verbs??

PP as adjunct?

haben vs. sein

wenn das pro-synsem ist, wieso kann man es dann realiisieren?

Das Buch ist bis Montag zu lesen. Kopula ist Anhebung      Subjektlose Prädikate???

Unpersönliche Konstruktionen

Morgen ist zu arbeiten.



obligation/omission/ability  Gelhaus 1977



32 Einmal Relation als Typ, einmal nicht

37 funktioniert nicht.
